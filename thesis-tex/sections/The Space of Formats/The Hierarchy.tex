\sub {

    Up until now, we defined and learned about various specific tournament formats the properties that they have. This is a sort of bottom-up approach. In this chapter, we will examine the space of tournament formats form the top down, and see if we can discern properties about this space that aren't as specific to any particular design.

    In particular, to make this analysis a little more tractable, we will look a particular subset of tournament formats: winner-take-all formats.

    \begin{definition}{Winner-Take-All}{}
        A \textit{winner-take-all} tournament format is a tournament format that outputs only a single champion $t \in \T.$
    \end{definition}

    Despite the restriction, there is still a vibrant array of winner-take-all formats: traditional brackets, reseeded brackets, randomized cohort seedings, and various round-robins (pure or otherwise) are some examples that we've seen so far. In fact, any arbitrary format can be converted to winner-take-all by just playing out the format and then only reporting the champion (and, say, tf there are co-champions, selecting one randomly.)

    One question that one might ask is, for various $n$, how many $n$-team winner-take-all formats are there?

    \theo{}{
        There are no zero-team winner-take-all formats, and one one-team winner-take-all format.
    }{
        Any winner-take-all format must declare a champion $t \in \T$, so if $\T$ is empty, there can be no such format. Meanwhile, if there is only a single team, no games can be played, and only a single team can be declared champion.
    }{}

    This one-team format is goes by many names: $\bracksig{1}$ and the one-team round robin are two.

    How many two-team formats are there? We enumerate a bunch of them, based on how they act on input $\T = [t_1, t_2]$:
    \begin{itemize}
        \item Play no games, declare $t_1$ champion.
        \item Play no games, declare $t_2$ champion.
        \item Play one game, declare the winner of that game champion.
        \item Play one game, declare the loser of that game champion.
        \item Play one game, then no matter who wins declare $t_1$ champion.
        \item Play three games, declare the team that wins two or more champion.
        \item Play three games. If $t_1$ wins all three, declare $t_1$ champion, otherwise declare $t_2$ champion.
        \item Play 23 games. If $t_1$ wins a prime number of games, declare them champion. Otherwise, run the algorithm from the previous bullet.
        \item Play no games. Flip a coin. If heads, return declare $t_1$ champion, if tails declare $t_2$ champion.
    \end{itemize}

    Many of these tournaments aren't very good, for various reasons, but they are all legitimate formats. Clearly then,

    \theo{}{
        For $n \geq 2$, there is an infinite number of $n$-team winner-take-all formats.
    }{
        Let $n \geq 2$. Then consider the class of formats $\A_i$, in which $t_1$ plays $t_2$ $i$ times and whoever won more games is declared champion (ties going to $t_1$.) There are an infinite number of such formats (one for each $i \in \N$).
    }{}




}