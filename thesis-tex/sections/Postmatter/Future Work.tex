\sub {

    There are four areas of continued research that we are interested in pursuing.

    Firstly, we would like to answer the two open questions presented at the end of Chapter 2: for all $r$, does there exist an $r$-round deterministic ordered balanced knockout tournament? And for all $r$, does there exist an $r$-round dramatic ordered balanced knockout tournament? Of course finding such formats would be ideal, but we think an impossibility theorem is more likely.
    
    Secondly, we would like to expand on the study of multibrackets in Chapter 3. Some areas in particular include: the expansion of the degrees of properness that linear multibrackets can exhibit, the count and enumeration of multibrackets that are efficient with respect to a given prize structure, an analysis of which Swiss signature to use to select a top-$m$ out of $n$ teams for arbitrary $m$ and $n$, and the extension of signatures and properness to nonlinear multibrackets.

    Thirdly, we would like to expand our analysis to formats that are not networked: round robins and pool-based formats in particular. Additionally, we are interested in the analysis of the space of formats as a whole, working from the top-down by defining universal properties and observing which formats uphold them, rather than continuing to define specific formats for analysis.

    Fourthly, we would like to develop statistical models to analyze how fair and accurate tournament formats are, in order to distinguish between multiple ordered (or not ordered) formats. In particular, we are interested in finding an algorithm that returns the most fair and accurate format that is efficient with respect to a given prize structure for use at club sport regional tournaments.
}