\sub {

    There are four areas of research in the field of tournament design that we hope to pursue in the future.

    The first are answers to the two open questions presented at the end of Chapter 2: for all $r$, does there exist an $r$-round deterministic ordered balanced knockout tournament? And for all $r$, does there exist an $r$-round dramatic ordered balanced knockout tournament? Of course, finding such formats would be ideal, but we think an impossibility theorem is more likely.
    
    The second is a continuation of the study of multibrackets. Some areas in particular include: defining a full range of the degrees of properness and respectfulness that linear multibrackets can exhibit, counting the number of multibrackets that are efficient with respect to a given prize structure, determining which Swiss signature to use to select a top-$m$ out of $n$ teams for arbitrary $m$ and $n$, and extending the notions of signatures and properness to nonlinear multibrackets.

    The third is an expansion of our analysis to formats that are not networked: round robins and pool-based formats in particular. Additionally, we are interested in a treatment of the space of formats as a whole, working from the top down by defining universal properties and observing which formats uphold them, rather than continuing to define specific formats for analysis.

    And the fourth is a statistical model to measure how fair and accurate tournament formats are, allowing us to distinguish between multiple ordered (or unordered) formats. In particular, we are interested in an algorithm that returns the most accurate format that is efficient with respect to a given prize structure for use at club sport regional tournaments.
}