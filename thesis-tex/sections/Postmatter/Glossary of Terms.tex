\sub {

%B
    \glossbalanced
    \glossbracket
    \glossbracketSignature
    \glossbye
%C
    \glosschalk
    \glosscohortRandomized
    \glosscontainment
    \glossTHcontainment
%D
    \glossdet
%E
    %\glossTHedwards
    \begin{tcolorbox}[title={
        Theorem \ref{th:edwards}: 
        Edwards's Theorem  \hspace*{\fill} (Edwards, 1991, p. \pageref{th:edwards})
        },enhanced, breakable, colback=purple!5, colframe=violet!100!,fonttitle=\bfseries]
        The set of ordered brackets is exactly the set of proper brackets whose signature is formed by the following process:
        \begin{enumerate}
            \item Start with the list $\bracksig{0}$ (note that this not yet a bracket signature).
            \item As many times as desired, prepend the list with $\bracksig{1}$ or $\bracksig{3; 0}.$
            \item Then, add 1 to the first element in the list, turning it into a bracket signature.
        \end{enumerate}
    \end{tcolorbox}

    \glossexcitingBracket
    \glossexcitingKnockout
%F
    \glossTHfundamental
%G
    \glossgameplay
%H
    \glosshigherLower
%I 
    \glossiseed
%K
    \glossknockout
    \glossknockoutSignature
%M
    \glossmatchup
    \glossmonotonic
%N
    \glossnetworked
%O
    \glossordered
%P    
    \glosspij
    \glossplaying
    \glossproperBracket
    \glossproperKnockout
    \glossproperSeed
%R
    \glossround
    \glossreseeded
%S
    \glossseeding
    \glossshape
    \glossTHstapling
    \glossstarting
    \glosssst
%T
    \glosstotallyRandomized
    \glosstournament
%W
    \glosswatt

    
    
   
    
    
   
    






    %1.2definitions?
    %theorems (big4)
}