\sub{

\begin{definition}{Round Robin}{rrdef}
    A \textit{round robin} is a tournament format in which each team plays each other team once, and then teams are ranked according to how many games they won.
\end{definition}

Round robins, or close variants, are used in many leagues across many sports, especially during the regular season or qualifying rounds. For example, the 2014 Ivy League Football Regular Season was structured as round robin. At the conclusion of a round robin, a league table can be used to display the results and rank the teams.

\begin{figg}{2014 Ivy League Football Regular Season}{}
    \centering
    \begin{tabular}{| c | c | c | c | c |}
        \hline
        Rank & Team & Games & Wins & Losses\\ \hline
        1 & Harvard & 7 & 7 & 0\\ \hline
        2 & Dartmouth & 7 & 6 & 1\\ \hline
        3 & Yale & 7 & 5 & 2\\ \hline
        4 & Princeton & 7 & 4 & 3\\ \hline
        5 & Brown & 7 & 3 & 4\\ \hline
        6 & Penn & 7 & 2 & 5\\ \hline
        7 & Cornell & 7 & 1 & 6\\ \hline
        8 & Columbia & 7 & 0 & 7\\ \hline
    \end{tabular}
\end{figg}

At the end of an $n$-team round robin, each team has played each other team once, for a total of $n-1$ games. There are $n$ possible records a team could have after playing $n-1$ games, so it is possible for each team to end the tournament with a different record: the 2014 Ivy League Football Regular Season has this property.

However, this is far from guaranteed: consider the 2019 Big 12 Football Regular Season (strangely enough, in 2019 the Big 12 had only ten teams).

\begin{figg}{2019 Big 12 Football Regular Season}{}
    \centering
    \begin{tabular}{| c | c | c | c | c |}
        \hline
        Rank & Team & Games & Wins & Losses\\ \hline
        1 & Oklahoma & 9 & 8 & 1\\ \hline
        2 & Baylor & 9 & 8 & 1\\ \hline
        3 & Texas & 9 & 5 & 4\\ \hline
        4 & Oklahoma State & 9 & 5 & 4\\ \hline
        5 & Kansas State & 9 & 5 & 4\\ \hline
        6 & Iowa State & 9 & 5 & 4\\ \hline
        7 & West Virginia & 9 & 3 & 6\\ \hline
        8 & TCU & 9 & 3 & 6\\ \hline
        9 & Texas Tech & 9 & 2 & 7\\ \hline
        10 & Kansas & 9 & 1 & 8\\ \hline
    \end{tabular}
\end{figg}

Nearly every team, including the two leaders Oklahoma and Baylor, ended the season tied with at least one other team. To rank teams with the same record, a \textit{tiebreaking algorithm} is used.

\begin{definition}{Tiebreaking Algorithm}{}
    A \textit{tiebreaking algorithm} is an algorithm for ranking teams that finish with the same record at the conclusion of a round robin.
\end{definition}

Since every team in the 2014 Ivy League Football Regular Season ended with a different record, the tiebreaking algorithm wasn't employed: no matter what algorithm the Ivy League had prescribed, the ranking would have been the same. Not so for the 2019 Big 12 Football Regular Season, where many teams ended the season with identical records: different tiebreaking algorithms might have resulted in different rankings of the tied teams. (Though of course, Oklahoma and Baylor would always be first and second in some order, Texas and the three States would always be third through sixth in some order, etc.)

\begin{definition}{Tiebreaker}{}
    A \textit{tiebreaker} is a single statistic that can be used to compare teams that finish with the same record at the conclusion of a round robin. Tiebreakers need not be able to successfully generate an order for any given set of tied teams.
\end{definition}

Most tiebreaking algorithms are composed of a sequence of individual \textit{tiebreakers}. These tiebreaker are applied one-by-one: if the first tiebreaker successfully breaks the tie, then the algorithm is complete. Otherwise, we proceed to the next tiebreaker.

Although individual tiebreakers are not required to be able to break all possible ties, the tiebreaking algorithm is. Thus, that last tiebreaker (and only the last tiebreaker) in a tiebreaking algorithm must be \textit{terminal}.

\begin{definition}{Terminal Tiebreaker}{}
    A tiebreaker is \textit{terminal} if it is guaranteed to generate an order for a set of tied teams.
\end{definition}

\begin{definition}{\tbreak{PointsScored}}{}
    The \tbreak{PointsScored} tiebreaker ranks teams by how many points they scored over the course of the round robin: the more points, the better.
\end{definition}

\tbreak{PointsScored} is not terminal: any number of tied teams might have scored the same number of points over the course of the round robin and thus would remain tied.

\begin{definition}{\tbreak{Random}}{}
    The \tbreak{Random} tiebreaker ranks teams randomly: with each ordering being equally likely.
\end{definition}

\tbreak{Random} is terminal, and in fact, \tbreak{Random} is used as the last tiebreaker in the tiebreaking algorithm of many leagues.

On the other hand, many leagues' first tiebreaker is \tbreak{HeadToHead}.

\begin{definition}{\tbreak{HeadToHead}}{}
    The \tbreak{HeadToHead} tiebreaker ranks teams by their record against the other tied teams: the more wins, the better.
\end{definition}

\theo{}{
    \tbreak{HeadToHead} is terminal as a two-team tiebreaker, but not terminal for more than two teams.
}{
    If only two teams are tied, then their record against each other must be 1-0 and 0-1, in some order, and so  \tbreak{HeadToHead} will successfully break the tie.\\
    
    If $n$ teams are tied for $n \geq 3,$ then let $t_1, t_2,$ and $t_3$ be three of the tied teams, and consider the situation where $t_1$ beat $t_2$, $t_2$ beat $t_3$, $t_3$ beat $t_1$, and all three of $t_1, t_2,$ and $t_3$ beat every other tied team. Then each of $t_1, t_2$, and $t_3$ will have $(n-2)$ wins and one loss against tied teams, so \tbreak{HeadToHead} cannot break their tie and thus is not terminal.
}{}


}