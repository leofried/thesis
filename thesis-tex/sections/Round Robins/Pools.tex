\sub {

    Consider the 2023 FIFA Women's World Cup. 32 teams qualified for the World Cup, and they were divided into eight pools of four teams each. Each pool played out a self-contained round robin. Then, the bottom two teams in each pool were eliminated, and the remaining sixteen teams (two from each pool) played out the multibracket $\bracksig{16;0;0;0;0} \to \bracksig{1} \to \bracksig{2;0}.$ The entire format as it played out is displayed in Figure \ref{fig:world_cup}.

    \begin{figg}{The 2023 FIFA Women's World Cup}{world_cup}\end{figg}

    Note that the seeds in the primary semibracket $\bracksig{16;0;0;0;0}$ are not just the numbers $1$ through $16$, but instead letter-number combination. $\bracklabel{B2}$ indicates that the second-place team from pool $\bracklabel{B}$ should be placed in that location. To avoid confusion, we start at $I$ when labeling the rounds. In alignment with the discussion in Section XYZ, the 2023 FIFA Women's World Cup attempts to avoid rematches, ensuring that the two teams that qualify to bracket from each pool can only meet in the finals.
    %section xyz

    The format in Figure \ref{fig:world_cup} combines the two genres of format we've discussed thus far: the first half is a (bunch of separate) round robins, and the second half is a multibracket. How teams perform in the round robin phase of the format affects how they are seeded (and whether they qualify at all) in the second half. We call such formats \textit{pool to bracket formats.}

    \begin{definition}{Pool to Bracket}{}
        A \textit{pool to bracket} is a tournament format in which teams are divided amongst a number of pools, each pool plays a round robin, and then the results of the round robins are used to place (some) of the teams into a multibracket which is then played out to get the final placements.
    \end{definition}

    Pool to bracket formats make use to the strengths of both round robins and multibrackets in order to cover for each others weaknesses. Round robins are maximally faithful, ensuring that unlucky draws don't screw teams out of top-$m$s for any $m$, but they take $O(n^2)$ games to complete, and they aren't very exciting to watch: there is no championship game, and often times games are played in which one or both teams have already been eliminated from placing (or have already secured a particular finish). Meanwhile, multibrackets can be completed in $O(n\log n)$ games or even $O(n)$ games, and are often incredible exciting, with every game in an (efficient) multibracket being played for stakes, but struggle to be faithful: a poor first-round matchup can significantly hinder an otherwise deserving teams ability to finish highly.
    
    Pool to bracket formats use ...

    Because of this, pool to bracket formats are nearly ubiquitous across sporting leagues. Both the Men's and Women's FIFA World Cup use pool to bracket formats, as well as many olympic sports. Most club sport regional and national tournaments also use pool to bracket formats. Even many NCAA sports use a format very similar to pool to bracket: colleges are divided into conferences, play something resembling a round robin with the teams in their conferences, and then proceed to the post-season which is a bracket that mixes together teams from various conferences. Professional leagues such as the NFL can even be looked at through this lense as well.


    %signatures / notation
    %this can be used to get major faithfulness wins (regionals)
}