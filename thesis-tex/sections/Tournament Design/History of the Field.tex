\sub { 
    
    The formal theory of tournament design was born out of the study of paired comparisons, a field that began in 1927 with Thurstone's \i{A Law of Comparative Judgement} \cite{thurstone}. Thurstone was a psychologist investigating how individuals rank a collection of objects on some axis (weight, beauty, excellence, etc), while only being able to examine two of the objects at a time. The similarity to the problems of tournament design is clear, though Thurstone did not draw the connection.

    In 1963, David wrote \i{The Method of Paired Comparisons} \cite{stochastic}, aiming to gather all of the theory that had been developed about paired comparisons thus far, as well as several contributions of his own, into a single monograph. At the time of publication, the field was still viewed through Thurstone's psychological lens, rather than the lens we will use involving teams competing in a game or sport. Where we will say ``two teams play a game,'' David said ``a single judge must choose between two objects.'' Still, the formalizations of the problems are equivalent.

    In the years following David's work, the field of tournament design came into its own, with various authors examining the fairness and accuracy of a wide range of tournament formats, most commonly either round robins or brackets. Much of the work at the time, however, was an analysis of what happens when a specific set of teams (or a narrow class of sets of teams) takes part in a specific tournament design, rather than anything more general.

    In 1991, Edwards submitted his doctoral thesis, \i{The Combinatorial Theory of Single-Elimination Tournaments} \cite{montana}, the single most complete analysis of brackets that has been published to date. Edwards counted and cataloged the full space of brackets, defined \i{orderedness}, which we will soon see is a natural and desirable property, and completely determined which brackets are ordered. Edwards's Theorem, after which Section \ref{sec:Edwards's Theorem} is named, was first proved in that thesis.

    Since then, the field has become much more statistical, with most of the analysis being done by way of Monte Carlo simulations. Dabney's \i{Tourney Geek} \cite{geek}, for example, evaluates various tournament designs based on several different statistical measurements of fairness that are estimated via simulation.
    
    In this thesis, we return to the type of study conducted by Edwards: proofs of claims about the outcomes of various tournament designs, rather than statistical results. We will work from first principles, beginning with the definition of a game and a tournament format, constructing various classes of formats, and then examining those formats and the properties they might have. Like most studies in the field of tournament design, we are \i{game-ambivalent}. We abstract away the underlying game or sport: our results apply to football as well as they will to chess as well as they will to competitive rock-paper-scissors.

    In this way, the theory of tournament design and the theory of sorting algorithms are quite similar: the types of questions posed in the fields are nearly identical as well. In both cases, the designer is given a list of objects (teams), may make an arbitrary number of comparisons (games), and then must output a sorting (champion). There are, however, a number of differences that separate the fields.

    \;

    The first difference is that of noise. The sorting theorist works with the guarantee that if two objects are compared more than once, the comparison will give the same result every time. For this reason, the sorting theorist often finds it wasteful to compare the same pair of objects more than once. But the tournament theorist's job is much harder, as team performance is noisier. When two teams play, there is no guarantee that the better team will win, and when they play more than once, there is no guarantee that the same team will win every time.

    The second difference is that of accuracy. An algorithm submitted by the sorting theorist is required to correctly sort any list of objects, otherwise it is not a sorting algorithm. The tournament theorist is under no such constraints: the noise makes such an algorithm impossible. Thus algorithms like ``randomly select a winner'' and ``play lots of games and then declare as champion the team with the fewest wins'' are valid tournament designs, even if they are (probably) not particularly good ones.

    The third difference is that of priors. While the sorting theorist typically begins their algorithms with no priors on the set of objects, the tournament theorist is often given a ``seeding'' of teams, identifying which teams have been judged as better. This seeding can be varyingly accurate: in some cases, the tournament theorist begins their algorithms with very strong priors, while in others, the seeding provides minimal information.

    The fourth difference is that of fairness. The sorting theorist is working with a set of lifeless objects whose feelings will not be hurt based on the algorithm, freeing the sorting theorist to focus only on the task of accurately sorting the objects. The tournament theorist, on the other hand, must appeal to the sense of fairness held by the competitors: in many cases, fairness is a more important consideration than accuracy. 
    
    The final difference is that of viewership. The sorting theorist works in private, comparing objects and gathering data until a sort can be published. The tournament theorist, on the other hand, works in front of an audience, who are looking not just for an accurate tournament, but for an exciting one: the NCAA College Basketball Tournament is a classic example, as we will soon see, of a tournament that is not very accurate but none the less very exciting for viewers.

    Still, there is a lot of overlap between the two fields. Knuth's \i{The Art of Computer Programming: Sorting and Searching} \cite{knuth} often used the language of teams and games when presenting various sorting algorithms. We borrow from the field of sorting in turn: in particular the concept of a \i{sorting network}.

    Sorting networks, first patented by Armstrong, Nelson, and O'Connor \cite{pat}, are sorting algorithms with the additional property that, after a comparison is made between $a$ and $b$, the rest of the algorithm is identical no matter the result, except that $a$ and $b$ are swapped. Knuth's text contains a section about the space and properties of sorting networks.

    This thesis will primarily examine networked tournament formats, that is, tournament formats with this networking property. These formats are a particularly nice set of formats to study, both because the networking property turns out to be a powerful one, and because many formats used in the postseason of leagues are networked, giving our study applications to many tournaments across many sports.
}



