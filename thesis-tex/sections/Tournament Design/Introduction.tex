\sub {
   
    Although tournaments have been in use for as long as humans have played sports, their formal study is underdeveloped, with many key questions in the field remaining open. In this thesis we aim to address five big open questions in the field, and we begin by focusing on the \i{bracket}: a tournament format in which teams are eliminated upon a loss, games are played until one team remains, and the matchups between game winners are determined in advance of any games being played.

    {\bf Is there a succinct notation for describing brackets?} The space of brackets is quite large, and fully drawing out a given bracket can be quite time-consuming, difficult to quickly interpret, and nearly impossible to read.  We introduce the \i{bracket signature}, a new system for compressing an arbitrary bracket into a single succinct list of digits, allowing for brackets to be easily communicated and important properties of brackets to be easily verified.

    {\bf Which brackets are fair?} We construct the notion of a \i{proper bracket}, with a host of desirable fairness properties, noting that nearly all brackets in use by leagues around the world are proper. We then develop insight into the order in which games are played and teams are eliminated in a proper bracket. Finally, we prove the \i{fundamental theorem of brackets}: there is exactly one proper bracket with each bracket signature.

    {\bf Which brackets are accurate?} Edwards's \cite{montana} answered this question in 1991 using a measure of accuracy he called \i{orderedness}, but at the time the previous two questions were still open. With properness and signatures now defined, as well as the fundamental theorem proved, we present a much simpler and more direct proof of the answer and establish a few other generalizable results along the way. Unfortunately, Edwards's Theorem is quite pessimistic about the number of ordered brackets: even the standard eight-team bracket is not ordered, leading to the next question.

    {\bf Are there other bracket-like formats that are ordered?} Hwang \cite{reseeding} published a proof that \i{reseeded brackets} are ordered, but we show that his proof was incorrect. After an analysis of a few other options, we conjecture that any bracket-like format that is ordered for any number of teams lacks several other key properties including determinism.

    {\bf How can brackets select runners-up?} We propose the notion of a \i{multibracket}, a generalization of brackets that takes a wide variety formats (including, but not limited to, \i{consolation brackets}, \i{semibrackets}, \i{linear multibrackets}, \i{swiss systems}, \i{Page-McIntyre systems}, and \i{double-elimination}) that have been studied independently and unifies them under a single umbrella. We use this framework to prove several key results about the number of such formats, their efficiency, and their accuracy, and discuss how a tournament designer might select which format to use.

    Over the course of this work, we analyze 16 tournament formats in use by leagues across the globe, as well as 42 constructed for the purpose of analysis. We prove 37 theorems, of which 28 are novel (including one that disproves Hwang's theorem), and define 31 new terms. We hope this thesis will aid leagues in designing more effective tournaments, as well as serve as a jumping off point for future work in the field of tournament design.
}


