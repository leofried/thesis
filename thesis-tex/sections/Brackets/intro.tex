\sub{
\begin{definition}{Bracket}{}
    A \textit{bracket} is a tournament format in which:
    \begin{itemize}
        \item Teams don't play any games after their first loss,
        \item Games are played until only one team has no losses, and that team is crowned champion, and
        \item The matchups between teams that have not yet lost are determined based on the ordering of the teams in $\T$ in advance of the outcomes of any games.
    \end{itemize}
\end{definition}

We can draw brackets as a tree-like structure in the following way:

\fig{1}{cfp_start}{The 2023 College Football Playoff}

The numbers 1, 2, 3, and 4 indicate where the first, second, third and fourth team in $\T$ are placed to start. In the actual 2023 College Football Playoff, the list of teams $\T$ was Georgia, Michigan, TCU, and Ohio State, in that order, so the bracket was filled in like so:

\fig{1}{cfp_middle}{The 2023 CFP After Team Placement}

As games are played, we write the name of the winning teams on the corresponding lines. This bracket tells us that Georgia played Ohio State, and Michigan played TCU. Georgia and TCU won their respective games, and then Georgia beat TCU, winning the tournament.

\fig{1}{cfp_end}{The 2023 CFP After Completion}

Rearranging the way the bracket is pictured, if it doesn't affect any of the matchups, does not create a new bracket. For example, Figure \ref{fig:cfp_alt} is just another way to draw the same 2023 CFP Bracket. 

\fig{1}{cfp_alt}{Alternative Drawing of the 2023 CFP}

One key piece of bracket vocabulary is the \textit{round}.

\begin{definition}{Round}{}
    A \textit{round} is a set of games such that the winners of each of those games have the same number of games remaining to win the tournament.
\end{definition}

For example, the 2023 CFP has two rounds. The first round included the games Georgia vs Ohio State and Michigan vs TCU, and the second round was just a single game: Georgia vs TCU.
}

%Summary of future sections?
