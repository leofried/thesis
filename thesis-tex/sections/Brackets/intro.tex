\sub{





The next format that we will study is the bracket. Brackets are based on binary trees, but we use some different langauge.

\begin{definition}{Line}{}
    A \textit{line} is a node.
\end{definition}

\begin{definition}{Starting Line}{}
    A \textit{starting line} is a leaf node.
\end{definition}

\begin{definition}{Game Line}{}
    A \textit{game line} is a non-leaf node.
\end{definition}

\begin{definition}{Input Line}{}
    An \textit{input line} of a game line is one of its two direct children.
\end{definition}

\begin{definition}{Championship Line}{}
    The \textit{championship line} is the root node.
\end{definition}

\begin{definition}{Round}{}
    A \textit{round} is a level of a binary tree. Rounds are zero indexed and start from the lowest level. (So the first round is the lowest level with game lines, and the last round is the level containing only the championship line.)
\end{definition}

With the new lexicon in place, we can introduce the format.

\begin{definition}{Bracket}{}
    An $n$-team \textit{bracket} is a format defined by a binary tree $\B$ with $n$ starting lines labeled $1$ through $n$. The format proceeds as follows:
    \tcblower
    \begin{algorithmic}[1]
        \Require {$\T$}
        \ForEach {$i$ in $\{1, ..., n\}$}
            \State {Write $\T[i]$ on the starting line labeled $i$}
        \EndFor
        \ForEach {round $r$ in $\B$}
            \ForEach {game line $g$ in $r$}
                \State {$\ell_1, \ell_2 \gets$ the two input lines of $g$}
                \State {$t_1 \gets$ the team written on $\ell_1$}
                \State {$t_2 \gets$ the team written on $\ell_2$}
                \State {Play $t_1$ vs $t_2$}
                \State {Write the winner of $t_1$ vs $t_2$ on $g$}
            \EndFor
        \EndFor
        \Ensure {The team written on the championship line.}
    \end{algorithmic}
\end{definition}


% \sub{

% \begin{definition}{Bracket}{}
%     A \textit{bracket} is a tournament format in which
%     \begin{itemize}
%         \item Teams don't play any games after their first lost.
%         \item Games are played until only one team has no losses.
%         \item The matchups between game winners are determined in advance of the outcomes of the games.
%     \end{itemize}
% \end{definition}

% We can draw brackets in a tree-like structure in the following way:

% \fig{1}{cfp_start}{The 2023 College Football Playoff}

% As games are played, we write the name of the winning teams on the corresponding lines. In the 2023 CFP, Georgia played Ohio State, and Michigan played TCU. Georgia and TCU won their respective games, and then in the second round Georgia beat TCU, winning the tournament.

% \fig{1}{cfp_end}{The 2023 CFP After Completion}

% A few vocab terms:


%Summary of future sections?

}

%Summary of future sections?

}