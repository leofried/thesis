\sub {

    One common application of multibrackets is when an $n$-team tournament is needed to select the top $m$ teams to move on to the next stage of the competitions (perhaps this is the regional tournament, and the top $m$ teams from this region qualify for nationals.) For a tournament such as this, a natural option would be an $m$-team multibracket of order $n$. This is exactly what the USA Ultimate Manual of Championship Series Tournament Formats \cite{ultimate} calls for in many circumstances. One such example is the following eight-team order four simple multibracket:

    \fig{0.8}{ultimate_format}{$\bracksig{8;0;0;0} \to \bracksig{4;2;0;1;0} \to \bracksig{1} \to \bracksig{1}$}

    One observation about this format is that a few of the games played seem unnecessary. In particular, games $\bracklabel{F1}$ and $\bracklabel{G1}$ are downright silly: after round $\bracklabel{E}$ the four teams that advance to nationals are already set: the winner and loser of $\bracklabel{C1}$, and the winners of the two round $\bracklabel{E}$ games. Perhaps more subtly, game $\bracklabel{C1}$ is also unnecessary: both the winner and loser will advance. A more efficient alternative might look like so.

    \fig{0.8}{ultimate_efficient}{A More Efficient Alternative}

    The format in Figure \ref{fig:ultimate_efficient} isn't an example of anything we've defined thus far. It's almost a simple multibracket but not quite: for one thing, some teams start in the ``second bracket.'' To formally describe what is going in Figure \ref{fig:ultimate_efficient}, we introduce the notion of semibracket.

    \begin{definition}{Semibracket}{}
        A \textit{semibracket} is a tournament format in which:
        \begin{itemize}
            \item Teams don't play any games after their first loss,
            \item The matchups between teams that have not yet lost are determined based on the ordering of the teams in $\T$ in advance of the outcomes of any games.
        \end{itemize}
    \end{definition}

    \begin{definition}{Order of a Semibracket}{}
        The \textit{order} of a semibracket is the number of teams that finish the semibracket with no losses.
    \end{definition}

    This is the same definition of a bracket but without the requirement that games be played until only one team is without losses. Semibrackets of order one are just traditional brackets.
    
    Under this frame, the format in Figure \ref{fig:ultimate_efficient} is composed of two semibrackets.



}