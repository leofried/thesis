\sub{

    \begin{definition}{Semibracket}{}
        A \textit{semibracket} is a set of brackets that are played concurrently such that no team plays in more than one bracket.
    \end{definition}

    \begin{definition}{Order}{}
        The \textit{order} of a semibracket is the number of brackets it contains.
    \end{definition}

    All brackets are already semibrackets of order one. You can easily transform a bracket into a semibracket of order two by not playing the championship game.

    \fig{1}{[8;0;0].png}{$[[8;0;0;0]]$ as a Semibracket of Order Two}{}
    
    Likewise, a bracket can be transformed into a semibracket of order four by not playing the semifinals or the finals. Semibrackets can be somewhat pathological like this semibracket of order four.

    \fig{1}{pathalogical_semibracket.png}{Pathological Semibracket of Order Four}{pathological}

    In the semibracket in Figure \ref{fig:pathological}, three teams must play no games to win their repesctive brackets, while six other teams compete to win the final bracket.

    The empty semibracket is the unique semibracket of order zero.

    \begin{figg}{The Empty Semibracket}{}
    \end{figg}

    The ideas of properness and signatures that we used to discuss brackets are also useful when applied to the space of semibrackets, so we port them over.

    \begin{definition}{Proper Semibracket}{}
        A \textit{proper semibracket} is a semibracket that has been properly seeded.
    \end{definition}

    \begin{definition}{Semibracket Signature}{}
        The \textit{signature} of a semibracket is a list of integers of length $r + 1$, where $r$ is the number of rounds in that semibracket, such that the $i$th position in the list denotes the number of teams that play their first game in that round.
    \end{definition}

    We omit the proofs for brevity, but all of the important theorems about proper brackets and bracket signatures have semibracket analogues.
}