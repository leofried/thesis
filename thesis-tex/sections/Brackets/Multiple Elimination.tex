\sub{
    While the bracket is a very powerful and important tournament design, it has two intimately related shortcomings. First, brackets lead teams to play wildely different numbers of games: in the bracket $\bracksig{8; 0; 0; 0}$, for example, two teams will play three game, two teams will play two, and four teams will play only one.

    And second, brackets tend to do a very poor job of ranking teams beyond just selecting a winner. Again considering the bracket $\bracksig{8; 0; 0; 0}$, the first-place finish is obviosly the winner of the bracket, and we can easily grant the loser of the championship game second-place, but the two semifinals losers both might have a claim to the third-place, and sorting through the fifth- through eight-place finishes is even trickier.
    
    These problems are reflections of each other: the reason that ranking the lower-placing teams is so hard is because they play so few games. It's easier to differentiate between first and second because both teams have played three games, but differentiating between the four teams that have played only a single game is nigh impossible.

    In some cases, these problems aren't issues: perhaps we are only interested in crowning a champion and don't care about exactly who came in third, or maybe this bracket is being played at the conclusion of a long season and so teams playing variable numbers of games is not a big deal. But the two problems interconnected natures lets us solve them together, leveraging the extra games that lower-ranked teams have left in order to rank them.

    The most common such system is a third-place game, in which the losers of the two semifinal game play each other for third place. The 2015 Asian Football Confederation Asian Cup, which uses a $\bracksig{8; 0; 0; 0}$ uses a third-place game.

    \fig{.7}{third_place_game.png}{2015 AFL Asian Cup}{asian_cup}

    A few notes on the figure. First, there are no seeds: brackets don't have to be seeded if the teams are assigned to there spots randomly (in some ways this is a special case of a Randomized Proper Ordered Bracket). This is particular common for balanced brackets such as $\bracksig{8; 0; 0; 0}$, as no teams given an unfair advantage of a bye without having earned it with higher seeding.

    Second, the bracket games are labeled. Games in the first round are $\bracklabel{A1}$ through $\bracklabel{A4}$, while the semifinals are $\bracklabel{B1}$ and $\bracklabel{B2}$, and the finals is game $\bracklabel{C1}$. The actual labeling scheme is arbitrary, but we adopt the system used in Figure $\ref{fig:asian_cup},$ in which each round is given a letter and then games are numbered from the top to bottom of the bracket within each round. Note that the third-place game is labeled $\bracklabel{D1}$: even though it could be played concurrently to the championship game, we label it as a different round.

    Finally, we indicate that the third-place game is to be played in between the losers of games $\bracklabel{B1}$ and $\bracklabel{B2}$ by labeling the starting lines in the third-place game with those games. This is not ambiguous because the winner of those games always just continues on in the original bracket, so those labels refer to the losers.
}