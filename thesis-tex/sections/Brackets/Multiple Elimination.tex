\sub{
    While the bracket is a very powerful and important tournament design, it has two intimately related shortcomings. First, brackets lead teams to play wildely different numbers of games: in the bracket $\bracksig{8; 0; 0; 0}$, for example, two teams will play three game, two teams will play two, and four teams will play only one.

    And second, brackets tend to do a very poor job of ranking teams beyond just selecting a winner. Again considering the bracket $\bracksig{8; 0; 0; 0}$, the first-place finish is obviously the winner of the bracket, and we can easily grant the loser of the championship game second-place, but the two semifinals losers both might have a claim to the third-place, and sorting through the fifth- through eight-place finishes is even trickier.
    
    These problems are reflections of each other: the reason that ranking the lower-placing teams is so hard is because they play so few games. It's easier to differentiate between first and second because both teams have played three games, but differentiating between the four teams that have played only a single game is nigh impossible.

    In some cases, these problems do not cause concern: perhaps we are only interested in crowning a champion and don't care about exactly who came in third, or maybe this bracket is being played at the conclusion of a long season and so teams playing variable numbers of games is not a big deal. But the interconnected nature of the two problems lets us solve them together, leveraging the extra games that lower-ranked teams have left in order to rank them.

    We consider class of formats called \textit{simple multibrackets}.

    \begin{definition}{Simple Multibracket}{} 
        A simple multibracket is a sequence of brackets in which the losers of certain games in the upper brackets fall into the lower brackets rather than being eliminated outright, and team place based on which bracket they won.
    \end{definition}

    The simplest example of a simple multibracket is the third-place game, in which the losers of the two semifinal games play each other for third place. The 2015 Asian Football Confederation Asian Cup, whose bracket is of signature $\bracksig{8; 0; 0; 0}$, employs a third-place game.

    \fig{.7}{third_place_game}{2015 AFL Asian Cup}

    Each game in this figure is labeled. In the primary bracket, first-round games are $\bracklabel{A1}$ through $\bracklabel{A4}$, while the semifinals are $\bracklabel{B1}$ and $\bracklabel{B2}$, and the finals is game $\bracklabel{C1}$. The actual labeling scheme is arbitrary, but we adopt the system used in Figure $\ref{fig:third_place_game},$ in which each round is given a letter and then games are numbered from the top to bottom of the bracket within each round. Note that the third-place game is labeled $\bracklabel{D1}$: even though it could be played concurrently to the championship game, it is part of a different bracket and so we label it as a different round.

    We indicate that the third-place game is to be played in between the losers of games $\bracklabel{B1}$ and $\bracklabel{B2}$ by labeling the starting lines in the third-place game with those games. This is not ambiguous because the winner of those games always just continues on in the original bracket, so such labels always refer to the losers. In theory, the Asian Cup could have had any two teams play for third.

    \fig{0.85}{third_place_game_bad}{2015 AFL Asian Cup Alternative Format}

    In this alternative, the third-place game is played between the loser of two of $\bracklabel{A1}$ and $\bracklabel{A3}$ instead of between the two semifinal losers. This is probably a bad idea: the losers of $\bracklabel{B1}$ and $\bracklabel{B1}$ are clearly more deserving than the losers of $\bracklabel{A1}$ and $\bracklabel{A3}$, and so they ought to be the teams playing the third-place game.

    We can use this intuition to define what it means for a simple multibracket to be proper.

    \begin{definition}{Proper Simple Multibracket}{}
        A simple multibracket is \textit{proper} if: 
        NEED BETTER DEFINITION
        %todo: define proper simple bracket.
    \end{definition}

    Note that as a simple multibracket with only two brackets, the Asian Cup is actually not quite proper: remember that because a team's final place is based on which bracket it wins. In order for the Asian Cup to have a top three, it needs to have three brackets in its multibracket. Otherwise, the winner of game $\bracklabel{D1}$ would be given second place, leaving the loser of $\bracklabel{C1}$ out in the cold and violating the properness condition.

    In reality there is an implied one-team bracket in between the primary bracket and the third-place game. Rigorously, the Asian Cup multibracket looks like this:

    \fig{0.85}{second_place_default}{2015 AFL Asian Cup, Rigorously}

    However for clarity, when drawing figures, we will omit brackets in a multibracket of signature $\bracksig{1}$. We restrict our study of simple multibrackets to proper ones, so if the loser of the championship of a bracket doesn't appear in the next bracket, then there is an implied bracket of signature $\bracksig{1}$ between them.

    \begin{definition}{Multibracket Signature}{}
        The \textit{signature} of a multibracket is simply the list of signatures of its brackets.
    \end{definition}

    So the 2015 AFL Asian Cup has signature $\bracksig{8;0;0;0} \to \bracksig{1} \to \bracksig{2;0}.$ But the multibracket with this signature is far from the only multibracket that the AFL could have used to give out gold, silver, and bronze. In fact, it's not clear the loser of $\bracklabel{C1}$, who comes in second place, is really more deserving than the winner of $\bracklabel{D1}$, who comes in third. One could imagine the UAE saying: both South Korea and us finished with two wins and one loss -- a first round win, a win against Iraq, and a loss against Australia. The only reason that South Korea came in second and we came in third was because South Korea lucked out by having Australia on the other half of the bracket as them. That's not fair!

    If the AFL took this complaint seriously, they could modify their format to have signature $\bracksig{8;0;0;0} \to \bracksig{2; 1; 0} \to \bracksig{1}.$
    
    \fig{0.85}{second_place_game}{$\bracksig{8;0;0;0} \to \bracksig{2; 1; 0} \to \bracksig{1}$}

    Again, the third bracket of signature $\bracksig{1}$ is implied and left out of the figure.

    If the AFL used the format in Figure \ref{fig:second_place_game} in 2015, then South Korea and the UAE would have played each other for second place after all of the other games were completed. In some sense, this is a more equitable format than the one used in reality: we have the same data about the UAE and South Korea and so we ought to let them play for second place instead of having decided almost randomly.

    However, swapping formats doesn't come without costs. For one thing, South Korea and the UAE would've had to play a fourth game: if the AFL had only three days to put on the tournament and teams can play at most one game a day, then the format in \ref{fig:second_place_game} isn't feasible.

    Another concern: what if Iraq had beaten the UAE when they played in game $\bracklabel{D1}$? Then the two teams with a claim to second-place would have been South Korea and Iraq, except South Korea already beat Iraq! In this world, South Korea being given second place without having to win a rematch with Iraq seems more equitable than giving Iraq a second chance to win. To address this, one could imagine a format in which game $\bracklabel{E1}$ is played only if it is not rematch, although this would no longer be just a multibracket.

    Ultimately, whether including game $\bracklabel{E1}$ is worth it depends on the goal of the format. If there is a huge difference between the prizes for coming in second and third, for example, if the top two finishing teams in the Asian Cup qualified for the World Cup, then $\bracklabel{E1}$ is quite important. If, on the other hand, this is self-contained format played purely for bragging rights, $\bracklabel{E1}$ could probably be left out. In reality, the 2015 AFL Asian Cup qualified only its winner to another tournament (the 2017 Confederations Cup), and gave medals to its top three, and so game $\bracklabel{E1}$ which distinguishes between second and third place, is probably unnecessary.

    Let's imagine, however, that instead of just the champion, the top four teams from the Asian Cup advanced to the Confederations Cup. In this case, the format used in 2015 would be quite poor, as teams finish in the top-four based only on the result of their first-round game: the rest of the games don't even have to be played. A better format for selecting the top four teams might look like this:

    \fig{0.85}{fourth_place_game}{$\bracksig{8;0;0;0} \to \bracksig{1} \to \bracksig{4; 2; 0; 0} \to \bracksig{1}$}

    The multibracket in Figure \ref{fig:fourth_place_game} has the really attractive property that a team will finish in the top four if and only if it wins two of its first three games. In fact,

    % \theo{All multibrackets with this property have signature $\bracksig{8;0;0;0} \to \bracksig{1} \to \bracksig{4; 2; 0; 0} \to \bracksig{1}$}{}{}
     
%     %2-1s
%     %MPC sets four teams
%     %can drop the final games
%     %like swiss?
%     %rematches
% }

% %Open Questions:
% %-- can you not play the second place game if it was a semifinal game.
% %-- should you seed in someway that is not proper.
% %-- rounds just for reseeding and replaying
% %-- swapping adjacent rounds for rematch
% %-- how to think pairing swisses

}