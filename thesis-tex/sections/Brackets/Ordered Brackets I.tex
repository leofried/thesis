\sub {

Let's consider the proper bracket $\bracksig{16; 0; 0; 0; 0}$, which was used in the 2022 NCAA Women's Basketball Tournament Wichita Region, and is shown below. (Sometimes brackets are drawn in the manner below, with teams starting on both sides and the winner of each side playing in the championship game.)

\fig{0.8}{ncaa}{2022 NCAA Women's Basketball Tournament Wichita Region}{}

The definition of a proper seeding ensures that as long as the bracket goes chalk (that is, higher seeds always beat lower seeds), it will always be better to be a higher seed than a lower seed. But what if it doesn't go chalk?

One counter-intuitive fact about the NCAA Basketball Tournament is that it is probably better to be a 10-seed than a 9-seed. (This doesn't violate the proper seeding property because 9-seeds have an easier first-round matchup than 10-seeds, and for further rounds, proper seedings only care about what happens if the bracket goes chalk, which would eliminate both the 9-seed and 10-seed in the first round.) Why? Let's look at whom each seed-line matchups against in the first two rounds:

\begin{figg}{NCAA Basketball Tournament 9- and 10-seed Schedules}{}
    \centering
    \begin{tabular}{ c | c c }
         Seed & First Round & Second Round \\
         \hline
         9 & 8 & 1\\
        10 & 7 & 2
    \end{tabular}
\end{figg}

The 9-seed has an easier first-round matchup, while the 10-seed has an easier second-round matchup. However, this isn't quite symmetrical. Because the teams are probably drawn from a roughly normal distribution, the expected difference in skill between the 1- and 2-seeds is far greater than the expected difference between the 7- and 8-seeds, implying that the 10-seed does in fact have an easier route than the 9-seed.

Nate Silver investigated this matter in full, finding that in the NCAA Basketball Tournament, seed-lines 10 through 15 give teams better odds of winning the region than seed-lines 8 and 9. Of course this does not mean that the 11-seed (say) has a better chance of winning a given region than the 8-seed does, as the 8-seed is a much better team than the 11-seed. But it does mean that the 8-seed would love to swap places with the 11-seed, and that doing so would increase their odds to win the region.

This is not a great state of affairs: the whole point of seeding is confer an advantage to higher-seeded teams, and the proper bracket $\bracksig{16; 0; 0; 0; 0}$ is failing to do that. Not to mention that giving lower-seeded teams an easier route than higher-seeded ones can incentivize teams to lose during the regular season in order to try to get a lower but more advantageous seed.

To fix this, we need a stronger notion of what makes a bracket effective than properness. The issue with proper seedings is the false assumption that higher-seeded teams will always beat lower-seeded teams. A more nuanced assumption might look like this:

\begin{definition}{Strongly Stochastically Transitive}{}
    A list of teams is \textit{strongly stochastically transitive} if for each $i, j, k$ such that $j < k$, $$\P(t_i \textrm{ beats } t_j) \leq \P(t_i \textrm{ beats } t_k).$$
\end{definition}

A list of teams being strongly stochastically transitive (or SST) captures the intuition that each team ought to do better against lower-seeded teams than against higher-seeded teams. A few quick implications of this definition are:

\begin{corollary}{}{}
    If a list of teams is SST, then for each $i < j$, $\P(t_i \textrm{ beats } t_j) \geq 0.5.$
\end{corollary}

\begin{corollary}{}{}
    If a list of teams is SST, then for each $i, j, k, \ell$ such that $i < j$ and $k < \ell$, $$\P(t_i \textrm{ beats } t_\ell) \geq \P(t_j \textrm{ beats } t_k).$$
\end{corollary}

Note that not every set of teams can be seeded to be SST. Consider, for example, the game of rock-paper-scissors. Rock beats paper which beats scissors which beats rock, so no ordering of these ``teams'' will be SST. For our purposes, however, SST will work well enough.

Our new, nuanced alternative a proper bracket is an \textit{ordered bracket}.

\begin{definition}{Ordered Bracket}{}
    A bracket and seeding are \textit{ordered} if, for any SST list of teams, if $i < j$, then $\P(t_i\textrm{ wins the tournament} )\geq \P(t_j\textrm{ wins the tournament}).$
\end{definition}

In an informal sense, a bracket being ordered is the strongest thing we can want without knowing more about why the tournament is being played. Depending on the situation, we might be interested in a format that almost always declares the most-skilled team as the winner, or in a format that gives each team roughly the same chance of winning, or anywhere in between. But certainly, better teams should win more, which is what the ordered bracket condition requires.

In particular, a bracket being ordered is a stronger claim than it being proper.

\theo{}{Every ordered bracket is proper.}{
    Let $\A$ be an ordered $n$-team bracket with $r$ rounds.\\
    
    Consider a list of teams such that every team wins every game with probability 0.5. These teams are SST. A team that plays their first game in the $i$th round will win the tournament with probability $(\frac{1}{2})^{r-i}$, so teams that get more byes will have a higher probability to win the tournament than teams with fewer byes. This implies that higher-seeded teams must have more byes than lower-seeded teams, so in each round, the teams with byes must be the highest-seeded teams that are still alive. Thus, condition (1) is met.\\

    We show that condition (2) is met by proving the stronger condition from Lemma \ref{th:proper_condition_two}: if $m$ teams have a bye and $k$ games are being played in round $s$, then if the bracket goes chalk, those matchups will be $t_{m + i}$ vs $t_{(m + 2k + 1) - i}$ for $i \in \{1, ..., k\}.$ We show this by strong induction on $s$ and on $i$.\\

    Assume that this is true for every round up until $s$ and for all $i < j$ for some $j$. Let $\ell = (m + 2k + 1) - j.$ We want to show that if the the bracket goes chalk, $t_{m + j}$ will face off against seed $t_{\ell}$ in the given round. Consider the following SST matchup table: every game is a coin flip, except for games involving a team seeded $\ell$ or lower, in which case the higher seed always wins. Then, each team seeded between $\ell - 1$ and $m + j$ will win the tournament with probability $(\frac{1}{2})^{r-s},$ other than the team slated to play $t_\ell$ in round $s$ who wins with probability $(\frac{1}{2})^{r-i-1}.$ In order for $\B$ to be ordered, that team must be $t_{m+j}.$\\

    Thus $\A$ satisfies both conditions, and so is a proper bracket.
}{ordered_proper}

With Theorem \ref{th:ordered_proper}, we can use the language of bracket signatures to describe ordered brackets without worrying that two ordered brackets might share a signature. Now we examine three particularly important examples of ordered brackets.

We begin with the unique one-team bracket.

\fig{1}{one_team}{The One-Team Bracket $\bracksig{1}$}
\theo{}{The one-team bracket $\bracksig{1}$ is ordered.}{Since there is only team, the ordered bracket condition is vacuously true.}{}

Next we look at the unique two-team bracket.

\fig{1}{two_team}{The Two-Team Bracket $\bracksig{2; 0}$}
\theo{}{The two-team bracket $\bracksig{2; 0}$ is ordered.}{Let $\A = \bracksig{2; 0}.$ Then,
\begin{align*}
   \P[t_1 \textrm{ wins } \A] = \P[t_1 \textrm{ beats } t_2] \geq 0.5 \geq \P[t_2 \textrm{ beats } t_1] = \P[t_2 \textrm{ wins } \A],
\end{align*}    
so $\A$ is ordered.
}{}

And thirdly, we show that the balanced four-team bracket is ordered.

\fig{0.63}{four_team}{The Four-Team Bracket $\bracksig{4; 0; 0}$}
\theo{}{The four-team bracket $\bracksig{4; 0; 0}$ is ordered.}{Let $\A = \bracksig{2; 0}.$ Let $p_{ij}$ be the probability that the $i$-seed beats the $j$-seed. Then,
\begin{align*}
    \P[t_1 \textrm{ wins } \A] &= p_{14} \cdot (p_{23}p_{12} + p_{32} p_{13})\\
    &= p_{14}p_{23}p_{12} + p_{14}p_{32} p_{13}\\
    &\geq p_{14}p_{23}p_{21} + p_{24}p_{41} p_{23}\\
    &= p_{23} \cdot (p_{14}p_{21} + p_{41}p_{24})\\
    &= \P[t_2 \textrm{ wins } \A]
\end{align*}
\begin{align*}
    \P[t_2 \textrm{ wins } \A] &= p_{23} \cdot (p_{14}p_{21} + p_{41} p_{24})\\
    &\geq p_{32} \cdot (p_{14}p_{31} + p_{41} p_{34})\\
    &= \P[t_3 \textrm{ wins } \A]
\end{align*}
\begin{align*}
    \P[t_3 \textrm{ wins } \A] &= p_{32} \cdot (p_{14}p_{31} + p_{41} p_{34})\\
    &= p_{32}p_{14}p_{31} + p_{32}p_{41} p_{34}\\
    &\geq p_{42}p_{23}p_{41} + p_{32}p_{41} p_{43}\\
    &= p_{41} \cdot (p_{23}p_{42} + p_{32}p_{43})\\
    &= \P[t_4 \textrm{ wins } \A]
\end{align*}

Thus the four-team bracket $\bracksig{4; 0; 0}$ is ordered.
}{}

In the next section, we move on from describing particular ordered brackets in favor of a more general result.
}