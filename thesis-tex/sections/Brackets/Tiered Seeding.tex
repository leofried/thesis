\sub{

    Consider the 2016 Olympic Basketball Tournament. Twelve teams qualified for the Olympics, and they were divided into two groups of six teams each. Each group conducted a mini-tournament, ranking the teams in each group form first through sixth. Then, the bottom two teams in each group were eliminated, with the remaining eight teams (four from each group) entering the bracket $\bracksig{8;0;0;0}.$ The entire format as it played out is displayed in Figure \ref{fig:olympics}.

    \fig{0.78}{olympics}{The 2016 Olympic Basketball Tournament}

    The seeding going into the bracket portion of the 2016 Olympic Basketball Tournament is a little different than the seedings that we have discussed so far. Rather than the ranking on the teams being a complete ordering, it is a partial one: teams are grouped into tiers, which are themselves ordered. Two teams, one from each group, occupy each tier.

    \begin{definition}{Tiered Seeding}{}
        A \textit{tiered seeding} is a partial ordering on the teams entering a tournament.
    \end{definition}

    This is as opposed to traditional seedings, which are a complete ordering (although we can view a traditional seeding as a special example of a tiered seeding where each tier has a single team). When filling out a bracket using a tiered seeding, we continually assign the top remaining seeds to the teams in the top remaining tier. Recall the proper bracket $\bracksig{8;0;0;0}$:

    \fig{0.8}{eight}{$\bracksig{8;0;0;0}$}

    The United States and Croatia, as the two teams in the top tier, are given seeds one and two. The two tier-two teams, Australia and Spain, get seeds three and four, and so on. The actual algorithm used for assigning the seeds to the teams within each tier can be arbitrary: unless stated otherwise, we will assume that it is done randomly. (In the particular case of the 2016 Olympic Basketball Tournament, teams from Group A were given the odds seeds and teams from Group B the evens.)

    


}



% Possible, Chalk (strong vs weak?), Round/Compensation
% (Open question proper)
% (Open question, giving tiers, does there exists a strong, weak seeding)
% ordered brackets stay ordered (but new ones become ordered) [e.g. tenis]