\sub {

    % Round robins are a particularly nice example of larger class of formats.

    % \begin{definition}{Fixed Formats}{}
    %     A \i{fixed format} is one in which the number of games played is constant
    % \end{definition} %ordered

    % \begin{definition}{Scheduled Formats}{}
    %     A \i{scheduled format} is one in which every team knows which other teams they are playing and in what order before any of the games are played.
    % \end{definition}

    % Another scheduled format is the \i{partial round robin}.

    % \begin{definition}{Partial Round Robin}{}
    %     An $n$-team $g$-game \i{partial round robin} is a format in which each team plays exactly $g$ of their opponents, and then teams are ranked based on how many wins they have, with some tiebreaking scheme.
    % \end{definition}

    % An $n$-team round robin the special case of a $n$-team $(n-1)$-game round robin. 

}



% Round Robins:
%     --Intro (landau's theorem) (round robins are ordered)
%     --Scheduled Formats (impure)
%     --Tiebreakers
%     --Faithfulness
%     --Pools
%     --Pool Extensions (seeding, variants (playing games outside of pools (nfl) power pools, wildcards (formal e.g. 13 and informal e.g. mlb)), "finishing pool play", pools but not round robin insides)
%     --Impure 
%         (play too few [ramsey theorem, two different seven play four, etc])
%         (play too many [double round robin])
%         (ramsey theorem, two different seven play four, divisions, mlb, soccer ties, etc)
    
