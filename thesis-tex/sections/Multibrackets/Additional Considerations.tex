\sub {
    
    % As discussed at the end of the previous section, we will now consider properties that multibrackets 
    %IDK INTRO

    As a case study throughout this section, we will consider the losers' bracket of a double elimination tournament who's winners' bracket has signature $\bracksig{8;0;0;0}.$ And while we won't worry about whether the format is perfect or not, we will assume that the format in $\B$ only takes two teams. In summary, our job is to construct a bracket to narrow the four $\bracklabel{A}$-round losers, two $\bracklabel{B}$-round losers, and three $\bracklabel{C}$-round loser down to a single team, that will face the winner of the winners' bracket in the consolidation phase.

    \fig{1}{winners}{$\bracksig{8;0;0;0}$}

    Recall that the standard double-elimination format $\D_3$ uses the following losers' bracket.

    \fig{1}{standard_losers}{$\bracksig{4;2;0;1;0}$}

    But the standard losers' bracket isn't the only option: any bracket on seven team will do. Figure \ref{fig:ladder_losers} gives another such bracket.

    \fig{1}{ladder_losers}{$\bracksig{2;1;1;1;1;1;0}$}

    The standard losers' is quite a bit better than alternative in Figure \ref{fig:ladder_losers}. For one thing, it is more compact, taking four rounds instead of six. But more importantly it feel more fair: teams that lose in the same round treated equally. This is as opposed to Figure \ref{fig:ladder_losers}, in which $\bracklabel{A4}$ is given two more byes than $\bracklabel{A1}$ despite them having lost in the same round. We call a semibracket that respects this equality between teams that lose in the same round \i{weakly respectful}.

    \begin{definition}{Weakly Respectful Semibracket}{weakly_respectful}
        A semibracket $\A_i$ for $i > 1$ in a multibracket $\A$ is \i{weakly respectful} if it satisfies the following conditions:
        \begin{enumerate}
            \item If two teams fall into $\A_i$ from different brackets, the team that fell from the lower bracket gets at least as many byes as the other.
            \item If two teams fall into $\A_i$ from the same bracket but different round, the team that fell from the later round gets as least as many byes as the other.
            \item If two teams fall into $\A_i$ from the same round of the same bracket, they get the same number of byes.
        \end{enumerate}
    \end{definition}

    \begin{definition}{Weakly Respectful Mulibracket}{}
        A multibracket is \i{weakly respectful} if each of its semibrackets are weakly respectful.
    \end{definition}


    Most important to us at the moment is the third condition, which ensures that the losers' bracket in Figure \ref{fig:ladder_losers} is not weakly respectful. (Though the second condition is important for ensuring that a losers' bracket identical to the standard one but with $\bracklabel{C1}$ and $\bracklabel{A1}$ swapped isn't weakly respectful either. The first condition does't apply right now as we are considering multibrackets with only two semibrackets.)

    Note that Definition \ref{def:weakly_respectful} discusses a semibracket with respect to the multibracket that it's a part of. So the point is not that the bracket signature $\bracksig{4;2;0;1;0}$ is weakly respectful in a vacuum, but that it is weakly respectful in the multibracket $\D_3.$ By the same token, there are multibrackets for which the signature $\bracksig{2;1;1;1;1;1;0}$ could be weakly respectful (Figure \ref{fig:double_ladder} is one such example), but its's not weakly respectful to a primary bracket of $\bracksig{8;0;0;0}.$

    \fig{0.8}{double_ladder}{$\bracksig{2;1;1;1;1;1;1;0} \to \bracksig{2;1;1;1;1;1;0}$}

    Returning the point at hand, we know that the standard $\bracksig{4;2;0;1;0}$ is weakly respectful for purposes. But there are two other weakly respectful bracket signatures, which are displayed below.

    \fig{1}{losers_6100}{$\bracksig{6;1;0;0}$}
    \fig{1}{losers_40300}{$\bracksig{4;0;3;0;0}$}

    Both of these losers' brackets are weakly respectful, but the first one looks a bit uneven. Even though all four of the $\A$-round losers are given the same number of byes, they aren't treated the exact same. In Figure \ref{fig:losers_6100}, $\bracklabel{A3}$ and $\bracklabel{A4}$ play each other the first round and $\bracklabel{C1}$ in the second, while $\bracklabel{A1}$ and $\bracklabel{A4}$ play a $\bracklabel{B}$-round loser in the first round and can't play a $\bracklabel{C}$-round loser until the third. It's not clear which route is better, but they are definitely different: Figure \ref{fig:losers_6100} is not \i{strongly repectful}.

    \begin{definition}{Strongly Respectful Semibracket}{strongly_respectful}
        A semibracket $\A_i$ for $i > 1$ in a multibracket $\A$ is \i{strongly respectful} if it is weakly respectful and $\A_i$ looks the same from the perspective of each team that dropped from the same round of the same bracket.
    \end{definition}

    \begin{definition}{Strongly Respectful Multibracket}{}
        A multibracket is \i{strongly respectful} if each of its semibrackets are strongly respectful.
    \end{definition}

    (Definition \ref{def:strongly_respectful} can be formalized by counting automorphisms the graph underlying the bracket up to the letters labeling the starting lines.)

    So if we are looking for strong respectfulness, there are two signatures, $\bracksig{4;2;0;1;0}$ and $\bracksig{4;0;3;0;0},$ that satisfy all of our requirements. Let's say we settled on the first, which is used in the standard format $\D_3.$ How do we decide where to place the various $\bracklabel{A}$- and $\bracklabel{B}$-round drops? To illustrate the point, consider the following losers' bracket, also with signature $\bracksig{4;2;0;1;0}$ as opposed to the one used in the standard system in Figure \ref{fig:standard_losers}.

    \fig{1}{rematches_allowed}{Alternative $\bracksig{4;2;0;1;0}$}

    Like the standard losers' bracket, this one is strongly respectful: the only difference is that $\bracklabel{A2}$ and $\bracklabel{A3}$ have been swapped. So why does the standard losers' bracket put the teams where it does? Rematches.

    In any tournament format, rematches are far from ideal. From an information theoretical perspective, a rematch is less informative then a new matchup: we already have some data on how those two team compare. From a competitive perspective, they are unsatisfying: without the ability to play a third ``rubber'' match, if each team wins one game, we are left in a disappointing state of uncertainty. And in multibrackets these issues are exacerbated: nothing feels worse than being eliminated in a double-elimination to two losses from the same team. 

    So the choice between the two instantiations of $\bracksig{4;2;0;1;0}$ comes down to which bracket will lead to fewer rematches. A quick analysis shows that this is done by the standard construction, where only $\bracklabel{F1}$ or $\bracklabel{G1}$ could be a rematch, as opposed to the format on the right, where any game after the first round could be.

    In fact, rematches in multiple elimination formats are so bad that many tournaments include a provision, allowing the structure of a given round to be rearranged to avoid them. While there are no opportunities to do so in $\D_3$, an format using $\D_4$ should strongly consider using such a provision, swapping $\bracklabel{C1}$ and $\bracklabel{C2}$ in round $\bracklabel{H}$ if doing so would reduce the number of rematches in that round. (Earlier rounds are free of rematches, and later rounds are one game each.)

    \fig{0.75}{D4additional}{$\D_4$}

   The USA Ultimate Manual of Championship Series Tournament Formats includes a full page detailing their policy for when teams should be swapped in an effort to avoid rematches during a multibracket \cite{ultimate}.

    %properness






}

