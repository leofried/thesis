\sub {

    For the four sections we have focused our study on \i{linear} multibrackets with the property that when a team loses in a given semibracket they drop into a different, lower semibracket. But many leagues use nonlinear multibrackets as well, and so while our tools of signature and properness are less equipped to study them, we look at what the space looks like.

    An simple example of a nonlinear multibracket was the format used by the 2019 Suncorp Super Netball Playoffs \cite{wiki_netball}, sometimes called the Page-McIntyre system \cite{wiki_page}.

    \fig{0.6}{page4}{2019 Suncorp Super Netball Playoffs}

    Nonlinear multibrackets are a bit strange: while the winner of game $\bracklabel{A1}$ goes directly to the final, the loser falls into the semifinal of the \i{same bracket}. This poses problems for both attempts to define a signature as well as a notion of properness.

    Beginning with signature, the shape of the bracket is a bit strange: the winner of game $\bracklabel{A1}$ gets a bye \textit{after winning a game}, something that never happens in a traditional bracket. Attempts to give this bracket a signature might lead to $\bracksig{4;1;0;0}$ or even $\bracksig{4;0;0;0}$, neither of which are actually bracket signatures (they both violate Theorem \ref{th:signature_sum}). The issue here is that game $\bracklabel{A1}$ is actually a semifinal, and so ``should'' (if it didn't deliver its loser to the other semifinal) live in the second round, producing a signature of $\bracksig{2;3;0;0}$. But then of course this format is quite different from traditional brackets with that same signature. Bracket signatures on nonlinear multibrackets are in general not well-defined.

    To make matters worse, the first round appears to have an ``improper'' set of matchups: the games are 1v2 and 3v4 rather than then ``proper'' 1v4 and 2v3. However, properness is a much trickier concept for nonlinear multibrackets. While the 1- and 2-seeds to have tougher first round matchups than the 3- and 4-seeds, this is compensated by them getting an extra life: if they lose, they play the winner of the 3v4 matchup, while the 3v4 loser is just eliminated, so no team would prefer to be seeded lower than they are. One could imagine developing this intuition of extra lives into a formal notion of properness, but we leave that question untreated.

    \begin{oq}{}{}
        How can we define signatures and properness for nonlinear multibrackets?
    \end{oq}

    One final thing of note is that $\bracklabel{C1}$ can be a rematch of game $\bracklabel{A1}.$ In fact, this pretty likely: if the bracket goes chalk, the 1- and 2-seed will find themselves replaying the game they played just two rounds ago. In Figure \ref{fig:page4}, the bracket did not go chalk, but game $\bracklabel{C1}$ was still a rematch. As we discussed in Section \ref{sec:Flowcharts}, this can be pretty unsatisfying: indeed, in the 2019 Suncorp Super Netball Playoffs, the Swifts and the Lightning each beat each other once, but the Swifts won the game that mattered and so was declared champion. 
    
    As discussed earlier in the chapter, one option would be to be to make game $\bracklabel{C1}$ contingent on it not being a rematch: if it is a rematch, then the game is skipped and whichever team won the previous game is declared champion. While this solution is effective for the second-place game in our alternative AFL Asian Cup format (Figure \ref{fig:afl_second} on page \pageref{fig:afl_second}), it doesn't work here. Making the game contingent would mean that the loser of $\bracklabel{A1}$ is actually eliminated upon their loss: even if they win $\bracklabel{B1}$, they wouldn't have the ability to play in the championship game.

    A better solution might be a \i{double-elimination tournament}, as employed by the 2016 NCAA Softball Ann Arbor Regional \cite{wiki_soft}.

    \fig{0.4}{softballv1}{2016 NCAA Softball Ann Arbor Regional}

    \begin{definition}{Double-Elimination Tournament}{}
        A \i{double-elimination} tournament is a multibracket (plus one contingent game) consisting of a \i{winners' bracket}, where every team starts, a \i{losers'} bracket, that every winners' bracket loser falls into, and a \i{grand finals}, in which the winner of the winners' bracket plays the winner of the losers' bracket for the championship, with the losers' bracket winner needing to win twice, while the winners' bracket winner only needs to win once. 
    \end{definition}

    A double-elimination tournament guarantees that the winner will finish undefeated or with only one loss, while every other team finishes with two.

    The 2016 NCAA Softball Ann Arbor Regional is an example of a double-elimination format: the winners' bracket consists of games $\bracklabel{A1}, \bracklabel{A2},$ and $\bracklabel{B1}$; the losers' bracket consists of games $\bracklabel{B2}$ and $\bracklabel{C1}$; and the grand finals is game $\bracklabel{D1}$ and then, if necessary, $\bracklabel{E1}.$ Michigan finished undefeated while Valparaiso, Notre Dame, and Miami (OH) each finished with two losses.

    Because double-elimination tournaments are so common, and all use a contingent game that is played only if the lower team wins ($\bracklabel{E1}$ in the case of Figure \ref{fig:softballv1}), that contingent game has a name.

    \begin{definition}{Recharge Game}{}
        A \i{recharge game} is a contingent game in a multibracket that is a rematch of a previous game and played only if the lower team won the first game.
    \end{definition}

    Recharge games are so common that we introduce a special notation: if the name of a game has a star after it, then that game is followed by a recharge game (if necessary). This allows us to condense the format in the Figure \ref{fig:softballv1} a little bit, as displayed in Figure \ref{fig:softballv2}.

    \fig{0.45}{softballv2}{2016 NCAA Softball Ann Arbor Regional}

    The only issue with this notation is that, if the recharge game was triggered but won by the upper team, there is no natural place to denote that the recharge game was played. We adopt the convention of writing the the lower team \i{under} the line that the winner of the recharge game is placed over in this case. This is depicted in Figure \ref{fig:softballv3}.
    
    \fig{0.5}{softballv3}{Figure \ref{fig:softballv2} if Notre Dame Beat Michigan Once}

    While the recharge game is necessary to ensure that the format is a truly a double-elimination tournament, as well as preventing the problem in the Page-McIntyre System where the champion and runner-up each finish with one-loss, it's not all upside. For one thing, Dabney \cite{recharge_rounds} found some evidence that a tournament with no recharge game actually does a better job of crowing the best team as champion then the truer double-elimination with the recharge game included. Additionally, formats with recharge games tend to be less exciting, as they risk not playing a true championship game (a game in which either team wins the format if they win that game).

    In any case, whether the recharge game is used or not, double-elimination tournaments are a powerful tool in a tournament designer's arsenal, as they are in some sense more accurate than their single elimination counterparts. We prove this fact for a simplified case where the winners' and losers' bracket are relatively nice, and where there is a single best team that is favored against every other team with a constant probability $1/2 < p < 1$.

    \theo{}{ 
        Let $n$ be a positive integer, $p$ be a probability such that $1/2 < p < 1$, and $\T$ be a list of $2^n$ teams with a team $t \in \T$ such that for every other team $s$, $$\G{t}{s} = p.$$ 
        Let $\A$ be the balanced bracket on $2^n$ teams, let $\B$ be a bracket on $2^{n}-1$ teams such that the linear multibracket $\A \to \B$ is weakly proper, and let $\C$ be the double-elimination format with winners' bracket $\A$ and losers' bracket $\B$. Then, $$\W{\C}{t}{\T} \geq \W{\A}{t}{\T}$$ with equality only when $n=1$ and there is no recharge round.
    }{
        To win $\A$, $t$ simply has to win $n$ games. Thus $$\W{\A}{t}{\T} = p^n.$$
    
        Now consider $\C.$ Let $r$ be the number of rounds in $\B$, let $r_i$ be the round of $\B$ that teams that lose in the $i$th round of $\A$ fall into, and let $c_i = r-r_i+1,$ so teams that lose in the $i$th round of $\A$ need to win $c_i$ games in $\B$ in order to make the grand finals.\\
        
        Since there are $2^{n-i}$ $i$-round losers, by Theorem \ref{th:signature_sum},
        $$\sum_{i=1}^n 2^{n-i} \cdot \left(\frac{1}{2}\right)^{c_i} = 1,$$
        so,
            \[\sum_{i=1}^n \left(\frac{1}{2}\right)^{c_i + i - 1} = \left(\frac{1}{2}\right)^{n-1}.  \tag{*}\]
        Letting $q = 1 - p,$ note that $t$ wins the winners' bracket with probability $p^n$, and the losers' bracket with probability
        $$\sum_{i=1}^n p^{i-1} \cdot q \cdot p^{c_i} = q \cdot \sum_{i=1}^n p^{c_i+i-1} \geq q \cdot p^{n-1},$$
        with the inequality coming by equation $(*)$ because $p > \frac{1}{2}$, and with equality only when $n = 1.$\\

        Now, if there is a recharge round, then
        \begin{align*}
            \W{\C}{t}{\T} &= \W{\A}{t}{\T} \cdot (p + qp) + \W{\B}{t}{\T} \cdot p^2\\
            &\geq p^n(p + qp) + (q \cdot p^{n-1}) \cdot p^2 &\textrm{with equality only when $n=1$}\\
            &= p^n (p + 2qp)\\
            &> p^n\\
            &= \W{\A}{t}{\T}.
        \end{align*}
        
        If there is no recharge round, then
        \begin{align*}
            \W{\C}{t}{\T} &= \W{\A}{t}{\T} \cdot p + \W{\B}{t}{\T} \cdot p\\
            &\geq p^n(p) + (q \cdot p^{n-1}) \cdot p &\textrm{with equality only when $n=1$}\\
            &= p^n\\
            &= \W{\A}{t}{\T}.
        \end{align*}
        

        Thus, $$\W{\C}{t}{\T} \geq \W{\A}{t}{\T}$$ with equality only when $n=1$ and there is no recharge round.
    }{}

    We conclude our discussion of nonlinear multibrackets with a few more interesting examples. The first is the 2022 Big Ten Baseball Tournament \cite{wiki_bigten}.

    \fig{0.5}{baseball}{2022 Big Ten Baseball Tournament}

    The 2022 Big Ten Baseball Tournament wanted to balance two effects: first, that double-elimination formats lead to more accurate results, but second, that championship games are exciting and double-elimination games risk not including one. 2022 Big Ten Baseball Tournament innovates to solve the latter issue by including recharge games in the \i{semifinals}, and then having the championship game be single winner-take-all game.

    Note that this format does not fully solve all the problems it is attempting to tackle: for one thing, it is not a true double-elimination, as Rutgers gets eliminated with only a single loss. That said, Michigan is unambiguously the most deserving winner: every team other than Michigan and Rutgers lost once, and Michigan defeated Rutgers in their one matchup.
    
    However, this property was not guaranteed: had Penn State beaten Iowa in game $\bracklabel{C1}$, Michigan twice in game $\bracklabel{D1}$ and the recharge game, and then Rutgers in the final, we would be back to the issue with the Page-McIntyre System. Penn State and Rutgers would have each finished with only one-loss to the other team, with the champion being determined somewhat arbitrarily by who won the most recent game. This illustrates an important point: the desire for an unambiguous champion and the desire for an unambiguous championship \i{game} are fundamentally in conflict in the world of multibrackets.

    Another interesting nonlinear multibracket of note is the NBA Playoffs. You may recall from Figure \ref{fig:nba} that in 2004, the NBA Eastern Conference Playoffs used a simple bracket of signature $\bracksig{8;0;0;0}$ to determine its champion (the Western Conference did the same, and then the two conference champions played each other in the NBA finals). However, in 2020, after a much of the NBA regular season was cut short due to Covid, there was a feeling that the regular season wasn't as accurate a measure as it usually is. So the playoffs were expanded slightly: if the 8th and 9th place teams were close enough in record, the playoff for that conference expanded to $\bracksig{2;7;0;0;0},$ allowing both teams in \cite{wiki_nba1}. After the success of that system, the playoffs were expanded further starting in 2021 to the following nonlinear multibracket \cite{wiki_nba2}.
    
    \fig{0.5}{nbaplayin}{2022 NBA Eastern Conference Playoffs}

    The first two rounds of the new NBA playoffs are similar in structure to the Page-McIntyre system: two lower-seeded teams play each other and two higher-seeded teams play each other, and then the winner of the first game plays the loser of the second. But because the two qualifying teams get dumped into a larger eight-team bracket, rather than facing off immediately, the issues of the original Page-McIntyre system are avoided.

    A final nice example of nonlinear multibrackets is bitonic sort. Bitonic sort was developed by Batcher \cite{batcher} as a networked sorting algorithm with low delay (the sorting-theory equivalent to a low number of rounds). As every sorting algorithm can be transformed into a tournament format, and every networked sorting algorithm can be transformed into a multibracket, we can construct an nonlinear multibracket that executes Batcher's bitonic sort.

    \begin{definition}{Bitonic Sort}{}
        The \i{bitonic sort} on $2^r$ teams proceed by diving the teams into two groups of $2^{r-1}$ teams, recursively running the bitonic sort on $2^{r-1}$ teams on each group, and then running the standard swiss format $\S_r$ on the full group of $2^r$ teams, with one of the groups getting the odd seeds in $\S_r$ and the other group getting the even seeds.
    \end{definition}

    The 8-team bitonic sort is displayed in Figure \ref{fig:bitonic}. The $\bracklabel{A}$-, $\bracklabel{B}$-, and $\bracklabel{C}$- round games facilitate the running of two parallel instantiations of the 4-team bitonic sort, while the $\bracklabel{D}$-, $\bracklabel{E}$-, and $\bracklabel{F}-$round games carry out $\S_3.$

    \fig{1}{bitonic}{8-Team Bitonic Sort}

    We leave it to the reader to verify that bitonic sort is in fact a sorting algorithm: that is, if the matchup table is SST with all win probabilities being 0 or 1 (even if the teams are not seeded in the correct order initially!), bitonic sort will correctly sort the teams. Impressively, the 8-team bitonic sort does this in only six rounds: no team needs to play every other team in order to complete the sort.
}