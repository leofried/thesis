\sub {

    Properness as defined in the previous section has a weakness: it doesn't require that teams in the \i{same} tier be treated similarly, only that teams in higher tiers be treated better than teams in lower tiers. To illustrate this, consider the following proper linear multibracket of signature $\bracksig{8;0;0;0} \to \bracksig{2;0}$.

    \fig{1}{not_minimal}{$\bracksig{8;0;0;0} \to \bracksig{2;0}$}

    This linear multibracket is proper: the primary bracket is simply the proper seeding of $\bracksig{8;0;0;0}$, and the teams that lost in the semifinals are given at least as good of spots as the teams that lost in the first rounds. But it is still pretty clearly wrong. Properness only guarantees that teams that advanced further in the primary bracket are treated better than teams that didn't. But we also have an intuition that teams that lost in the same round of the primary bracket ought to be treated the same. This intuition has a name: \i{respectfulness}. Unlike properness, respectfulness comes in a few different levels, the weakest of which is \i{minimal respectfulness}.    

    \begin{figg}{Respectfulness Properties\\ (Weakest at the Top, Strongest at the Bottom)}{respectful_lattice}
        \begin{center}
            \overfullhbox{
\begin{tikzcd}[ampersand replacement=\&, transform shape, scale=0.2]
    \& \textrm{Proper}\arrow[dd, no head]                                    \&                                                  \\
    \&                           \&                                                  \\
    \& \textrm{Minimally Respectful}\arrow[ldd, no head] \arrow[rdd, no head] \&                                                  \\
    \&                           \&                                                  \\
    \textrm{Round-Respectful}\arrow[rdd, no head] \&                                                                       \& \textrm{Schedule-Respectful}\arrow[ldd, no head] \\
    \&                           \&                                                  \\
    \& \textrm{Weakly Respectful}\arrow[dd, no head]                         \&                                                  \\
    \&                           \&                                                  \\
    \& \textrm{Strongly Respectful}                                         \&                                                 
\end{tikzcd}
            }
\end{center}
    \end{figg}

    \begin{definition}{Minimally Respectful Linear Multibracket}{}
        A proper linear multibracket is \i{minimally respectful} if, for every round of every semibracket, all the losers of that round fall into the same other semibracket (or are all eliminated).
    \end{definition}

    But minimal respectfulness as the name implies, is just the minimum. Consider, the following minimally respectful linear multibracket of signature $\bracksig{8;0;0;0} \to \bracksig{1}\to \bracksig{2;0} \to \bracksig{1} \to \bracksig{2;1;1;0}.$

    \fig{1}{not_round}{$\bracksig{8;0;0;0} \to \bracksig{1}\to \bracksig{2;0} \to \bracksig{1} \to \bracksig{2;1;1;0}$}

    This consolation bracket is minimally respectful: every team that lost in the first round of the primary bracket falls in into the same semibracket. But still, teams that lost in the same round are not being treated the same: some first-round losers are getting more byes than others. It is not \i{round-respectful}.

    \begin{definition}{Round-Respectful Consolation Bracket}{}
        A minimally respectful linear multibracket is \i{round-respectful} if, for every round of every semibracket, all the losers of that round fall into the same round of the same other semibracket (or are all eliminated).
    \end{definition}

    The next level of respectfulness to examine is called schedule-respectfulness, and to understand it, we recall our example from last chapter that showed the fundamental theorem doesn't apply to linear multibrackets.

    \fig{0.95}{4200tiers_respect}{$\bracksig{4;2;0;0}$}

    Both of these options for a bracket of signature $\bracksig{4;2;0;0}$ are proper, and in fact, both are minimally respectful while neither is round-respectful. But are they equally good? The right bracket seems a little bit more fair (that is, respectful). In the left bracket, $\bracklabel{F1}$ lucks out, getting both a first-round bye and dodging the highest-tiered $\bracklabel{G1}$ until the final game of the bracket. In the left bracket, however, the advantages are distributed: $\bracklabel{F1}$ gets a first-round bye, but has the toughest second-round matchup, while the other $\bracklabel{F}$ round losers each have to play an extra game but are on the opposite side of the bracket as $\bracklabel{G1}$. We say the right bracket, but not the left, is \i{schedule-respectful}

    \begin{definition}{Schedule-Respectful Linear Multibracket}{}
        A \i{schedule-respectful} linear multibracket is a minimally respectful linear multibracket in which [[RIGOR IZE THIS DEFINITION]]
    \end{definition}

    A given minimally respectful linear multibracket can be any of round-respectful, schedule-respectful, neither, or both. If a linear multibracket is both round- and schedule-respectful, we say it is \i{weakly respectful}.

    \begin{definition}{Weakly Respectful Linear Multibracket}{}
        A linear multibracket is \i{weakly respectful} if it is both round-respectful and schedule-respectful.
    \end{definition}

    The linear multibracket in Figure \ref{fig:second61} is weakly respectful.

    \fig{1}{second61}{$\bracksig{8;0;0;0} \to \bracksig{6;1;0;0}$}

    Finally, we can simply require that teams that lost in the same round of the a given semibracket be given symmetric spots in the semibracket they fall into.

    \begin{definition}{Strongly Respectful Linear Multibracket}{}
        A \i{strongly respectful} linear multibracket is a minimally respectful one in which teams that lost in the same round of the primary bracket are given the same path in the linear multibracket.
        %  (up to rearranging teams of the same tier).
    \end{definition}

    Strong respectfulness is the gold standard of respectfulness in consolation brackets: while the consolation bracket in \ref{fig:second61} is not strongly respectful, while the two linear multibrackets displayed below, are.

    \fig{0.7}{strong}{Two Strongly Respectful Linear Multibrackets}

    % We discussed in the previous section how the cheap way that the fundamental theorem of brackets doesn't apply to linear multibrackets is that we can permute teams of the same tier to generate new proper linear multibrackets of the same signature. This trick doesn't work just for properness: it works for every level of respectfulness as well. The $\bracklabel{A}$-round losers in the left format of Figure \ref{fig:strong} could be permuted in any of the 24 configurations, and the result would still be strongly respectful. So why is it standard to place the $\bracklabel{A}$-round losers in the way we did in the left format of Figure \ref{fig:strong}? (And why did the 2023 Southern Conference Wrestling Tournament make the same decisions in Figure \ref{fig:socon}?)
    
    % Rematches.

    % In any tournament format, rematches are far from ideal. From an information theoretical perspective, a rematch is less informative than a new matchup: we already have some data on how those two team compare. From a competitive perspective, they are unsatisfying: without the ability to play a third ``rubber'' match, if each team wins one game, we are left in a disappointing state of uncertainty. And in a multibracket these issues are exacerbated: Being eliminated from contention due to two losses from the same team feels awful.

    % With any other configuration of $\bracklabel{A}$-round losers, we risk a rematch as soon as the second-round of the second-place bracket. By placing the $\bracklabel{A}$-round losers in the way that they were, rematches are delayed in the bracket for as long as possible.    

%     %[[Concluding paragraph about how to use these paragraphs.]]

%     %also the other bad part of properness (wanting the drops to be set up for  8000 the big idea you now you know)

% % also also that > minimal respectfulness is not alway possible! 321 eg.


}