\sub{

    At the end of the previous section, we arrived at the weakly proper simple multibracket in Figure \ref{fig:fourth_place_game} as a great option for selecting a first-, second-, third-, and fourth-place team out of eight competing teams. And while we presented it as the natural instantiation of of the signature $\bracksig{8;0;0;0} \to \bracksig{1} \to \bracksig{4; 2; 0; 0} \to \bracksig{1},$ it is far from the only one. Below, we display the format from Figure \ref{fig:fourth_place_game} on the left, along with a different weakly proper format of the same signature on the right.
    
    \fig{0.65}{fourth_place_game_bad}{Two Simple Multibrackets with Signature\\ $\bracksig{8;0;0;0} \to \bracksig{1} \to \bracksig{4; 2; 0; 0} \to \bracksig{1}$}

    The primary brackets are identical, and both formats are weakly respectful: the difference comes in where the $\bracklabel{A}$-round loses are placed. Why do we prefer the format on the left? Rematches.

    In any tournament format, rematches are far from ideal. From an information theoretical perspective, a rematch is less informative then a new matchup: we already have some data on how those two team compare. From a competitive perspective, they are unsatisfying: without the ability to play a third ``rubber'' match, if each team wins one game, we are left in a disappointing state of uncertainty. And in a multibracket of signature $\bracksig{8;0;0;0} \to \bracksig{1} \to \bracksig{4; 2; 0; 0} \to \bracksig{1}$ they can be particularly frustrating to the teams themselves: to end in the top-four, teams must win two games before they lose two. If a team loses two games to the same team, they might reasonably feel like they didn't get a fair shake.

    So the choice between the two simple multibrackets in Figure \ref{fig:fourth_place_game_bad} comes down to which bracket will lead to fewer rematches. A quick analysis shows that this is done by the format on the left, where only $\bracklabel{F1}$ could be a rematch, as opposed to the format on the right, where any of $\bracklabel{E1},$ $\bracklabel{E1},$ or $\bracklabel{F1}$ could be.

    In fact, the simple multibracket on the left is the weakly proper bracket of signature $\bracksig{8;0;0;0} \to \bracksig{1} \to \bracksig{4; 2; 0; 0} \to \bracksig{1}$ with the fewest rematch opportunities, and so we consider it the canonical multibracket with that signature. In the general case, however, finding this multibracket is an open question.

    \begin{oq}{}{}
        Given a multibracket signature $\A$, which instantiation of it has the fewest number of expected rematches?
    \end{oq}

    Not only is the bracket on the left not the only instantiation of its signature (though we established it is the one that best avoids rematches), its signature is not the only one of order four. Another option might be $\bracksig{8;0;0;0} \to \bracksig{1} \to \bracksig{2; 3; 0; 1; 0} \to \bracksig{1}.$

    \fig{0.75}{fourth_place_game_very_bad}{$\bracksig{8;0;0;0} \to \bracksig{1} \to \bracksig{2; 3; 0; 1; 0} \to \bracksig{1}$}

    Like the simple multibrackets in Figure \ref{fig:fourth_place_game_bad}, the simple multibracket in Figure \ref{fig:fourth_place_game_very_bad} is weakly proper: for both (or technically, all four) of its brackets, teams that lost more recently are given more byes than teams that lost earlier. However, it still feels unfair in some sense: the losers of $\bracklabel{B1}$ and $\bracklabel{B2}$ lost in the same round and so they ought to be on the same foot. Instead, the loser of $\bracklabel{B1}$ has to win two games just to get in the same spot that the loser of $\bracklabel{B2}$ starts in. We can use the already-established language of tiered seeding to express this.

    \begin{definition}{Weakly Respectful Multibrackets}{}
        A multibracket is \textit{weakly respectful} if each of its brackets weakly respects the tiering that groups teams by the letter of the round that they lost, with teams that lost more recently getting at least as many byes.
    \end{definition}

    \begin{definition}{Strongly Respectful Multibrackets}{}
        A multibracket is \textit{strongly respectful} if each of its brackets strongly respects the tiering that groups teams by the letter of the round that they lost.
    \end{definition}

    Thus, the multibrackets of signature $\bracksig{8;0;0;0} \to \bracksig{1} \to \bracksig{4; 2; 0; 0} \to \bracksig{1}$ are weakly and strongly respectful, while ones of signature $\bracksig{8;0;0;0} \to \bracksig{1} \to \bracksig{2; 3; 0; 1; 0} \to \bracksig{1}$ are not respectful at all. As a general rule, we prefer multibrackets that are more respectful over ones that are less so, although there can be other factors that might convince us to use less respectful multibrackets, and under certain circumstances no strongly or weakly respectful brackets will be available.

    One natural question to ask is if there's a property called that fills in the final square in the grid.

    \begin{figg}{Strongly Respectful?}{}
        \begin{center}
            \begin{tabular}{ | c | c |}
            \hline
            & \\
            Weakly Respectful & Strongly Respectful\\
            & \\
            \hline
            & \\
            Weakly Proper & ??? \\
            & \\
            \hline
            \end{tabular}
        \end{center}
    \end{figg}

    Recall that respectfulness involves treating teams that lost in the same round similarly, while properness involves treating teams that lost more recently better. Additionally, the weak properties focus just on which bracket or how many byes teams are getting, while the strong properties try to be more complete. Thus,

    \begin{definition}{Strongly Proper Multibrackets}{}
        A multibracket is \textit{strongly proper} if
        \begin{itemize}
            \item It is weakly proper,
            \item Each of its brackets are proper,
            \item Teams that lost more recently are given the slots of higher seeds than teams that lost less recently.
        \end{itemize}
    \end{definition}

    In the traditional bracket world, the idea that proper brackets were preferable to non-proper ones was pretty well accepted, and when combined with the fundamental theorem, meant that there was a clear best bracket for each bracket signature. While strong properness seems like a natural extension of the accepted notion of properness in traditional brackets, it is not obvious that it is quite as desirable in the context of multibrackets. Let's examine two examples to illustrate this point.
    
    First, consider a simple multibracket with signature $\bracksig{4;4;1;0;0} \to \bracksig{1} \to \bracksig{2;0} \to \bracksig{4;2;0;0}.$ The first three brackets are shown in Figure \ref{fig:top_half} (with the second bracket implied by weak properness left off of the figure).

    \fig{0.75}{top_half}{$\bracksig{4;4;1;0;0} \to \bracksig{1} \to \bracksig{2;0}$}

    All of the teams that didn't win one the first three brackets will play in the third bracket, but what should that look like? Below, a strongly proper bracket is on the left, but a potentially better option is on the right.

    \fig{0.75}{bottom_half}{Two options for $\bracksig{4;2;0;0}$}

    In the strongly proper bracket, there is clear variation between the difficulty of the paths of the round-$\bracklabel{B}$ losers: $\bracklabel{B1}$ both gets a first-round bye and avoids the likely best team $\bracklabel{E1}$ until the finals, while $\bracklabel{B2}$ gets neither of those luxuries. This is fine in the traditional bracket setting when we want to treat the $2$-seed better than the $4$-seed, but in a multibracket this is far from ideal. The alternative bracket on the right distributes the advantages more evenly by having the round-$\bracklabel{B}$ loser that receives the bye face $\bracklabel{E1}$ in the semifinals.
    
    There are certainly still reasons to support the strongly proper bracket over the alternative, but it's not so cut and dry. One mitigating factor is strong respectfulness: if a multibracket is strongly respectful, then each team that lost in the same round is treated identically, and so strong properness will not give out these lopsided advantages. 

    For a second example of strongly proper multibrackets not being quite ideal, recall the strongly proper format in Figure \ref{fig:fourth_place_game} that we began the section with: $\bracksig{8;0;0;0} \to \bracksig{1} \to \bracksig{4; 2; 0; 0} \to \bracksig{1}$.

    The notion of properness in the traditional setting is that, in each round, if the bracket goes to chalk, in each round it is better to be a higher seed than a lower seed. But what happens when $\bracksig{8;0;0;0} \to \bracksig{1} \to \bracksig{4; 2; 0; 0} \to \bracksig{1}$ goes to chalk?

    \fig{0.75}{fourth_place_game_chalk}{$\bracksig{8;0;0;0} \to \bracksig{1} \to \bracksig{4; 2; 0; 0} \to \bracksig{1}$ Chalk}

    In round $\bracklabel{E}$, the matchups are not proper: the 4-seed has an easier opponent than the 3-seed. We could fix this by moving around where each seed starts in the primary bracket, but then the primary bracket would no longer be proper. (We could also fix this by swapping where the losers of $\bracklabel{B1}$ and $\bracklabel{B2}$ go, but this would massively increase the rate of rematches.)

    Overall, strong properness is simply not as nice of a condition in the world of multibrackets as properness was for traditional brackets. While it is still often used, especially when there is a desire to drastically shrink the space of formats that need analyzing such as in monte-carlo simulations of formats, it leaves a lot to be desired.

    \begin{oq}{}{}
        Is there a better alternative property to strong properness?
    \end{oq}

    We conclude the section with an important lemma about weakly proper multibrackets.

    \lemm{}{
        In a weakly proper multibracket $\A$, if the loser of game $\bracklabel{G}$ goes to bracket $\A_j$, then the winner of game $\bracklabel{G}$ will either:
        \begin{enumerate}
            \item Win a bracket $\A_i$ for $i \leq j$, or
            \item Lose a game in $\A_j$.
        \end{enumerate}
    }{
        Let $\A$ be a weakly proper multibracket $\A$, and $\bracklabel{G}$ be a game in $\A$ such that the loser of $\bracklabel{G}$ goes to bracket $\A_j$. Let $t$ be the team that won $\bracklabel{G}$. Assume that $t$ does not win any bracket $\A_i$ for $i \leq j.$ Thus, $t$ must have lost at least one game after playing $\bracklabel{G}$. Upon losing this game, $t$ will have lost more recently than the loser of game $\bracklabel{G}$, and so must fall into bracket $\A_i$ for $i \leq j$. If they fall into $\A_i$ for $i < j$, then again they must lose and again fall into $\A_i$ for $i \leq j$. At some point, then, $t$ must fall into $\A_j$. And since $t$ does not win bracket $\A_j$, they must lose in $\A_j$ as well.
    }{weak_lemma}
}