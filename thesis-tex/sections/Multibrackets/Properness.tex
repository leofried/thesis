\sub{

    % Up until now, we have mostly restricted our study of brackets to that of proper brackets...

    % For traditional brackets, this has a very convincing justification: ... 
    
    % However, we have continued to restrict our study to proper brackets in other domains, in particular, teired seedings and simple mulitbrackets. Its worth ...

    % Let's begin with teried seedings. Consider the bracket signature $\bracksig{2;3;2;0;0}$ and tiered seeding $\bracktier{2, 3, 1, 1}.$ The properly respectful bracket is on the left, but a potentially better bracket is on the right.

    % \fig{0.65}{23200 tiered}{Signature $\bracksig{2;3;2;0;0}$ with Seeding $\bracktier{2, 3, 1, 1}$}

    % In the properly respectful bracket, one tier 3 team gets extremely lucky: they have an easier quarterfinal matchup (a tier 4 team rather than another tier 3 team), and an easier semifinal matchup (a tier 2 team rather than a tier 1 team). In the bracket on the right, the luck is more evenly distributed: the tier 3 team that draw the easier quarterfinal matchup also gets the harder semifinal one.

    % A more trivial but more clear example of the same effect is with bracket signature $\bracksig{4;2;0;0}$ and tiered seeding $\bracktier{5, 1}.$ Again the properly respectful bracket is displayed on the left and the alternative bracket is on the right.
    
    % \fig{0.65}{4200 tiered}{Signature $\bracksig{4;2;0;0}$ with Seeding $\bracktier{5, 1}$}

    % In the properly respectful bracket, one tier 2 team gets a first round bye, and dodges the lone tier 1 team until the final. The alternative bracket distributes the advantage by having the tier 2 team that receives the bye be matched up with tier 1 team in the semifinals.

    % Unfortunately, this notion of distributing the luck more fairly is difficult to make rigorous. Additionally, it requires sometime using non-proper bracket \textit{shapes} meaning we lose access to the powerful fundamental theorem. Finally, we can avoid this odd effect by ensuring that our bracket signatures strongly respect the tiered seedings that are given to us, if possible. (As we will show next section, a strongly respectful bracket will give each team in a given tier the same path to win the tournament.) For these reasons, we focus primarily on properly respectful brackets, even though there is a compelling argument to be made that other brackets might be preferable in certain circumstances.




    % \begin{oq}{}{}
    %     Are properly respectful brackets always best?
    % \end{oq}




}