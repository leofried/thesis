\sub {


In traditional brackets, defining properness is relatively straightforward: the teams are totally ordered by seed, and a bracket is proper if, assuming it goes chalk, in every round is better to be a higher seed than a lower one. Properness in the setting of linear multibrackets is a little trickier, as many teams competing in lower semibracket are not there by virtue of being a particular seed, but instead because they lost an earlier game.

So taking inspiration from the way seeding represents a total ordering on the teams in a traditional bracket, we define a partial order on the teams that might be competing in a given semibracket.

\begin{definition}{Linear Multibracket Tiers}{}
    We say a team $t_i$ is of a higher tier than team $t_j$ if
    \begin{enumerate}[(a)]
        \item Neither $t_i$ and $t_j$ have lost a game yet and $t_i$ is seeded higher,
        \item $t_i$ has lost a game, while $t_j$ has not yet lost one yet.
        \item $t_i$'s most recent loss was from a lower semibracket than $t_j$'s most recent loss, or
        \item $t_i$ most recent loss was from a later round of the same semibracket as $t_j$'s most recent loss.
    \end{enumerate}
    If two teams most recent losses are from the same round of the same semibracket, then they are of the same tier.
\end{definition}

While conditions (a) and (d) are seem logical (condition (a) is just the traditional bracket order, and condition (d) says that teams that made it further in a given semibracket belong to a higher tier), conditions (b) and (c) might be a bit more confusing: why should a team that already lost be of a higher tier than one that has not, and why should a team that lost in a lower semibracket be of a \i{higher} tier than a team that lost in a higher semibracket?

To answer these questions, consider the linear multibracket shape of signature $$\bracksig{2;3;0;0} \to \bracksig{1} \to \bracksig{2;0} \to \bracksig{2;1;1;0}.$$

\fig{1}{tierform1}{$\bracksig{2;3;0;0} \to \bracksig{1} \to \bracksig{2;0} \to \bracksig{2;1;1;0}$}

Before fully defining properness for linear multibrackets, let's try to fill out the starting lines in Figure \ref{fig:tierform1} just based on what feels intuitively proper. The primary bracket is easy enough: we use the proper seeding of $\bracksig{2;3;0;0}$ to fill it out. The second and third brackets are also pretty clear: the second bracket just assigns second place, and so ought to contain only the championship game loser, while the third bracket is a third-place game and so should be played between the two semifinals losers. Filling this all in, we are left with the following linear multibracket.

\fig{1}{tierform2}{$\bracksig{2;3;0;0} \to \bracksig{1} \to \bracksig{2;0} \to \bracksig{2;1;1;0}$}

Last is to fill out the final bracket. It is to played between four teams: the loser of $\bracklabel{D1}$, who just lost the fourth-place game; the loser of $\bracklabel{A1}$, who lost in the first game of the format and hasn't played since; and two teams who haven't played a game at all yet, the $6$- and $7$-seeds. Which teams should go where?

The central idea behind properness is that, if a format goes to chalk, you should never prefer to be a lower seed than a higher seed, or to lose than to win. With this mind, the proper seeding becomes clear. The loser of $\bracklabel{D1}$ should get the double bye to the finals of the fourth bracket: if they didn't, then the $4$- and $5$-seeds might prefer to lose game $\bracklabel{A1}$ rather than risk losing in games $\bracklabel{B1}$ and $\bracklabel{D1}$ and getting a worse starting line in the fourth-place bracket. Similarly, the loser of $\bracklabel{A1}$ should get the single bye: if they didn't, then a team interested in a top-four finish might prefer to be the $6$- or $7$-seed to get a better spot in the fourth-place bracket, rather than the $4$- or $5$-seed and risk losing game $\bracklabel{A1}$ and having to win three more games to claim fourth in the format.

Thus, the proper seeding of $\bracksig{2;3;0;0} \to \bracksig{1} \to \bracksig{2;0} \to \bracksig{2;1;1;0}$ is displayed below.

\fig{1}{tierform3}{$\bracksig{2;3;0;0} \to \bracksig{1} \to \bracksig{2;0} \to \bracksig{2;1;1;0}$}

Returning back to the definition of the linear multibracket tiers, this is why teams that lose in later multibrackets are in a higher tier than teams that lose in earlier ones, and why teams that have already lost are in a higher tier than teams that have not: it matches up with our intuitive notion of what properness is trying to do.

We now formalize the analysis we just conducted to define properness on linear multibrackets.

\begin{definition}{Proper Linear Multibracket}{}
    We a linear multibracket is proper, if, assuming higher-tiered teams always beat lower-tiered ones, then in every round of every semibracket it is better to be a higher-tiered team than a lower-tiered team, where:
    \begin{enumerate}[(a)]
        \item It is better to have already won a semibracket than to have not.
        \item It is better to be competing in the current semibracket than to have not won a previous semibracket and not be competing in the current one.
        \item It is better to have a bye than be playing a game.
        \item It is better to be playing a lower-tiered team than a higher-tiered team.
    \end{enumerate}
\end{definition}

With signatures and properness defined, we can address the question posed last question: does the fundamental theorem apply to linear multibrackets? There are two ways to answer this question. The first is a cheap hack that shows the answer is no, and the second is a more through analysis that also shows the answer is no.

We begin with the cheap hack. Consider the The 1988 Men's College Basketball Maui Invitational, which was a multibracket of signature $\bracksig{8;0;0;0} \to \bracksig{1} \to \bracksig{2;0} \to \bracksig{4;0;0} \to \bracksig{1} \to \bracksig{2;0} \to \bracksig{1}.$

\fig{0.8}{maui_pr}{The 1998 Men's College Basketball Maui Invitational}

The cheap idea that proves the fundamental theorem doesn't apply to linear multibrackets is that we can swap (say) $\bracklabel{A2}$ and $\bracklabel{A3}$ and the resulting multibracket has the same signature and is still proper. Why is this cheap? Because its easily patched over: it would still be a meaningful and important result for the fundamental theorem to be true up the rearranging of teams in the same tier.

Unfortunately, this too is not the case. Consider a linear multibracket containing at some point a bracket of signature $\bracksig{4;2;0;0}$, in which the six teams that are set to play in the bracket (by properness) are one team of a higher tier (which we will name $\bracklabel{G1}$), and five teams of a lower tier (which we will name $\bracklabel{F1}$, $\bracklabel{F2}$, $\bracklabel{F3}$, $\bracklabel{F4}$, and $\bracklabel{F5}$.) What might a proper instantiation of $\bracksig{4;2;0;0}$ look like? In fact, there are two.

\fig{0.95}{4200tiers}{Two Proper Instantiations of $\bracksig{4;2;0;0}$ in a Linear Multibracket}

Because linear multibracket properness doesn't distinguish between teams of the same tier, both options in Figure \ref{fig:4200tiers} are proper. This issue is not fixable by adjusting the wording of the fundamental theorem. The two brackets are more than just a shuffling of same-tiered teams away from each other: they are of a different shape! Thus the fundamental theorem doesn't hold for linear multibrackets: we are left only with the existence half.

\theo{}{
    There is at least one proper linear multibracket with each linear bracket signature.
}{
    Let $\A = \A_1 \to ... \to \A_k.$ We proceed by induction on $k.$ For $k = 1,$ $\A = \A_1$, so the proper semibracket of signature $\A_1$ suffices. For larger $k$, begin with the proper linear multibracket of signature $\A_1 \to ... \to \A_{k-1}$, and then add a semibracket to the end whose shape is the shape of the proper semibracket of signature $\A_k$, and whose seeding is derived by replacing the $1$-seed with the highest-tiered remaining team, and then the $2$-seed with the highest-tiered remaining seed, etc.
}{}

    The fact that the fundamental theorem doesn't hold for linear multibrackets hints that there is more to be investigated: how should we decide which of various proper linear multibrackets should we use? One tool used to answer this question is i{respectfulness}.
}