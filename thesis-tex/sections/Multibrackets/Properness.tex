\sub {
    In traditional brackets, defining properness is relatively straightforward: the teams are totally ordered by seed, and a bracket is proper if, assuming it goes chalk, in every round is better to be a higher seed than a lower one. Properness in the setting of linear multibrackets is a little trickier, as many teams competing in lower semibracket are not there by virtue of being a particular seed, but instead because they lost an earlier game. To account for this, we define a new concept first, \i{labels}, before defining an ordering on these labels and then properness using this ordering.

    One quick note before we proceed: in previous sections, we never formally defined how we name certain games in a linear multibracket. To define properness it will be important to use a specific convention: each game is given a name consisting of a letter and then a number. Every game in a given round of a semibracket must have the same letter, and the letters must be distributed in a way that satisfies two requirements. First, that in a given semibracket later rounds have later letters in the alphabet than earlier rounds. And second, that every round of a lower semibracket has a later letter in the alphabet than every round of an upper semibracket. Finally, numbers are distributed such that no two games have the same name.

    With that established, we define a label.

    \begin{definition}{Label}{}
        A \i{label} in a multibracket is either 
        \begin{enumerate}[(a)]
            \item a seed, or
            \item the name of a game.
        \end{enumerate}
    \end{definition}

    \begin{definition}{Label Used by a Semibracket}{}
        We say the label $\LL$ is \i{used by the semibracket} $\A_i$ in a linear multibracket if $\LL$ is placed on one of the starting lines of $\A_i.$
    \end{definition}

    Each label in a linear multibracket is used by at most one of its semibrackets.

    \begin{definition}{Labels Available to a Semibracket}{}
        Let $\A = \A_1 \to ... \to \A_k$ be a linear multibracket. Then label $\LL$ is \i{available} to the semibracket $\A_i$ if $\LL$ is either
        \begin{enumerate}[(a)]
            \item a seed, or
            \item the name of a game in semibracket $\A_j$ for $j < i$,
        \end{enumerate}
        and $\LL$ is not used by any semibracket $\A_j$ for $j < i.$
    \end{definition}

    Linearity guarantees that every label used by a given semibracket is available to it.

    Traditional brackets (and thus the primary semibracket of a linear multibracket) only have the seeds available to them, and so defining a total order on the labels available to a linear bracket is easy: higher seeds are better and more deserving than lower seeds. We want to develop an analogous ordering for the labels available to a later semibracket: rather than a total ordering, we divide the labels into tiers.

    \begin{definition}{Linear Multibracket Tiers}{}
        Let $\A = \A_1 \to ... \to \A_k$ be a linear multibracket, and let $\LLone$ and $\LLtwo$ be two labels available to the semibracket $\A_i.$ Then label $\LLone$ is of a higher \i{tier} than label $\LLtwo$ if
        \begin{enumerate}[(a)]
            \item $\LLone$ and $\LLtwo$ are both seeds and $\LLone$ is a higher seed, or
            \item $\LLone$ is the name of the game and $\LLtwo$ is a seed, or
            \item $\LLone$ and $\LLtwo$ are both the names of a game, but $\LLone$ is later in the alphabet than $\LLtwo$ is.
        \end{enumerate}
        If $\LLone$ and $\LLtwo$ are names of games in the same round, then they are of the same \i{tier}.
    \end{definition}

    While condition (a) is intuitive, conditions (b) and (c) might be a bit more confusing: why should the name of a game be a higher tier than a particular seed, and and why should the names of games with later letters of the alphabet be of a higher tier than the names of a games with an earlier letter of the alphabet.

    To answer these questions, consider the seven-team linear multibracket shape of signature $$\bracksig{2;3;0;0} \to \bracksig{1} \to \bracksig{2;0} \to \bracksig{2;1;1;0}.$$

    \fig{1}{tierform1}{$\bracksig{2;3;0;0} \to \bracksig{1} \to \bracksig{2;0} \to \bracksig{2;1;1;0}$}

    Before fully defining properness for linear multibrackets, let's try to fill out the starting lines in Figure \ref{fig:tierform1} just based on what feels intuitively proper. The primary bracket is easy enough: we use the proper seeding of $\bracksig{2;3;0;0}$ to fill it out. The second and third brackets are also pretty clear: the second bracket just assigns second place, and so ought to contain only the championship game loser, while the third bracket is a third-place game and so should be played between the two semifinal losers. Filling this all in, we are left with the following linear multibracket.

    \fig{1}{tierform2}{$\bracksig{2;3;0;0} \to \bracksig{1} \to \bracksig{2;0} \to \bracksig{2;1;1;0}$}

    The only remaining choice is how to fill out the final semibracket. There are four labels available: $\bracklabel{D1}$, $\bracklabel{A1},$ 6 and 7. Which label should go where?

    The central idea behind properness is that, if a format goes to chalk, you should never prefer to be a lower seed than a higher seed, or prefer to lose than to win. With this in mind, the proper seeding becomes clear. The loser of $\bracklabel{D1}$ should get the double bye to the finals of the fourth bracket: if they didn't, then the $4$- and $5$-seeds might prefer to lose game $\bracklabel{A1}$ rather than risk losing in games $\bracklabel{B1}$ and $\bracklabel{D1}$ and getting a worse starting line in the fourth-place bracket. Similarly, the loser of $\bracklabel{A1}$ should get the single bye: if they didn't, then a team interested in a top-four finish might prefer to be the $6$- or $7$-seed to get a better spot in the fourth-place bracket, rather than the $4$- or $5$-seed and risk losing game $\bracklabel{A1}$ and having to win three more games to claim fourth in the format.

    Thus, the proper seeding of $\bracksig{2;3;0;0} \to \bracksig{1} \to \bracksig{2;0} \to \bracksig{2;1;1;0}$ is displayed below.

    \fig{1}{tierform3}{$\bracksig{2;3;0;0} \to \bracksig{1} \to \bracksig{2;0} \to \bracksig{2;1;1;0}$}

    This example justifies why conditions (b) and (c) are what they are. We now formalize the analysis we just conducted to define properness on linear multibrackets.

    \begin{definition}{Label Representing a Team}{}
        The label $\LL$ \i{represents} the team $t$ if either
        \begin{enumerate}[(a)]
            \item $\LL$ is a seed and $t$ is the $\LL$-seed, or
            \item $\LL$ is the name of a game and $t$ lost in game $\LL.$
        \end{enumerate}
    \end{definition}

    \begin{definition}{Proper Linear Multibracket}{}
        A linear multibracket is \i{proper}, if, assuming teams representing higher-tiered labels always beat teams representing lower-tiered ones, then in every round of every semibracket it is better to be a team representing a higher-tiered label than a lower-tiered label, where:
        \begin{enumerate}[(a)]
            \item It is better to have already won a semibracket than to have not.
            \item It is better to be competing in a semibracket than to be available to a semibracket but not competing.
            \item It is better to have a bye than be playing a game.
            \item It is better to be playing the team representing a lower-tiered label than the team representing a higher-tiered one.
        \end{enumerate}
    \end{definition}

    With signatures and properness defined, we can address the question posed last section: does the fundamental theorem apply to linear multibrackets? There are two ways to answer this question. The first is a cheap hack that shows the answer is no, and the second is a more thorough analysis that also shows the answer is no.

    We begin with the cheap hack. Consider the 1988 Men's College Basketball Maui Invitational \cite{wiki_maui}, which was a multibracket of signature $\bracksig{8;0;0;0} \to \bracksig{1} \to \bracksig{2;0} \to \bracksig{4;0;0} \to \bracksig{1} \to \bracksig{2;0} \to \bracksig{1}.$

    \fig{0.8}{maui_pr}{1988 Men's College Basketball Maui Invitational}

    The cheap idea that proves the fundamental theorem doesn't apply to linear multibrackets is that we can swap (say) $\bracklabel{A2}$ and $\bracklabel{A3}$ and the resulting multibracket has the same signature and is still proper. Why is this cheap? Because its easily patched over: it would still be a meaningful and important result for the fundamental theorem to be true up to rearranging labels in the same tier.

    Unfortunately, this too is not the case. Consider a linear multibracket containing at some point a bracket of signature $\bracksig{4;2;0;0}$, in which the six teams that are set to play in the bracket (by properness) are one team of a higher tier (which we will name $\bracklabel{G1}$), and five teams of a lower tier (which we will name $\bracklabel{F1}$, $\bracklabel{F2}$, $\bracklabel{F3}$, $\bracklabel{F4}$, and $\bracklabel{F5}$.) What might a proper instantiation of $\bracksig{4;2;0;0}$ look like? In fact there are two.

    \fig{0.95}{4200tiers}{Two Proper Instantiations of $\bracksig{4;2;0;0}$ in a Linear Multibracket}

    Because linear multibracket properness doesn't require that teams in the same tier be treated equally, only that teams in higher tiers be treated better, both options in Figure \ref{fig:4200tiers} are proper. This issue is not fixable by adjusting the wording of the fundamental theorem. The two brackets are more than just a shuffling of same-tiered teams away from each other: they are of a different shape! Thus the fundamental theorem doesn't hold for linear multibrackets: we are left only with the existence half.

    \theo{}{
        There is at least one proper linear multibracket with each linear bracket signature.
    }{
        Let $\A = \A_1 \to ... \to \A_k.$ We proceed by induction on $k.$ For $k = 1,$ $\A = \A_1$, so the proper semibracket of signature $\A_1$ suffices. For larger $k$, begin with the proper linear multibracket of signature $\A_1 \to ... \to \A_{k-1}$, and then add a semibracket to the end whose shape is the shape of the proper semibracket of signature $\A_k$, and whose seeding is derived by replacing the $1$-seed with the highest-tiered remaining label, and then the $2$-seed with the highest-tiered remaining label, etc.
    }{}

    It is reasonable to insist, however, not only that teams of higher tier should be treated better than teams of lower tiers, but also that teams of the same tier should be treated equally. What it means for two teams to be treated equally turns out to be a somewhat nuanced question that is treated in the next section.
}