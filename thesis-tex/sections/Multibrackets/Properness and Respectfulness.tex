\sub {

    % INTRO

    % Unlike in the case of traditional brackets, starting lines in the semibrackets of a linear multibracket can be labeled either with a seed, or with the name of a game from an earlier semibracket.

    % \begin{definition}{Label}{}
    %     A \i{label} in a linear multibracket is either a seed or the name of a game.
    % \end{definition}

    % In the case of traditional brackets, only seeds can be labels, and so there is a clear total order on the labels: higher seeds are better, and lower seeds are 







    % With a notion of linear multibracket signatures established, the next question to answer is what properness looks like in the context of linear multibrackets. And while properness in the primary bracket where all the starting lines are seeded can be defined in the same way as properness on traditional bracket, its trickier to define in consolation brackets, where some or all of the starting lines are filled by losers of games in other brackets.







    % We adapt properness to these new conditions by first defining a partial order that aims to extend the total order on seeds used in traditional brackets to also include teams that lost in an earlier bracket.

    % \begin{definition}{Higher and Lower Tier}{}
    %     If two teams both lost in an earlier bracket, the one that lost in a more recent round (that is, the round with a letter later in the alphabet) is a higher tier.

    %     [[RIGOUR IZE]]
    % \end{definition}

    % From here, we adapt our definition of proper seeding.

    % \begin{definition}{Subproper Semibracket of a Liner Multibracket}{}
    %     We say a semibracket in a linear multibracket is \i{subproper} if, assuming the semibracket goes to chalk, it is better to be of a higher tier than of a lower tier, where:
    %     \begin{itemize}
    %         \item[(a)] It is better to have been already placed in an earlier semibracket than not,
    %         \item[(b)] It is better to be placed in this semibracket than not,
    %         \item[(c)] It is better to have a bye than to play a game, and
    %         \item[(d)] It is better to play a lower-tiered team than a higher-tiered team.
    %     \end{itemize}
    % \end{definition}

    % \begin{definition}{Proper Liner Multibracket}{}
    %     We say a linear multibracket is \i{proper} if each of its semibrackets are sub-proper.
    % \end{definition}


}