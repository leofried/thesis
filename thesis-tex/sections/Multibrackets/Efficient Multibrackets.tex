\sub {

    At the end of the previous section, we considered the format in Figure \ref{fig:fourth_place_game}: a multibracket of signature $\bracksig{8;0;0;0} \to \bracksig{1} \to \bracksig{4; 2; 0; 0} \to \bracksig{1},$ which categorizes the top four teams. If it's important to rank the teams one trough four then that format works well enough.
    
    But if all we care about is which of the teams are in top four, and not the ranking among them, then some of the games are unnecessary. In particular, games $\bracklabel{C1}$ and $\bracklabel{F1}$ could be left unplayed, as both the winner and loser of each of those games finish in the top four.

    The resulting format is shown below.

    \fig{0.75}{800_to_42}{An Efficient Format for Selecting a Top Four}

    The format in Figure \ref{fig:800_to_42} is still a multibracket of rank four. But instead of being composed of four traditional brackets, it is composed of two semibrackets each of which have rank two: one with the $\bracklabel{A}$ and $\bracklabel{B}$ round games, and one with the $\bracklabel{C}$ and $\bracklabel{D}$ round games. And, as desired, there no games played between two teams such that both the winner and loser of each of those games finish in the top four.

    This format has signature $\bracksig{8;0;0}_2 \to \bracksig{4;2}_2$ and we say that it is \textit{efficient}.

    \begin{definition}{Efficient}{}
        A multibracket is \textit{efficient} if there are no games played within it such that both the winner and loser of that game are guaranteed to win a semibracket.
    \end{definition}

    Identifying whether a multibracket is efficient can be done just by looking at its signature.

    \lemm{}{
        In a multibracket $\A$, if the loser of game $\bracklabel{G}$ goes to bracket $\A_j$, then the winner of game $\bracklabel{G}$ will either:
        \begin{enumerate}
            \item Win a bracket $\A_i$ for $i \leq j$, or
            \item Lose a game in $\A_j$.
        \end{enumerate}
    }{
        Let $\A_i$ be a semibracket in $\A$, and $\bracklabel{G}$ be a game in $\A_i$ such that the loser of $\bracklabel{G}$ goes to bracket $\A_j$. Let $t$ be the team that won $\bracklabel{G}$. Assume that $t$ does not win any bracket $\A_i$ for $i \leq j.$ Thus, $t$ must have lost at least one game after playing $\bracklabel{G}$. Upon losing this game, by multibracket rule (3), they must fall into bracket $\A_\ell$ for $i < \ell \leq j$. If they fall into $\A_\ell$ for $\ell < j$, then again they must lose and again by multibracket rule (4) fall into $\A_m$ for $\ell < m \leq j$. At some point, then, $t$ must fall into $\A_j$. And since $t$ does not win bracket $\A_j$ either, they must lose in $\A_j$ as well.
    }{weak_lemma}

    \theo{}{
        A multibracket $\A = \A_1 \to ... \to \A_k$
        is efficient if and only if there is some $j$ such that all brackets $\A_i$ for $i < j$ are trivial and all brackets $\A_i$ for $i \geq j$ are not.
    }{
        Let $\A$ be a multibracket.\\

        Assume no such $j$ exists, and let $i$ be the first trivial bracket that follows a nontrivial one. Thus there is at least one game $\bracklabel{G}$ such that the loser drops into $\A_i.$ Because $\A_i$ is trivial, the loser of $\bracklabel{G}$ wins $\A_i$. Applying Lemma \ref{th:weak_lemma}, we see that the winner of $\bracklabel{G}$ will either win a semibracket as well, or lose in $\A_i$. But $\A_i$ is trivial, so they must win a semibracket. Thus, $\A$ is not efficient.\\

        Now assume that such a $j$ exists. We will show by inducting on the semibrackets in $\A$ in reverse that none of the semibrackets contain a game that violates the efficiency condition. Firstly, $\A_k$ upholds the condition because any team that loses a game in $\A_k$ doesn't fall into another semibracket, much less have a chance to win one.\\
        
        Now we must show that if all of the semibrackets from $\A_{i+1}$ to $\A_k$ uphold the condition, then $\A_i$ does as well. If $i<j$, then $\A_i$ is trivial so there are no games to violate the condition with. Otherwise, let $\bracklabel{G}$ be a game in $\A_i$. If the loser of $\bracklabel{G}$ does not fall into another semibracket, then we are done. If they do, then because that bracket is not trivial, they will play another game. However, by induction, the loser of this game is not guaranteed to win a semibracket. Thus neither is the loser of $\bracklabel{G}$.\\

        So by induction, if such a $j$ exists, then $\A$ is efficient. Thus we have proved the theorem.
    }{}

    Another example of an efficient multibracket is the 2021 NBA Western Conference Play-in Tournament, which was a ten-team multibracket with order eight and the following signature:
    $\bracksig{6}_6 \to \bracksig{2;0}_1 \to \bracksig{2;1;0}_1.$ The play-in tournament was used to whittle the top ten teams in the conference down to eight teams who would qualify for the playoffs.

    \fig{0.7}{nba_playin}{2021 NBA Western Conference Play-in}

    Finally, the USA Ultimate Manual of Championship Series Tournament Formats \cite{ultimate}, which is used to determine the formats to be used at the various sectional and regional tournaments in the sport of ultimate frisbee, contains a host of efficient multibrackets for selecting the top $m$ teams out of a list of $n$ for $m$ and $n$ ranging from $1$ to $24.$

    Efficient multibrackets are great tournament designs for tournaments whose primary goal is to select the top $m$ teams to move on to the next stage of the competitions, as discussed in the beginning of this section. They do so excitingly, with each spot in the top $m$ being awarded as the winner of a particular game; efficiently, with no games being played between teams who have each already clinched spots; and fairly, as the multibracket rules ensure that winning is always better than losing.
}