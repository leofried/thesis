\sub {

    Consider the 1998 Men's College Basketball Maui Invitational, which used a proper linear multibracket of signature $\bracksig{8;0;0;0} \to \bracksig{1} \to \bracksig{2;0} \to \bracksig{1} \to \bracksig{4;0;0} \to \bracksig{1} \to \bracksig{2;0} \to \bracksig{1}.$

    \fig{0.7}{maui}{The 1998 Men's College Basketball Maui Invitational}

    The format used in the Maui Invitational has a number of nice properties. Firstly, its proper and strongly respectful. Second, unlike some of the efficient linear multibrackets we saw last section, every team starts in the primary bracket, ensuring that if a team wins every game they will win the tournament. Third, the sum of the ranks of the semibrackets that make it up sum to the number of participating teams, ensuring that every team ends up ranked.

    Fourth, every team plays the same number of games, in this case three. And fifth every game is between teams with the same record, hopefully leading to evenly matched and exciting games. Linear multibracket with all five of these properties are called \i{swiss formats}, named because of their first recorded use at a chess tournament in Zürich, Switzerland. %cite

    \begin{definition}{Swiss Formats}{}
        A \i{swiss formats} is a proper minimally respectful linear multibracket with the following four properties.
        \begin{itemize}
            \item Every team starts in the primary semibracket.
            \item Every team wins a semibracket.
            \item Every team plays the same number of games.
            \item Every game is between two teams with the same record.
        \end{itemize}
    \end{definition}

    Note that even though the Maui Invitational was strongly respectful, we only require minimal respectfulness. 
    %   However,
    % --> Do we need to require that teams in the same semi are the same record? i think no bc of properness
    % \theo{}{
    %     Every swiss signature is round-respectful.
    % }{
    %     Every team playing in a given round of a given semibracket must have played the same number of games
    % }{}

    \begin{definition}{$r$-Round Swiss}{}
        We say a swiss format in which each team plays $r$ games is an $r$-round swiss format.
    \end{definition}

    Thus the 1998 Men's College Basketball Maui Invitational was an 8-team 3-round swiss format.

    Often times we will interested in just the signature of a swiss format, rather than the specific details of the entire format.

    \begin{definition}{Swiss Signature}{}
        A \i{swiss signature} is a linear multibracket signature that admits a swiss format.
    \end{definition}

    Thus $\bracksig{8;0;0;0} \to \bracksig{1} \to \bracksig{2;0} \to \bracksig{1} \to \bracksig{4;0;0} \to \bracksig{1} \to \bracksig{2;0} \to \bracksig{1}$ is a swiss signature. 
    
    % We now look to identify, name, and analyze the various 1-, 2-, 4-, and 8-team swiss signatures.


The first key family of swiss signatures are the \i{standard swiss signatures}, which we abbreviate by $\S_r$ for some $r$.

\begin{definition}{Standard Swiss Signature $(\S_r)$}{standard_swiss}
    $\S_r,$ or the \i{standard $r$-round swiss signature}, is the multibracket signature defined recursively by $$\S_0 = \bracksig{1},$$ and
    $$\S_r = \bracksig{2^r; ...; 0} \to \S_{0} \to \S_1 \to ... \to \S_i \to ... \to \S_{r-1}.$$
\end{definition}

Thus we have
\begin{align*}
    \S_0 =\; &\bracksig{1}\\
    \S_1 =\; &\bracksig{2;0} \to \bracksig{1}\\
    \S_2 =\; &\bracksig{4;0;0} \to \bracksig{1} \to \bracksig{2;0} \to \bracksig{1}\\
    \S_3 =\; &\bracksig{8;0;0;0} \to \bracksig{1} \to \bracksig{2;0} \to \bracksig{1} \to \bracksig{4;0;0} \to \bracksig{1} \to \bracksig{2;0} \to \bracksig{1}
\end{align*}

Figure \ref{fig:three_small_systems} displays instances of $\S_0, \S_1,$ and $\S_2$, while the 1998 Men's College Basketball Maui Invitational was an instance of the standard swiss signature $\S_3.$

\fig{1}{three_small_systems}{$\S_0, \S_1,$ and $\S_2$}

    The standard swiss signatures are particularly nice: in addition to the other swiss format requirements, the primary semibracket of a standard swiss format has rank one, so a single champion is crowned. Not every swiss format has this property: consider, for example, the following 8-team swiss format.

    \fig{1}{two_by_two}{$\bracksig{8;0;0}_2 \to \bracksig{2}_2 \to \bracksig{4;0}_2 \to \bracksig{2}_2$}

    The format in Figure \ref{fig:two_by_two} is a swiss format, but it doesn't crown an individual champion, as two teams end the format undefeated. The swiss signature $\bracksig{8;0;0}_2 \to \bracksig{2}_2 \to \bracksig{4;0}_2 \to \bracksig{2}_2$ is not \i{compact}.

   \begin{definition}{Compact}{}
        We say a swiss signature is \i{compact} if its primary semibracket has rank one.
   \end{definition}

   We can think of non-compact swiss signature as multiple copies of another, smaller swiss signature running in parallel. %prove?
   For example, the format in Figure \ref{fig:two_by_two} can be viewed as two instances of the format $\S_2$ being played side by side. We use this to introduce $m\x$ notation. 

   \begin{definition}{$m\x\A$}{mx_notation}
        If $m \in \N$ and $\A$ is a multibracket signature, than $m\x\A$ is the multibracket signature formed by multiplying every number in every signature in $\A$ by $m$.
    \end{definition}

    So $\bracksig{8;0;0}_2 \to \bracksig{2}_2 \to \bracksig{4;0}_2 \to \bracksig{2}_2 = 2\x\S_2.$

    With the standard swiss signature and $m\x$ notation defined, we are ready for Figure \ref{fig:swiss_names}, which details the various swiss signatures for 1-, 2-, 4-, and 8-teams.

    \begin{figg}{The 1-, 2-, 4-, and 8-team Swiss Signatures}{swiss_names}
        \begin{center}
            \begin{tabular}{ c | c | c | c | c}
                & 1 Team & 2 Teams & 4 Teams & 8 Teams\\
                \hline
                0 Rounds & $\S_0$ & $2\x\S_0$ & $4\x\S_0$ & $8\x\S_0$\\
                \hline
                1 Round & & $\S_1$ & $2\x\S_1$ & $4\x\S_1$\\
                \hline
                2 Rounds & & & $\S_2$ & $2\x\S_2$\\
                \hline
                \multirow{1}{*}{3 Rounds} & & & &  $\S_3, \T_3$ \\
            \end{tabular}
        \end{center}
        \end{figg}

    The swiss signatures on the diagonal of Figure \ref{fig:swiss_names} are the compact ones. While standard swiss signature and $m\x$ notation are sufficient for explaining almost every signature in Figure \ref{fig:swiss_names}, there is a second 8-team 3-round swiss signature, $\T_3,$ that we have yet to define. It's worth attempting to construct $\T_3$ before looking at Figure \ref{fig:swisst3}.

    \fig{1}{swisst3}{$\T_3$}

    So $\T_3 = \bracksig{8;0;0;0} \to \bracksig{1} \to \bracksig{4;2;0}_2 \to \bracksig{2}_2 \to \bracksig{2;0} \to \bracksig{1}.$ $\T_3$ is also very similar to the format used in Figure \ref{fig:socon} by the 2023 Southern Conference Wrestling Championships: both use an primary eight-team balanced bracket and let their first-round losers fight their way back for a top-half finish.

    \theo{}{
        $\S_3$ and $\T_3$ are the only compact 3-round swiss signatures.
        }{
        Any compact $3$-round swiss signature must begin with  $\bracksig{8;0;0;0} \to \bracksig{1}.$ Now let $\A$ be the semibracket that first-round primary brackets losers fall into. $\A$ must have two rounds, and the first-round primary bracket losers must all get no byes (otherwise they would not play the requisite three games). Thus $\A = \bracksig{4;a_1;0}_{(a_1/2+1)}$ for some $a_1$. As neither of the two semifinal winners can fall into $\A$, $a_1 \leq 2.$ Additionally, if $a_1 = 1$, $\A$ would not be a signature. Thus, $a_1 = 0$ or $2$.\\
        
         If $a_1 = 0$, then in between the first two brackets and $\A$, we must have two more brackets for the second-round losers of the primary bracket: $\bracksig{2;0}$ and $\bracksig{1}.$ Then $\A$ must be followed by $\bracksig{1}$ for the loser of its championship game, and then $\bracksig{2;0}$ and $\bracksig{1}$ so that the last two teams get a third game. This is the swiss signature $\S_3$.\\
        
         If $a_1 = 2$, then the losers of the two championship games of $\A$ have already played all three of their games and so need to fall into the bracket $\bracksig{2}$. Then we need $\bracksig{2;0}$ and $\bracksig{1}$ so that the last two teams get a third game. This is the swiss signature $\T_3$.
        }{}

Figure \ref{fig:swiss_names} tells us that there are five $8$-team swiss signatures. How would a tournament designer decide which 3-round signature to use? Well, it depends on what the prize structure of the format is. If the goal is to identify a top-three, then signature $\S_3$ is preferable: signature $\T_3$ doesn't even recognize a third-place, instead assigning fourth-place to two teams. But if the goal is to identify a top-four, signature $\T_3$ is preferable: the team that comes in fourth in signature $\S_3$ actually finishes with only one win, while the team the comes in fifth finishes with two. While it is still reasonable to grant the one-win team fourth-place -- they had a more difficult slate of opponents -- this is a somewhat messy situation that is solved by just using signature $\T_3.$

(McGarry and Schutz \cite{four_five_swap} considered outright swapping the positions of the fourth- and fifth-place teams at the conclusion of $\S_3$, but this provides some incentive for losing in the first round in order to get an easier path to a top-half finish. Simply using $\T_3$ when identifying the top-four teams is preferable.)

For similar reasons, both formats are good for selecting a top-one or top-seven, and $\S_3$ but not $\T_3$ is good for selecting a top-five. Finally, it might seem that $\S_3$ and $\T_3$ are good formats for selecting a top-two or top-six: in both cases the top two and top six teams are clearly defined, and there are no teams with better records that don't make the cut. However, notice that if we use $\S_3$ or $\T_3$ to select a top-two, the final round of games are meaningless: the two teams that finish in the top-two are the two teams that win their first two games, irrespective or how game three went. Better than using either format $\S_3$ or $\T_3$ would be to use the non-compact $2\x\S_2,$ shortening the format down to two rounds without losing any important games.

There are eight compact 4-round signatures: $\S_4$ and seven others. The count of compact $r$-round signatures for general $r$, however, is still an open question.

\begin{conj}{}{}
    Let $s_r$ be the number of compact $r$-round swiss signatures. Then $s_r$ is given by:
    \begin{align*}
        s_0 &= s_ 1 = 1\\
        s_r &= s_{r-1} \cdot \sum_{i=1}^{r-1}s_i
    \end{align*}
\end{conj}

Overall, swiss signatures very useful and practical tournament designs: they give each team the same number of games, they ensure that games are being played between teams that have the same record and thus, hopefully, similar skill levels, and, for many values of $m$, they efficiently identify a top-$m$ in a fair and satisfying way.

Further, swiss or near-swiss signatures are great when the number of teams is exceedingly large. Even if the number of teams is not a power of two, or the signature is not compact, or there is a round at the end that doesn't affect placement, formats that are swiss in spirit tend to do a great job of gathering a lot of meaningful data about a large number of teams in a small number of rounds.








    
%     However, it is not the only eight-team swiss signature: $\bracksig{8;0}_4 \to \bracksig{4}_4$ is another one.

%     \phig{four_by_two}{$\bracksig{8;0}_4 \to \bracksig{4}_4$}

%    But the swiss signature $\bracksig{8;0}_4 \to \bracksig{4}_4$ is not compact.

%    \begin{definition}{Compact}{}
%         We say a swiss signature is \i{compact} if its primary semibracket has rank one.
%    \end{definition}

%    Non-compact swiss formats are usually just multiple copies of a smaller swiss formats running simultaneously. For example, $\bracksig{8;0}_4 \to \bracksig{4}_4$ can be thought of four copies of the compact two-team swiss signature $\bracksig{2;0} \to \bracksig{1}$ running in parallel. Thus we introduce $m\x$ notation.

%    \begin{definition}{$m\x\A$}{mx_notation}
%         If $m \in \N$ and $\A$ is a multibracket signature, than $m\x\A$ is the multibracket signature formed by multiplying every number in every signature in $\A$ by $m$.
%     \end{definition}

%     So $\bracksig{8;0}_4 \to \bracksig{4}_4$ is the same as $4\x(\bracksig{2;0} \to \bracksig{1}).$


    
    
    
    
    % It is not the only eight-team swiss signature, but before counting eight-team swiss signatures, we will start smaller.

    % There is only one one-team linear multibracket signature, $\bracksig{1},$ and it is indeed a zero-round swiss signature. There are two two-team swiss signatures: the zero-round $\bracksig{2}_2$, and the one-round $\bracksig{2;0} \to \bracksig{1}.$ And there are three 

    % \phig{two_team}{The Two Two-Team Swiss Formats}

    % However, the two-team swiss signatures are quite diff





}



% Many tournaments, particularly those in which there are many teams each looking to play a similar number of games against teams of similar skill, use a set of formats referred to as \i{swiss signatures.} In particular, swiss signatures or near-variants are commonly used in board game tournaments, such as chess or Magic: The Gathering.

% The idea behind a swiss signature is to play a fixed number of rounds, and in each round have each matchup be between teams with the same record. This gives every team a bunch of games, while ensuring that teams are paired with teams that are probably similarly skilled. We can formally describe a swiss signature in the language of multibrackets.

% \begin{definition}{Simple Multibracket}{}
%     We say a multibracket is \i{simple} if all teams start in the primary semibracket.
% \end{definition}

% \begin{definition}{Complete Multibracket}{}
%     We say a multibracket is \i{complete} if it ranks every team.
% \end{definition}

% \begin{definition}{Swiss Signature}{}
%     A \i{swiss signature} is a simple complete multibracket signature in which all teams play the same number of games and each matchup is between teams of the same record.
% \end{definition}

% One key family of swiss signatures are the \i{standard signatures}, $\S_r,$ which operate on $2^r$ teams over $r$ rounds. We will define them formally soon, but first, we display the first four standard signature $\S_0, \S_1, \S_2,$ and $\S_3.$

%  % label graphic
% \fig{0.8}{three_small_signatures}{$\S_0, \S_1,$ and $\S_2$}

% The 1988 Men's College Basketball Maui Invitational used the format $\S_3.$

% \fig{0.7}{maui}{The 1998 Men's College Basketball Maui Invitational ($\S_3$)}

% The intuition behind the standard swiss signatures is simple: each team plays a first-round game, and then the winners and losers each independently play out $\S_{r-1}.$ We can formalize this into a signature:

% \begin{definition}{Standard Swiss Signature $(\S_r)$}{standard_swiss}
%     $\S_r,$ or the \i{standard swiss signature on $2^r$ teams}, is the multibracket signature defined recursively by $$\S_0 = \bracksig{1},$$ and
%     $$\S_r = \bracksig{2^r; ...; 0} \to \S_{0} \to \S_1 \to ... \to \S_{r-1}.$$
% \end{definition}

% So,

% \begin{figg}{Standard Swiss Signatures}{std_swiss}
% \begin{center}
%     \overfullhbox{
%     \begin{tabular}{ c |  l }
%         Signature & Signature\\
%         \hline
%         $\S_0$ & $\bracksig{1}$\\
%         $\S_1$ & $\bracksig{2;0} \to \bracksig{1}$\\
%         $\S_2$ & $\bracksig{4;0;0} \to \bracksig{1} \to \bracksig{2;0} \to \bracksig{1}$\\
%         $\S_3$ & $\bracksig{8;0;0;0} \to \bracksig{1} \to \bracksig{2;0} \to \bracksig{1} \to \bracksig{4;0;0} \to \bracksig{1} \to \bracksig{2;0} \to \bracksig{1}$\\
%     \end{tabular}
%     }
% \end{center}
% \end{figg} 

% Figure \ref{fig:std_swiss} matches with Figures \ref{fig:three_small_signatures} and \ref{fig:maui}.

% One thing to note about the standard swiss signatures is that there is no chance of a rematch.

% \theo{}{
%     Standard swiss signatures will never play a rematch.
% }{
%     We show this by induction on $r$. No games, and thus no rematches, are played in $\S_0.$ For any other $r$, after the first round, teams are separated into two instances of $\S_{r-1}$. No team in the same instance of $\S_{r-1}$ played each other in the first round, and, by induction, no rematches occur within either instance of $\S_{r-1}$. Thus, $\S_r$ never plays a rematch.
% }{swiss_no_rematches}

% Theorem \ref{th:swiss_no_rematches} is made a bit weaker but more general in Theorem \ref{th:swiss_most_no_rematches}.

% \theo{}{
%     Every swiss signature has a multibracket instantiation that will never play a rematch.
% }{
%     %proof_needed
% }{swiss_most_no_rematches}

% What do some of the non-standard swiss signatures look like? Well for one thing, we can run multiple instances of the same standard signature at the same time to form a bigger signature. Figure \ref{fig:8242}, for example, displays the signature created by running two copies of $\S_2$ simultaneously.

% \fig{1}{8242}{$\bracksig{8;0;0} \to \bracksig{2} \to \bracksig{4;0} \to \bracksig{2}$}

% The signature of running $m$ instances of a particular multibracket at the same time is just the signature of the original bracket, with every number multiplied by $m$. Because of this, the swiss signature in Figure \ref{fig:8242} is sometimes referred to as 2x$\S_2.$

% \begin{definition}{$m\x\A$}{mbya}
%     If $m \in \N$ and $\A$ is a multibracket signature, than $m\x\A$ is the multibracket signature formed by multiplying every number in every signature in $\A$ by $m$.
% \end{definition}

% Swiss signatures formed by the construction in \ref{def:mbya} are somewhat less interesting to study than other signatures, so we introduce the notion of a \i{compact} signature.

% \begin{definition}{Compact Swiss Signature}{}
%     We say a swiss signature is \i{compact} if its primary semibracket is a bracket.
% \end{definition}

% Thus, for all $r$, $\S_r$ is compact, but $m$x$\S_r$ is not for $m > 1.$ We can return now to the question of what a compact non-standard swiss signature would look like.

% As it happens, $\S_0$, $\S_1$, and $\S_2$ are the only compact 0-, 1-, and 2-round signatures, respectively.

% \theo{}{
%     There is only one compact $r$-round signature from $r \leq 2.$
% }{
% Any 0-round signature must be the unique 1-team multibracket $\S_0.$ Likewise, any 1-round signature must be on two teams, and simple and complete, making it equal $\S_1$. Finally, any compact 2-round signature will have a primary semibracket of signature $\bracksig{4;0;0}$. The loser of the primary bracket has no more games to play, so the secondary bracket must be $\bracksig{1}$. After that, there are two teams each with one game left, so they play each other, completing the final two brackets $\bracksig{2;0} \to \bracksig{1},$ which comes out to just $\S_2.$
% }{}

% $\S_3$ is not the unique compact $3$-round signature, however. The other signature, sometimes referred to as $\T_3$, is displayed in Figure \ref{fig:swisst3}. (Note that there are instantiations of $\T_3$ that risk playing rematches. As per Theorem \ref{th:swiss_most_no_rematches}, we show the instantiation that does not).

% \fig{1}{swisst3}{$\T_3 = \bracksig{8;0;0;0} \to \bracksig{1} \to \bracksig{4;2;0}_2 \to \bracksig{2}_2 \to \bracksig{2;0} \to \bracksig{1}$}

% \theo{}{
% $\S_3$ and $\T_3$ are the only compact 3-round swiss signatures.
% }{
% Any compact $3$-round swiss signature must have primary bracket $\bracksig{8;0;0;0}$ and secondary bracket $\bracksig{1}.$ Now let $\A$ be the semibracket that first-round primary brackets losers fall into. $\A$ must have two rounds, and the first-round primary bracket losers must all get no byes (otherwise they would not play the requisite three games). Thus $\A = \bracksig{4;a_1;0}_{(a_1/2+1)}$ for some $a_1$. As the primary and secondary bracket each have winners, $a_1 \leq 2.$ Additionally, if $a_1 = 1$, $\A$ would not be a signature. Thus, $a_1 = 0$ or $2$.\\

%  If $a_1 = 0$, then in between the first two brackets and $\A$, we must have two more brackets for the second-round losers of the primary bracket: $\bracksig{2;0}$ and $\bracksig{1}.$ Then $\A$ must be followed by $\bracksig{1}$ for the loser of its championship game, and then $\bracksig{2;0}$ and $\bracksig{1}$ so that the last two teams get a third game. This is the swiss signature $\S_3$.\\

%  If $a_1 = 2$, then the losers of the two championship games of $\A$ have already played all three of their games and so need to fall into the bracket $\bracksig{2}$. Then we need $\bracksig{2;0}$ and $\bracksig{1}$ so that the last two teams get a third game. This is the swiss signature $\T_3$.
% }{}
% How would a tournament designer decide which compact 3-round signature to use? Well, it depends on what the prize structure of the format is. If the goal is to identify a top-three, then signature $\S_3$ is preferable: signature $\T_3$ doesn't even recognize a third-place, instead assigning fourth-place to two teams. But if the goal is to identify a top-four, signature $\T_3$ is preferable: the team that comes in fourth in signature $\S_3$ actually finishes with only one win, while the team the comes in fifth finishes with two. While it is still reasonable to grant the one-win team fourth-place -- they had a more difficult slate of opponents -- this is a somewhat messy situation that is solved by just using signature $\T_3.$

% (McGarry and Schutz \cite{four_five_swap} considered outright swapping the positions of the fourth- and fifth-place teams at the conclusion of $\S_3$, but this provides some incentive for losing in the first round in order to get an easier path to a top-half finish. Simply using $\T_3$ when identifying the top-four teams is preferable.)

% For similar reasons, both formats are good for selecting a top-one or top-seven, and $\S_3$ but not $\T_3$ is good for selecting a top-five. Finally, it might seem that $\S_3$ and $\T_3$ are good formats for selecting a top-two or top-six: in both cases the top two and top six teams are clearly defined, and there are no teams with better records that don't make the cut. However, notice that if we use $\S_3$ or $\T_3$ to select a top-two, the final round of games are meaningless: the two teams that finish in the top-two are the two teams that win their first two games, irrespective or how game three went. Better than using either format $\S_3$ or $\T_3$ would be to use the non-compact $2\x\S_2,$ shortening the format down to two rounds without losing any important games.

% There are eight compact 4-round signatures: $\S_4$ and seven others. The count of compact $r$-round signatures for general $r$, however, is still an open question.

% \begin{conj}{}{}
%     Let $s_r$ be the number of compact $r$-round swiss signatures. Then $s_r$ is given by:
%     \begin{align*}
%         s_0 &= s_ 1 = 1\\
%         s_r &= s_{r-1} \cdot \sum_{i=1}^{r-1}s_i
%     \end{align*}
% \end{conj}

% Overall, swiss signatures very useful and practical tournament designs: they give each team the same number of games, they ensure that games are being played between teams that have the same record and thus, hopefully, similar skill levels, and, for many values of $m$, they efficiently identify a top-$m$ in a fair and satisfying way.

% Further, swiss or near-swiss signatures are great when the number of teams is exceedingly large. Even if the number of teams is not a power of two, or the signature is not compact, or there is a round at the end that doesn't affect placement, formats that are swiss in spirit tend to do a great job of gathering a lot of meaningful data about a large number of teams in a small number of rounds.
