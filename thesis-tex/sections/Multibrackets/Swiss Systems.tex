\sub {

%capitalize the word swiss and chapter name should not be system
% https://app.diagrams.net/

    Consider the 1998 Men's College Basketball Maui Invitational, which used a proper linear multibracket of signature $\bracksig{8;0;0;0} \to \bracksig{1} \to \bracksig{2;0} \to \bracksig{1} \to \bracksig{4;0;0} \to \bracksig{1} \to \bracksig{2;0} \to \bracksig{1}.$ %recall this was used before

    \fig{0.7}{maui}{The 1998 Men's College Basketball Maui Invitational}

    The format used in the Maui Invitational has several nice properties. First, it is strongly respectful. Secondly, unlike some of the efficient linear multibrackets we saw last section, every team starts in the primary bracket, ensuring that if a team wins every game they will win the tournament. Third, the sum of the ranks of its semibrackets sum to the number of participating teams, ensuring that every team ends up ranked. Fourth, every bracket is either trivial or competitive, ensuring that the final rankings are as granular as possible.

    Fifth, every team plays the same number of games, in this case, three. And sixth, every game is between teams with the same record, hopefully leading to evenly matched and exciting games. Linear multibracket with all six of these properties are called \i{swiss formats}, named because of their first recorded use at a chess tournament in Zürich, Switzerland in 1895. %cite

    \begin{definition}{Swiss Formats}{swiss}
        A \i{swiss format} is a minimally respectful linear multibracket with the following five properties.
        \begin{itemize}
            \item Every team starts in the primary semibracket.
            \item Every team wins a semibracket.
            \item Every semibracket is either trivial or competitive.
            \item Every team plays the same number of games.
            \item Every game is between two teams with the same record.
        \end{itemize}
    \end{definition}

    Note that even though the Maui Invitational was strongly respectful, we only require minimal respectfulness: this allows for a substantially larger space of swiss formats while still maintaining the important properties of the class of formats.

    \begin{definition}{$r$-Round Swiss}{}
        We say a swiss format in which each team plays $r$ games is an $r$-round swiss format.
    \end{definition}

    Thus the 1998 Men's College Basketball Maui Invitational was an 8-team 3-round swiss format.

    Often times, we will interested in just the signature of a swiss format, rather than the specific details of the entire format.

    \begin{definition}{Swiss Signature}{}
        A \i{swiss signature} is a linear multibracket signature that admits a swiss format.
    \end{definition}

    Thus $\bracksig{8;0;0;0} \to \bracksig{1} \to \bracksig{2;0} \to \bracksig{1} \to \bracksig{4;0;0} \to \bracksig{1} \to \bracksig{2;0} \to \bracksig{1}$ is a swiss signature.

    Without a close inspection, it can be difficult to see from the diagram, and certainly from the signature, that the Maui Invitational meets all the requirements of a swiss system: the last one in particular is tricky to confirm. But there is another diagram that we can use to depict swiss signatures in much more intuitive way: flowcharts. Figure \ref{fig:s3_flow} depicts the flow chart for the 1998 Men's College Basketball Maui Invitational. %maybe prove something about flowchart = signature

    \fig{0.4}{s3_flow}{The Maui Invitational Flowchart}


    The signature used in the Maui Invitational is a particular example of a family of swiss signature known as the the \i{standard swiss signatures}, which we abbreviate by $\S_r$ for some $r$.

    \begin{definition}{Standard Swiss Signature $(\S_r)$}{standard_swiss}
        $\S_r,$ or the \i{standard $r$-round swiss signature}, is the multibracket signature defined recursively by $$\S_0 = \bracksig{1},$$ and
        $$\S_r = \bracksig{2^r; ...; 0} \to \S_{0} \to \S_1 \to ... \to \S_i \to ... \to \S_{r-1}.$$
    \end{definition}

    Thus we have
    \begin{align*}
        \S_0 =\; &\bracksig{1}\\
        \S_1 =\; &\bracksig{2;0} \to \bracksig{1}\\
        \S_2 =\; &\bracksig{4;0;0} \to \bracksig{1} \to \bracksig{2;0} \to \bracksig{1}\\
        \S_3 =\; &\bracksig{8;0;0;0} \to \bracksig{1} \to \bracksig{2;0} \to \bracksig{1} \to \bracksig{4;0;0} \to \bracksig{1} \to \bracksig{2;0} \to \bracksig{1}
    \end{align*}

    Intuitively, you can think of the $\S_r$ as a $2^r$ team tournament, where, after the first round of games, the winners and losers each go off and play through separate instances of $\S_{r-1}.$

    Figures \ref{fig:three_small_systems} and \ref{fig:three_small_flow} display $\S_0, \S_1,$ and $\S_2$ as a linear multibracket and as a flowchart, while the 1998 Men's College Basketball Maui Invitational was an instance of the standard swiss signature $\S_3.$

    \fig{1}{three_small_systems}{$\S_0, \S_1,$ and $\S_2$}
    \fig{0.4}{three_small_flow}{$\S_0, \S_1,$ and $\S_2$}

    The standard swiss signatures are particularly nice: in addition to the other swiss format requirements, the primary semibracket of a standard swiss format has rank one, so a single champion is crowned. Not every swiss format has this property: consider, for example, the following 8-team swiss format.

    \fig{1}{two_by_two}{$\bracksig{8;0;0}_2 \to \bracksig{2}_2 \to \bracksig{4;0}_2 \to \bracksig{2}_2$}
    
    \fig{0.4}{2xs2_flow}{$\bracksig{8;0;0}_2 \to \bracksig{2}_2 \to \bracksig{4;0}_2 \to \bracksig{2}_2$}

    The format in Figure \ref{fig:two_by_two} is a swiss format, but it doesn't crown an individual champion, as two teams end the format undefeated. The swiss signature $\bracksig{8;0;0}_2 \to \bracksig{2}_2 \to \bracksig{4;0}_2 \to \bracksig{2}_2$ is not \i{compact}.

   \begin{definition}{Compact}{}
        We say a swiss signature is \i{compact} if its primary semibracket has rank one.
   \end{definition}

    We can think of a non-compact swiss signature as multiple copies of another, smaller swiss signature running in parallel. For example, the format in Figures \ref{fig:two_by_two} and \ref{fig:2xs2_flow} can be viewed as two independent instances of the format $\S_2$ being played side by side. We use this to introduce $m\x$ notation.  %prove?

   \begin{definition}{$m\x\A$}{mx_notation}
        If $m \in \N$ and $\A$ is a multibracket signature, then $m\x\A$ is the multibracket signature formed by multiplying every number in every signature in $\A$ by $m$.
    \end{definition}

    So $\bracksig{8;0;0}_2 \to \bracksig{2}_2 \to \bracksig{4;0}_2 \to \bracksig{2}_2 = 2\x\S_2.$

    With the standard swiss signature and $m\x$ notation defined, we are ready for Figure \ref{fig:swiss_names}, which details the various swiss signatures for 1-, 2-, 4-, and 8-teams.

    \begin{figg}{The 1-, 2-, 4-, and 8-team Swiss Signatures}{swiss_names}
        \begin{center}
            \begin{tabular}{ c | c | c | c | c}
                & 1 Team & 2 Teams & 4 Teams & 8 Teams\\
                \hline
                0 Rounds & $\S_0$ & $2\x\S_0$ & $4\x\S_0$ & $8\x\S_0$\\
                \hline
                1 Round & & $\S_1$ & $2\x\S_1$ & $4\x\S_1$\\
                \hline
                2 Rounds & & & $\S_2$ & $2\x\S_2$\\
                \hline
                \multirow{1}{*}{3 Rounds} & & & &  $\S_3, \T_3$ \\
            \end{tabular}
        \end{center}
        \end{figg}

    The swiss signatures on the diagonal of Figure \ref{fig:swiss_names} are the compact ones. While standard swiss signature and $m\x$ notation are sufficient for explaining almost every signature in Figure \ref{fig:swiss_names}, there is a second 8-team 3-round swiss signature, $\T_3,$ that we have yet to define. It's worth attempting to construct $\T_3$ before reading on.

    The key insight is to realize that teams with the same record in vertically adjacent cells of the flowchart can actually play against each other without violating any of the swiss format requirements, merging the cells. Thus the flow chart for $\T_3$ looks like so.

    \fig{0.4}{t3_flow}{$\T_3$}

    We can use the flowchart to reconstruct the bracket and signature.

    \fig{1}{swisst3}{$\T_3$}

    $\T_3 = \bracksig{8;0;0;0} \to \bracksig{1} \to \bracksig{4;2;0}_2 \to \bracksig{2}_2 \to \bracksig{2;0} \to \bracksig{1}.$ $\T_3$ is also very similar to the format used in Figure \ref{fig:socon} by the 2023 Southern Conference Wrestling Championships: both use a primary eight-team balanced bracket and let their first-round losers fight their way back for a top-half finish.

    \theo{}{
        $\S_3$ and $\T_3$ are the only compact 3-round swiss signatures.
        }{
        Any compact $3$-round swiss signature must begin with $\bracksig{8;0;0;0} \to \bracksig{1}.$ Now let $\A$ be the semibracket that first-round primary brackets losers fall into. $\A$ must have two rounds, and the first-round primary bracket losers must all get no byes (otherwise they would not play the requisite three games). Thus $\A = \bracksig{4;a_1;0}_{(a_1/2+1)}$ for some $a_1$. As neither of the two semifinal winners can fall into $\A$, $a_1 \leq 2.$ Additionally, if $a_1 = 1$, $\A$ would not be a signature. Thus, $a_1 = 0$ or $2$.\\
        
         If $a_1 = 0$, then in between the first two brackets and $\A$, we must have two more brackets for the second-round losers of the primary bracket: $\bracksig{2;0}$ and $\bracksig{1}.$ Then $\A$ must be followed by $\bracksig{1}$ for the loser of its championship game, and then $\bracksig{2;0}$ and $\bracksig{1}$ so that the last two teams get a third game. This is the swiss signature $\S_3$.\\
        
         If $a_1 = 2$, then the losers of the two championship games of $\A$ have already played all three of their games and so need to fall into the bracket $\bracksig{2}$. Then we need $\bracksig{2;0}$ and $\bracksig{1}$ so that the last two teams get a third game. This is the swiss signature $\T_3$.
        }{}

A similar style of proof for other numbers of teams and rounds can be used to determine that there are no other signatures missing from Figure \ref{fig:swiss_names}.

Figure \ref{fig:swiss_names} tells us that there are five $8$-team swiss signatures. How would a tournament designer decide which 3-round signature to use? Well, it depends on what the prize structure of the format is. If the goal is to identify a top-three, then signature $\S_3$ is preferable: signature $\T_3$ doesn't even recognize a third-place, instead assigning fourth-place to two teams. But if the goal is to identify a top-four, signature $\T_3$ is preferable: the team that comes in fourth in signature $\S_3$ actually finishes with only one win, while the team that comes in fifth finishes with two. While it is still reasonable to grant the one-win team fourth-place -- they had a more difficult slate of opponents -- this is a somewhat messy situation that is solved by just using signature $\T_3.$

(McGarry and Schutz \cite{four_five_swap} considered outright swapping the positions of the fourth- and fifth-place teams at the conclusion of $\S_3$, but this format is not proper and provides some incentive for losing in the first round in order to get an easier path to a top-half finish. Simply using $\T_3$ when identifying the top-four teams is preferable.)

For similar reasons, both formats are good for selecting a top-one or top-seven, and $\S_3$ but not $\T_3$ is good for selecting a top-five. Finally, it might seem that $\S_3$ and $\T_3$ are good formats for selecting a top-two or top-six: in both cases, the top two and top six teams are clearly defined, and there are no teams with better records that don't make the cut. However, notice that if we use $\S_3$ or $\T_3$ to select a top-two, the final round of games are meaningless: the two teams that finish in the top-two are the two teams that win their first two games, irrespective of how the third round of games went. Better than using either format $\S_3$ or $\T_3$ would be to use the non-compact $2\x\S_2,$ shortening the format down to two rounds without losing any important games.

There are eight compact 4-round signatures: $\S_4$ and seven others. The count of compact $r$-round signatures for general $r$, however, is still an open question.

% draw the flowchart for the four round team ones

\begin{conj}{}{}
    Let $s_r$ be the number of compact $r$-round swiss signatures. Then $s_r$ is given by:
    \begin{align*}
        s_0 &= s_ 1 = 1\\
        s_r &= s_{r-1} \cdot \sum_{i=1}^{r-1}s_i
    \end{align*}
\end{conj}

Overall, swiss formats are very useful and practical tournament designs: they give each team the same number of games, they ensure that games are being played between teams that have the same record and thus, hopefully, similar skill levels, and, for many values of $m$, they efficiently identify a top-$m$ in a fair and satisfying way.

Further, swiss or near-swiss formats are great when the number of teams is exceedingly large. Even if not every requirement in Definition \ref{def:swiss} is met, or the number of teams isn't a power of two, or or the signature is not compact, or there is a round at the end that doesn't affect placement for important places, formats that are swiss in spirit tend to do a great job of gathering a lot of meaningful data about a large number of teams in a small number of rounds. For this reason, they are often used in large tournaments for board or cards games, such as chess or Magic: The Gathering. %cite
}