\sub{
    
    Many tournaments, particularly those in which there are many teams each looking to play a similar number of games against teams of similar skill, use a set of formats referred to as \i{swiss systems.} In particular, swiss systems or near-variants are commonly used in board game tournaments, such as chess or Magic: The Gathering.

    The idea behind a swiss system is to play a fixed number of rounds, and in each round have each matchup be between teams with the same record. This gives every team a bunch of games, while ensuring that teams are paired with teams that are probably similarly skilled. We can formally describe a swiss system in the language of multibrackets.

    \begin{definition}{Simple Multibracket}{}
        We say a multibracket is \i{simple} if all teams start in the primary semibracket.
    \end{definition}

    \begin{definition}{Complete Multibracket}{}
        We say a multibracket is \i{complete} if it ranks every team.
    \end{definition}

    \begin{definition}{Swiss System}{}
        A \i{swiss system} is a simple complete multibracket signature in which all teams play the same number of games and each matchup is between teams of the same record.
    \end{definition}

    One key family of swiss systems are the \i{standard systems}, $\S_r,$ which operate on $2^r$ teams over $r$ rounds. We will define them formally soon, but first, we display the first four standard system $\S_0, \S_1, \S_2,$ and $\S_3.$

     % label graphic
    \fig{0.8}{three_small_systems}{$\S_0, \S_1,$ and $\S_2$}

    The 1988 Men's College Basketball Maui Invitational used the format $\S_3.$

    \fig{0.7}{maui}{The 1998 Men's College Basketball Maui Invitational ($\S_3$)}

    The intuition behind the standard swiss systems is simple: each team plays a first-round game, and then the winners and losers each independently play out $\S_{r-1}.$ We can formalize this into a signature:

    \begin{definition}{Standard Swiss System $(\S_r)$}{standard_swiss}
        $\S_r,$ or the \i{standard swiss system on $2^r$ teams}, is the multibracket signature defined recursively by $$\S_0 = \bracksig{1},$$ and
        $$\S_r = \bracksig{2^r; ...; 0} \to \S_{0} \to \S_1 \to ... \to \S_{r-1}.$$
    \end{definition}
    
    So,
    
    \begin{figg}{Standard Swiss Systems}{std_swiss}
    \begin{center}
        \overfullhbox{
        \begin{tabular}{ c |  l }
            System & Signature\\
            \hline
            $\S_0$ & $\bracksig{1}$\\
            $\S_1$ & $\bracksig{2;0} \to \bracksig{1}$\\
            $\S_2$ & $\bracksig{4;0;0} \to \bracksig{1} \to \bracksig{2;0} \to \bracksig{1}$\\
            $\S_3$ & $\bracksig{8;0;0;0} \to \bracksig{1} \to \bracksig{2;0} \to \bracksig{1} \to \bracksig{4;0;0} \to \bracksig{1} \to \bracksig{2;0} \to \bracksig{1}$\\
        \end{tabular}
        }
    \end{center}
    \end{figg} 

    Figure \ref{fig:std_swiss} matches with Figures \ref{fig:three_small_systems} and \ref{fig:maui}.

    One thing to note about the standard swiss systems is that there is no chance of a rematch.

    \theo{}{
        Standard swiss systems will never play a rematch.
    }{
        We show this by induction on $r$. No games, and thus no rematches, are played in $\S_0.$ For any other $r$, after the first round, teams are separated into two instances of $\S_{r-1}$. No team in the same instance of $\S_{r-1}$ played each other in the first round, and, by induction, no rematches occur within either instance of $\S_{r-1}$. Thus, $\S_r$ never plays a rematch.
    }{swiss_no_rematches}

    Theorem \ref{th:swiss_no_rematches} is made a bit weaker but more general in Theorem \ref{th:swiss_most_no_rematches}.

    \theo{}{
        Every swiss system has a multibracket instantiation that will never play a rematch.
    }{
        %proof_needed
    }{swiss_most_no_rematches}

    What do some of the non-standard swiss systems look like? Well for one thing, we can run multiple instances of the same standard system at the same time to form a bigger system. Figure \ref{fig:8242}, for example, displays the system created by running two copies of $\S_2$ simultaneously.

    \fig{1}{8242}{$\bracksig{8;0;0} \to \bracksig{2} \to \bracksig{4;0} \to \bracksig{2}$}

    The signature of running $m$ instances of a particular multibracket at the same time is just the signature of the original bracket, with every number multiplied by $m$. Because of this, the swiss system in Figure \ref{fig:8242} is sometimes referred to as 2x$\S_2.$

    \begin{definition}{$m\x\A$}{mbya}
        If $m \in \N$ and $\A$ is a multibracket signature, than $m\x\A$ is the multibracket signature formed by multiplying every number in every signature in $\A$ by $m$.
    \end{definition}

    Swiss systems formed by the construction in \ref{def:mbya} are somewhat less interesting to study than other systems, so we introduce the notion of a \i{compact} system.

    \begin{definition}{Compact Swiss System}{}
        We say a swiss system is \i{compact} if its primary semibracket is a bracket.
    \end{definition}

    Thus, for all $r$, $\S_r$ is compact, but $m$x$\S_r$ is not for $m > 1.$ We can return now to the question of what a compact non-standard swiss system would look like.

    As it happens, $\S_0$, $\S_1$, and $\S_2$ are the only compact 0-, 1-, and 2-round systems, respectively.

    \theo{}{
        There is only one compact $r$-round system from $r \leq 2.$
}{
    Any 0-round system must be the unique 1-team multibracket $\S_0.$ Likewise, any 1-round system must be on two teams, and simple and complete, making it equal $\S_1$. Finally, any compact 2-round system will have a primary semibracket of signature $\bracksig{4;0;0}$. The loser of the primary bracket has no more games to play, so the secondary bracket must be $\bracksig{1}$. After that, there are two teams each with one game left, so they play each other, completing the final two brackets $\bracksig{2;0} \to \bracksig{1},$ which comes out to just $\S_2.$
}{}

    $\S_3$ is not the unique compact $3$-round system, however. The other system, sometimes referred to as $\T_3$, is displayed in Figure \ref{fig:swisst3}. (Note that there are instantiations of $\T_3$ that risk playing rematches. As per Theorem \ref{th:swiss_most_no_rematches}, we show the instantiation that does not).

    \fig{1}{swisst3}{$\T_3 = \bracksig{8;0;0;0} \to \bracksig{1} \to \bracksig{4;2;0}_2 \to \bracksig{2}_2 \to \bracksig{2;0} \to \bracksig{1}$}

    \theo{}{
    $\S_3$ and $\T_3$ are the only compact 3-round swiss systems.
}{
    Any compact $3$-round swiss system must have primary bracket $\bracksig{8;0;0;0}$ and secondary bracket $\bracksig{1}.$ Now let $\A$ be the semibracket that first-round primary brackets losers fall into. $\A$ must have two rounds, and the first-round primary bracket losers must all get no byes (otherwise they would not play the requisite three games). Thus $\A = \bracksig{4;a_1;0}_{(a_1/2+1)}$ for some $a_1$. As the primary and secondary bracket each have winners, $a_1 \leq 2.$ Additionally, if $a_1 = 1$, $\A$ would not be a signature. Thus, $a_1 = 0$ or $2$.\\
    
     If $a_1 = 0$, then in between the first two brackets and $\A$, we must have two more brackets for the second-round losers of the primary bracket: $\bracksig{2;0}$ and $\bracksig{1}.$ Then $\A$ must be followed by $\bracksig{1}$ for the loser of its championship game, and then $\bracksig{2;0}$ and $\bracksig{1}$ so that the last two teams get a third game. This is the swiss system $\S_3$.\\
    
     If $a_1 = 2$, then the losers of the two championship games of $\A$ have already played all three of their games and so need to fall into the bracket $\bracksig{2}$. Then we need $\bracksig{2;0}$ and $\bracksig{1}$ so that the last two teams get a third game. This is the swiss system $\T_3$.
}{}
    How would a tournament designer decide which compact 3-round system to use? Well, it depends on what the prize structure of the format is. If the goal is to identify a top-three, then system $\S_3$ is preferable: system $\T_3$ doesn't even recognize a third-place, instead assigning fourth-place to two teams. But if the goal is to identify a top-four, system $\T_3$ is preferable: the team that comes in fourth in system $\S_3$ actually finishes with only one win, while the team the comes in fifth finishes with two. While it is still reasonable to grant the one-win team fourth-place -- they had a more difficult slate of opponents -- this is a somewhat messy situation that is solved by just using system $\T_3.$

    (McGarry and Schutz \cite{four_five_swap} considered outright swapping the positions of the fourth- and fifth-place teams at the conclusion of $\S_3$, but this provides some incentive for losing in the first round in order to get an easier path to a top-half finish. Simply using $\T_3$ when identifying the top-four teams is preferable.)

    For similar reasons, both formats are good for selecting a top-one or top-seven, and $\S_3$ but not $\T_3$ is good for selecting a top-five. Finally, it might seem that $\S_3$ and $\T_3$ are good formats for selecting a top-two or top-six: in both cases the top two and top six teams are clearly defined, and there are no teams with better records that don't make the cut. However, notice that if we use $\S_3$ or $\T_3$ to select a top-two, the final round of games are meaningless: the two teams that finish in the top-two are the two teams that win their first two games, irrespective or how game three went. Better than using either format $\S_3$ or $\T_3$ would be to use the non-compact $2\x\S_2,$ shortening the format down to two rounds without losing any important games.

    There are eight compact 4-round systems: $\S_4$ and seven others. The count of compact $r$-round systems for general $r$, however, is still an open question.

    \begin{conj}{}{}
        Let $s_r$ be the number of compact $r$-round swiss systems. Then $s_r$ is given by:
        \begin{align*}
            s_0 &= s_ 1 = 1\\
            s_r &= s_{r-1} \cdot \sum_{i=1}^{r-1}s_i
        \end{align*}
    \end{conj}

    Overall, swiss systems very useful and practical tournament designs: they give each team the same number of games, they ensure that games are being played between teams that have the same record and thus, hopefully, similar skill levels, and, for many values of $m$, they efficiently identify a top-$m$ in a fair and satisfying way.

    Further, swiss or near-swiss systems are great when the number of teams is exceedingly large. Even if the number of teams is not a power of two, or the system is not compact, or there is a round at the end that doesn't affect placement, formats that are swiss in spirit tend to do a great job of gathering a lot of meaningful data about a large number of teams in a small number of rounds.
}