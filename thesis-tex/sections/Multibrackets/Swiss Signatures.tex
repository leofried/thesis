\sub {

    \phig{maui_pr}{TEST}

    Linear multibracket signatures that admit round-respectful multibrackets can be drawn as a flowchart: each cell corresponds to a specific round of a specific semibracket (or a set of teams on bye waiting to play their first game), and winners follow the blue lines, while loses follow the red lines. Consider, for example, the 2023 Major League Quadball Championship Play-In Tournament \cite{wiki_mlq}, which used a linear multibracket of signature $\bracksig{4;2;0;0} \to \bracksig{4;0;1;0}.$ %officially define flowchart

    % \fig{0.5}{mlq}{2023 Major League Quadball Championship Play-In Tournament}

    The flowchart for this signature looks like so. (The top two rows represent the primary bracket, while the bottom two rows represent the consolation bracket.)

    \fig{0.4}{mlq_flow}{$\bracksig{4;2;0;0} \to \bracksig{4;0;1;0}$ Flowchart}
    
    Note that a flowchart corresponds to a signature, not to a format. For example, the way that the four teams in the first round of the secondary format are matched up with each other is not specified by Figure \ref{fig:mlq_flow}: only the signature is. Additionally, flowcharts can only be drawn for signatures that admit a round-respectful formats, as they require that teams that lose in the same cell (round of a semibracket) fall into the same cell (round of a semibracket). That said, flowcharts are a really nice way to visualize different round-respectful linear multibracket signatures in way that may be more intuitive than drawing out the brackets directly.

    As another example, consider the flowchart from 1988 Men's College Basketball Maui Invitational \cite{wiki_maui} from Figure \ref{fig:maui_pr} on page \pageref{fig:maui_pr}, which used a proper linear multibracket of signature $\bracksig{8;0;0;0} \to \bracksig{1} \to \bracksig{2;0} \to \bracksig{1} \to \bracksig{4;0;0} \to \bracksig{1} \to \bracksig{2;0} \to \bracksig{1}.$

    \fig{0.4}{s3_flow}{$\bracksig{8;0;0;0} \to \bracksig{1} \to \bracksig{2;0} \to \bracksig{1} \to\\ \bracksig{4;0;0}\;\;\;\, \to \bracksig{1} \to \bracksig{2;0} \to \bracksig{1}$ Flowchart}

    The signature used in the Maui Invitational has several properties that make the flowchart in Figure \ref{fig:s3_flow} look nicer than the flowchart in Figure \ref{fig:mlq_flow}.

    \begin{itemize}
        \item Every team starts in the same cell.
        \item Games are always between teams of the same record, so we can ambiguously assign a record to each cell.
        \item Every team plays the same number of games, so our flowchart is nicely divided into columns, with each team playing one game in each column.
        \item Every team wins a semibracket and so every team ends in a cell. Compare this to $\bracksig{4;2;0;0} \to \bracksig{4;0;1;0}$ which only ranks two teams, leaving the other four teams without a cell.
    \end{itemize}

    There is one other niceness property that both flowcharts have, which is that every cell with no out-arrows has only one in-arrow. This property is perhaps less intuitively important than the previous four, but the point is this: if there existed some cell with no-out arrows but more than one in-arrow, then it could be split into multiple cells, one for each in-arrow, with not meaningful change to the format (as the teams in cells with no out-arrows play no games anyways). Thus to avoid double-counting these nice signatures, we prohibit ones that don't have this property. (This is equivalent to having no semibrackets that are neither trivial nor competitive.)

    Signatures with all of these properties are called \i{Swiss signatures}, named because of their first recorded use at a chess tournament in Zürich, Switzerland in 1895 \cite{info_swiss}.

    \begin{definition}{Swiss Signatures}{Swiss}
        A \i{Swiss signature} is a round-respectful linear multibracket signature with the following additional properties:
        \begin{enumerate}[(a)]
            \item Every team starts in the primary semibracket.
            \item Every game is between two teams with the same record.
            \item Every team plays the same number of games.
            \item Every team wins a semibracket.
            \item Every semibracket is either trivial or competitive.
        \end{enumerate}
    \end{definition}

    \begin{definition}{$r$-Round Swiss}{}
        We say a Swiss signature in which each team plays $r$ games is an $r$-round Swiss signature.
    \end{definition}
    
    Thus $\bracksig{8;0;0;0} \to \bracksig{1} \to \bracksig{2;0} \to \bracksig{1} \to \bracksig{4;0;0} \to \bracksig{1} \to \bracksig{2;0} \to \bracksig{1}$ is a $3$-round Swiss signature. In fact, it is actually an example of a particular family of Swiss signature known as the the \i{standard Swiss signatures}, which we abbreviate by $\S_r$ for some $r$. We present the definition of a standard Swiss signature below, but in terms of flowcharts, the standard Swiss signatures are just binary trees where each team ends in their own cell.

    \begin{definition}{Standard Swiss Signature $(\S_r)$}{standard_Swiss}
        $\S_r,$ or the \i{standard $r$-round Swiss signature}, is the multibracket signature defined recursively by $$\S_0 = \bracksig{1},$$ and
        $$\S_r = \bracksig{2^r; ...; 0} \to \S_{0} \to \S_1 \to ... \to \S_i \to ... \to \S_{r-1}.$$
    \end{definition}

    Thus we have
    \begin{align*}
        \S_0 =\; &\bracksig{1}\\
        \S_1 =\; &\bracksig{2;0} \to \bracksig{1}\\
        \S_2 =\; &\bracksig{4;0;0} \to \bracksig{1} \to \bracksig{2;0} \to \bracksig{1}\\
        \S_3 =\; &\bracksig{8;0;0;0} \to \bracksig{1} \to \bracksig{2;0} \to \bracksig{1} \to \bracksig{4;0;0} \to \bracksig{1} \to \bracksig{2;0} \to \bracksig{1}
    \end{align*}

    Figure \ref{fig:three_small_flow} display the flowcharts $\S_0, \S_1,$ and $\S_2$, while the Maui Invitational used the standard Swiss signature $\S_3.$

    % \fig{1}{three_small_systems}{$\S_0, \S_1,$ and $\S_2$}
    \fig{0.4}{three_small_flow}{$\S_0, \S_1,$ and $\S_2$}

    The standard Swiss signatures are particularly nice: in addition to the other Swiss signature requirements, the primary semibracket of a standard Swiss signature has rank one, so a single champion is crowned. Not every Swiss signature has this property: consider, for example, the following 8-team Swiss signature.

    % \fig{1}{two_by_two}{$\bracksig{8;0;0}_2 \to \bracksig{2}_2 \to \bracksig{4;0}_2 \to \bracksig{2}_2$}
    
    \fig{0.4}{2xs2_flow}{$\bracksig{8;0;0}_2 \to \bracksig{2}_2 \to \bracksig{4;0}_2 \to \bracksig{2}_2$}

    The flowchart in Figure \ref{fig:2xs2_flow} corresponds to a Swiss signature, but it doesn't crown an individual champion, as two teams end the format undefeated. The Swiss signature $\bracksig{8;0;0}_2 \to \bracksig{2}_2 \to \bracksig{4;0}_2 \to \bracksig{2}_2$ is not \i{compact}.

    \begin{definition}{Compact}{}
            We say a Swiss signature is \i{compact} if only one teams ends undefeated.
    \end{definition}

    But non-compact Swiss signatures don't just allow multiple teams to end undefeated, they are actually just a compact Swiss signature being run multiple times in parallel.

    \theo{}{
        Let $\A$ be a non-compact Swiss signature where $m > 1$ teams end undefeated. Then $m$ will divide the number of teams in every cell in the flowchart of $\A$.
    }{

    }{compactless}

    For example, Figure \ref{fig:2xs2_flow} is just $\S_2$ but with the number of teams in each cell multiplied by two. We use this to introduce $m\x$ notation.

   \begin{definition}{$m\x\A$}{mx_notation}
        If $m \in \N$ and $\A$ is a multibracket signature, then $m\x\A$ is the multibracket signature formed by multiplying every number in every signature in $\A$ by $m$.
    \end{definition}

    So $\bracksig{8;0;0}_2 \to \bracksig{2}_2 \to \bracksig{4;0}_2 \to \bracksig{2}_2 = 2\x\S_2.$

    With the standard Swiss signature and $m\x$ notation defined, we are ready for Figure \ref{fig:Swiss_names}, which details the various Swiss signatures for 1-, 2-, 4-, and 8-teams.

    \begin{figg}{The 1-, 2-, 4-, and 8-team Swiss Signatures}{Swiss_names}
        \begin{center}
            \begin{tabular}{ c | c | c | c | c}
                & 1 Team & 2 Teams & 4 Teams & 8 Teams\\
                \hline
                0 Rounds & $\S_0$ & $2\x\S_0$ & $4\x\S_0$ & $8\x\S_0$\\
                \hline
                1 Round & & $\S_1$ & $2\x\S_1$ & $4\x\S_1$\\
                \hline
                2 Rounds & & & $\S_2$ & $2\x\S_2$\\
                \hline
                \multirow{1}{*}{3 Rounds} & & & &  $\S_3, \T_3$ \\
            \end{tabular}
        \end{center}
        \end{figg}

    How do we know that this diagram is complete? Well the cells below the diagonal must be empty, as after $r$ rounds, only one out of $2^r$ teams can remain undefeated, and if we tried to play an additional round, they would have no teams left to play. The cells above the diagonal are non-compact, and thus complete by Theorem \ref{th:compactless}. Finally, the most interesting cells are on the diagonal, as those are the compact signatures.

    \theo{}{
        The only compact 0-, 1-, and 2-round Swiss signatures are the standard ones.
    }{

    }{}

    Figure \ref{fig:Swiss_names} tells us that there are, two compact 3-round Swiss signatures: the standard $\S_3$, and the yet to be defined $\T_3.$ It's worth attempting to construct $\T_3$ before reading on.

    The key insight is to realize that teams with the same record in vertically adjacent cells of the flowchart can actually play against each other without violating any of the Swiss format requirements, merging the cells. Thus the flow chart for $\T_3$ looks like so.

    \fig{0.4}{t3_flow}{$\T_3$}

    We can use the flowchart to reconstruct the bracket and signature.

    \fig{1}{Swisst3}{$\T_3 = \bracksig{8;0;0;0} \to \bracksig{1} \to \bracksig{4;2;0}_2 \to \bracksig{2}_2 \to \bracksig{2;0} \to \bracksig{1}.$}

    $\T_3$ is also very similar to the format used by the 2023 Southern Conference Wrestling Championships in Figure \ref{fig:socon} on page \pageref{fig:socon}: both use a primary eight-team balanced bracket and let their first-round losers fight their way back for a top-half finish.

    \theo{}{
        $\S_3$ and $\T_3$ are the only compact 3-round Swiss signatures.
        }{
        Any compact $3$-round Swiss signature must begin with $\bracksig{8;0;0;0} \to \bracksig{1}.$ Now let $\A$ be the semibracket that first-round primary brackets losers fall into. $\A$ must have two rounds, and the first-round primary bracket losers must all get no byes (otherwise they would not play the requisite three games). Thus $\A = \bracksig{4;a_1;0}_{(a_1/2+1)}$ for some $a_1$. As neither of the two semifinal winners can fall into $\A$, $a_1 \leq 2.$ Additionally, if $a_1 = 1$, $\A$ would not be a signature. Thus, $a_1 = 0$ or $2$.\\
        
         If $a_1 = 0$, then in between the first two brackets and $\A$, we must have two more brackets for the second-round losers of the primary bracket: $\bracksig{2;0}$ and $\bracksig{1}.$ Then $\A$ must be followed by $\bracksig{1}$ for the loser of its championship game, and then $\bracksig{2;0}$ and $\bracksig{1}$ so that the last two teams get a third game. This is the Swiss signature $\S_3$.\\
        
         If $a_1 = 2$, then the losers of the two championship games of $\A$ have already played all three of their games and so need to fall into the bracket $\bracksig{2}$. Then we need $\bracksig{2;0}$ and $\bracksig{1}$ so that the last two teams get a third game. This is the Swiss signature $\T_3$.
        }{}

    Figure \ref{fig:Swiss_names} tells us that there are five $8$-team Swiss signatures. How would a tournament designer decide which 3-round signature to use? Well, it depends on what the prize structure of the format is. If the goal is to identify a top-three, then  $\S_3$ is preferable: $\T_3$ doesn't even recognize a third-place, instead assigning fourth-place to two teams. But if the goal is to identify a top-four, $\T_3$ is preferable: the team that comes in fourth in $\S_3$ actually finishes with only one win, while the team that comes in fifth finishes with two. While it is still reasonable to grant the one-win team fourth-place -- they had a more difficult slate of opponents -- this is a somewhat messy situation that is solved by just using s$\T_3.$

    (McGarry and Schutz \cite{four_five_swap} considered outright swapping the positions of the fourth- and fifth-place teams at the conclusion of $\S_3$, but this format is not proper and provides some incentive for losing in the first round in order to get an easier path to a top-half finish. Simply using $\T_3$ when identifying the top-four teams is preferable.)

    For similar reasons, both formats are good for selecting a top-one or top-seven, and $\S_3$ but not $\T_3$ is good for selecting a top-five. Finally, it might seem that $\S_3$ and $\T_3$ are good formats for selecting a top-two or top-six: in both cases, the top two and top six teams are clearly defined, and there are no teams with better records that don't make the cut. However, notice that if we use $\S_3$ or $\T_3$ to select a top-two, the final round of games are meaningless: the two teams that finish in the top-two are the two teams that win their first two games, irrespective of how the third round of games went. Better than using either $\S_3$ or $\T_3$ would be to use the non-compact $2\x\S_2,$ shortening the format down to two rounds without losing any important games.

    We now count the number of $r$-round signatures.

    \theo{}{
        Let $s_r$ be the number of compact $r$-round Swiss signatures. Then,
        \begin{align*}
            s_0 &= s_ 1 = 1\\
            s_r &= s_{r-1} \cdot \sum_{i=1}^{r-1}s_i
        \end{align*}
    }{

    }{}

    The eight compact 4-round signatures are displayed below.

    \fig{0.13}{4rounds}{The Eight Compact 4-round Swiss Signatures}
    
    Overall, Swiss formats are very useful and practical tournament designs: they give each team the same number of games, they ensure that games are being played between teams that have the same record and thus, hopefully, similar skill levels, and, for many values of $m$, they efficiently identify a top-$m$ in a fair and satisfying way.

    Further, Swiss or near-Swiss formats are great when the number of teams is exceedingly large. Even if not every requirement in Definition \ref{def:Swiss} is met, or the number of teams isn't a power of two, or the signature is not compact, or there is a round at the end that doesn't affect placement for important places, formats that are Swiss in spirit tend to do a great job of gathering a lot of meaningful data about a large number of teams in a small number of rounds. For this reason, they are often used in large tournaments for board or cards games, such as chess or Magic: The Gathering \cite{ifno_mtg}.
}