\sub {
    Many tournaments, particularly those in which there are many contestants each looking to play a similar number of games against players of similar skill, use a set of formats referred to as \textit{swiss} formats. In particular, \textit{swiss} systems or near-variants are commonly used in board game tournaments, such as chess or Magic: The Gathering.

    \begin{definition}{Swiss}{}
        A \textit{swiss} format is a perfect multibracket in which
        \begin{enumerate}
            \item All teams start in the primary semibracket,
            \item Each matchup is between teams of the same record,
            \item All teams play the same number of games,
        \end{enumerate}
    \end{definition}

    \begin{definition}{$r$-Round Swiss}{}
        We say a swiss format is an $r$-\textit{round swiss} if each team plays $r$ games.
    \end{definition}

    \theo{}{
        All $r$-round swiss formats are on $m \cdot 2^r$ teams for some $m$.
    }{
        In order for a multibracket to be swiss, its primary semibracket must be balanced. (Otherwise, teams that get byes will play fewer games than teams that don't.) Additionally, since each winner of the primary semibracket will play all of their games in that bracket, it must be exactly $r$ rounds long. A balanced semibracket that is $r$ rounds long has signature $\bracksig{m \cdot 2^r; 0; ...; 0}$ for some $m.$ Thus, since every team starts in the primary semibracket, there must be $m \cdot 2^r$ teams participating for some $m$.
    }{}



    % definition: swiss
    % theorem: no rematches
    % theorem: m * 2^r teams
    % definition: compact? swiss (primary bracket has order 1)
    % theorem: binomial distribution of teams
    % examples: 1, 2, 4
    % example: 8 --> depends on top how many
    % reply rounds





    %theorem -- all swiss formats are on m*2^n teams and play k < n games and all swiss primary brackets are balanced
    %Note that the m is mostly irrelevant, so we focus on the n [[not just the n, really the k]] -> the notion of "complete swiss."
    %1 team, 2 team, 4 team complete swiss' are unique
    %8 team complete swiss has two options -- depends on top how many. This generalizes.
    %What to do about top 2 in three games? bonus game could be a replay round
    %theorem -- all matchups in a swiss format are bt teams of the same record.
    %well....
    % 16  0   0   0   0
    %                 1
    %             2   0
    %                 1
    %         4   0   0
    %                 1
    %     8   0   2   0
    %                 2
    %             2   0
    %                 1
    %         4   0   0
    %                 1
    %             2   0
    %                 1

}