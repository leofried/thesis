\sub {
 
    \fig{0}{maui_pr}{}

    \begin{definition}{Flowchart}{}
        The \i{flowchart} of a linear multibracket that consists of $k$ semibrackets is a directed graph in which the nodes are arranged into rows, where
        \begin{enumerate}[(a)]
            \item There is a node for each team, each round of each semibracket, each place a team could finish in, one additional node representing elimination.
            \item The zeroth row has the nodes representing each team, arranged from lowest seed to highest seed.
            \item The $i$th row for $1 \leq i \leq k$ has the nodes representing the rounds of the $i$th semibracket, arranged in order, plus the node representing the place a team gets for winning the $i$th semibracket.
            \item The final row row has only the node representing elimination.
            \item There is an arrow from each team node to the node with that teams starting line.
            \item For each round $\bracklabel{R},$ there is a arrow from the node representing $\bracklabel{R}$ to the node(s) representing the places that $\bracklabel{R}$-round losers go.
        \end{enumerate}
    \end{definition}

    For example, consider the the 2023 Major League Quadball Championship Play-In Tournament \cite{wiki_mlq}, which used a linear multibracket of signature $\bracksig{4;2;0;0} \to \bracksig{4;0;1;0}.$

    \fig{0.5}{mlq}{2023 MLQ Championship Play-In Tournament}

    The flowchart for this format is displayed below.

    \fig{0.5}{mlq_flow1}{2023 MLQ Championship Play-In Tournament Flowchart}

    Now, the flowchart in Figure \ref{fig:mlq_flow1} has a few nice properties, and its worth investigating which of those are guaranteed by the network condition or linearity, and which just happen to be true about this particular linear multibracket.

    First of all, the network condition is what makes a flowchart even reasonable in the first place: in non-networked format, more data than just which game a team most recently played in what the result of the game was are needed to determine a teams next game.

    Secondly, all the arrows are pointing down. This is a result of linearity: A non-linear multibracket could have arrows point up, or even from a given multibracket to itself, making a flowchart much more complex. Note that we don't even define a flowchart for nonlinear multibrackets as the niceness of the rows would be lost.
    
    Thirdly, every non-terminal node (that is, every node except the placement and eliminations ones) has exactly one arrow coming from it. 




}