\sub{

    In this chapter, we will develop the theory of a new kind of format called a \textit{multibracket}. The multibracket is a very versatile class of formats, unifying many seemingly different formats that are used in practice and allowing us to study them all more effectively.
    
    Amongst these formats are:
    \begin{enumerate}
        \item Third-place games, as in the 2015 AFL Asian Cup.
        \item Assigning bids for a future tournament, as in the Ultimate Frisbee Sectional and Regional Tournaments.
        \item Swiss systems, as in the 2023 League of Legends World Championships.
        \item Double elimination tournaments, as in the 2005 Women's College World Series.
    \end{enumerate}

    But before we can develop the concept of a multibracket or see how any of these tournaments are instances of it, we need to first investigate a class of formats called \textit{semibrackets} of which multibrackets are composed.

    Consider the following tournament design problem: we are tasked with designing an eight-team tournament to select the top two teams who will go on to compete as a part of the national tournament. The catch: here's only enough time for two rounds: perhaps due to field space or team fatigue, each team can only play two games. What design should we use?

    The most natural answer to this question is to simply use a traditional eight-team bracket, but to leave the championship game unplayed. This format is displayed in the figure below.

    \fig{1}{800}{$\bracksig{8;0;0;0}$ with no Championship Game}

    The format in Figure \ref{fig:800} does exactly what we need. The championship game being left unplayed is not a bug but a feature: each team plays a maximum of two games, and the two teams that advance to the national tournament are clear.

    While it would be reasonable to describe the format in Figure \ref{fig:800} as two brackets that run side-by-side, it would be nice to be able to describe it as a single format: a bracket in which the championship game is left unplayed.

    \begin{definition}{Semibracket}{semibracket}
        A \textit{semibracket} is a tournament format in which:
        \begin{itemize}
            \item Teams don't play any games after their first loss.
            \item The matchups between teams are determined based on the ordering of the teams in $\T$ in advance of the outcomes of any games.
        \end{itemize}
        All teams that finish a semibracket with no losses are declared co-champions.
    \end{definition}

    Recall that a bracket is a tournament format in which:
    \begin{itemize}
        \item Teams don't play any games after their first loss.
        \item Games are played until one team has no losses, and that team is crowned champion.
        \item The matchups between teams are determined based on the ordering of the teams in $\T$ in advance of the outcomes of any games.
    \end{itemize}

    Semibrackets are formats that adhere to the first and third requirement, but not (necessarily) the second one. Multiple teams can finish a semibracket undefeated, and they are each winners of the semibracket.

    Figure \ref{fig:types_of_brackets} describes which properties various bracket-like formats require. The row for multibrackets is included even though we won't formally define a multibracket until the next section.

    \begin{figg}{Properties of Bracket-like Formats}{types_of_brackets}
        \begin{center}
            \overfullhbox{
            \begin{tabular}{| c | c | c | c |}
                \hline
                & No games & Only one team & Matchups determined\\
                Format & after first loss & finishes undefeated & in advance  \\
                \hline
                Traditional Bracket & \Huge{\check} & \Huge{\check} & \Huge{\check}\\
                \hline
                Semibracket & \Huge{\check} & \Huge{\ex} & \Huge{\check}\\
                \hline
                Multibracket & \Huge{\ex} & \Huge{\ex} & \Huge{\check}\\
                \hline
                Reseeded Bracket & \Huge{\check} & \Huge{\check} & \Huge{\ex}\\
                \hline
            \end{tabular}
            }
        \end{center}
    \end{figg} 

    Definition \ref{def:semibracket} implies that traditional brackets are a subset of semibrackets. Of course, not all semibrackets are traditional brackets: the format in Figure \ref{fig:800} is one such example.

    That said, this format is not a particularly exciting example of a semibracket: after all, it is just a traditional bracket minus one game. Are there any examples of semibrackets that are not traditional brackets with some rounds left uncompleted?

    Indeed there are. Let's modify the original problem so that we need to pick a top three teams out of twelve. Again, no team can play more then two games. The natural choice is shown below in Figure \ref{fig:1200}.
    
    \fig{1}{1200}{A More Exciting Semibracket}

    There is no potential for the format in Figure \ref{fig:1200} to be completed into a traditional bracket, the next round would include three teams: and odd number. But as a semibracket, this is still a viable format, one that perfectly solves the tournament design problem that we were given.
    
    \begin{definition}{Rank of a Semibracket}{}
        If semibracket $\A$ has $m$ co-champions, then $\rank{\A} = m$. We say $\A$ has rank $m$ or that $\A$ \textit{ranks m teams}.
    \end{definition}

    Thus, traditional brackets are exactly the semibrackets that rank one team. The formats in Figures \ref{fig:800} and \ref{fig:1200} rank two and three teams, respectively.

    We can adapt the concept of a bracket signature to semibrackets.

    \begin{definition}{Semibracket Signature}{}
        The \textit{signature} $\bracksig{a_0; ...; a_r}_m$ of an $r$-round semibracket $\A$ is list such that $a_i$ is the number of teams with $i$ byes and $m = \rank{\A}.$ (In the case where $m = \rank{\A} = 1$, it can be omitted.)
    \end{definition}

    Thus the signature of traditional brackets are the same as when they are viewed as semibrackets that rank one team. The signatures of the formats in Figures \ref{fig:800} and \ref{fig:1200} are $\bracksig{8;0;0}_2$ and $\bracksig{12;0;0}_3,$ respectively.

    In analogy with traditional bracket signature's Theorem \ref{th:signature_sum}, we have

    \begin{theorem}{}{}
        Let $\A = \bracksig{a_0; ...; a_r}_m$ be a list of natural numbers. Then $\A$ is a semibracket signature if and only if $$\sum_{i=0}^r a_i \cdot \left(\frac{1}{2}\right)^{r - i} = m.$$
    \end{theorem}

    The proof is almost identical to that of Theorem \ref{th:signature_sum} so we leave it out for brevity. Likewise, the fundamental theorem still applies, again with almost the exact same proof (also left out for brevity).

    \begin{theorem}{}{}
        There is exactly one proper semibracket with each semibracket signature.
    \end{theorem}

    Finally, the space of semibrackets contains some slightly bizarre formats.

    \begin{definition}{Trivial Semibrackest}{}
        We say semibrackets is \textit{trivial} if it has signature $\bracksig{n}_n$ for some $n$.
    \end{definition}

    No games are played in a trivial semibracket: all teams that enter one exit having been declared champion. The only trivial traditional bracket is $\bracksig{1}$, but there is one trivial semibracket with each rank. Trivial semibrackets look a bit strange when drawn.

    \fig{1}{trivial}{$\bracksig{6}_6$}

    Arguably stranger than trivial semibrackets are \textit{semitrivial} semibrackets.

    \begin{definition}{Semitrivial Semibracket}{}
        We say a semibracket $\A = \bracksig{a_0; ...; a_r}_m$ is \textit{semitrivial} if $r \geq 1$ and $a_r \neq 0.$
    \end{definition}

    In a semitrivial semibracket, some teams are declared champion without playing any games, while other have games on their schedule. The simplest example of a semitrivial semibracket is $\bracksig{2;1}_2.$

    \fig{1}{21}{$\bracksig{2;1}_2$}

    There are no semitrivial traditional brackets: if a team wins a traditional bracket without playing any games, they must be the only team in the bracket. Semitrivial semibrackets are pretty unintuitive: luckily, we will soon see that semitrivial semibracket aren't required to develop the theory of multibrackets, and so we won't have to worry about them.

    With the idea of a semibracket developed and fleshed out, we can now move on to the meat of the chapter: multibrackets.
}