\sub{

    In this section, we present an algorithm for verifying whether a bracket signature strongly respects a tiered seeding. We will first intuitively describe what the algorithm is doing and then we will formally specify it, before running the algorithm on a few examples and then finally proving its correctness. 
    
    The idea behind the algorithm is to ensure that in each round, teams of the same tier are being assigned opponents of the same tier. This is done by keeping track of the tiers of the teams that will be playing in each round, and ensuring that the round-specific tiered signatures are palandromic. Formally,

    \begin{definition}{The Palandromic Algorithm for Tiered Signatures}{}
        Let $\A$ be a bracket signature and $\B$ be a tiered seeding. First, check if $\A$ weakly respects $\B$. If it doesn't, then it certainly doesn't strongly respect it. If it does, then let $\C$ be the tiered signature of $(\A, \B).$\\

        We define $\F$, a recursive operator that maps a tiered signature to either $\true$ or $\false$. Then, if $\F(\C)$ is true, $\A$ strongly respects $\B$, otherwise it does not.\\

        The operator $\F$ is defined in the following way on $\C = \bracksig{\C_{0}; ...; \C_{r}}.$
        \begin{itemize}
            \item If $r = 0$, then $\F(\C)$ is $\true.$
            \item Otherwise, if $\C_0$ is not palandromic, then $\F(\C)$ is $\false.$
            \item Otherwise, let $\D_0$ be the right half of $\C_0$ concatenated with $\C_1$, and $\D = \F(\bracksig{\D_0; \C_2; ...; \C_{r}}).$ Then, $\F(\C) = \F(\D).$
        \end{itemize} 

        (For the last step, if $\C_0$ has odd length, then the first element of $\D_0$ is half of the middle element of $\C_0.$ The middle element of $\C_0$ will always be even because it is palandromic and its sum must be even.)
    \end{definition}

    Let's go over a few examples. Consider the bracket signature $\A = \bracksig{8;6;3;0;0;0}$ along with the tiered seeding $\B = \bracktier{4,4,4,2,2,1}.$ As we verified earlier, $\A$ weakly respects $\B$, so we can apply the palandromic algorithm to check if it is strongly respectful.
    \begin{align*}
        \F(\C) &= \F(\bracksig{\bracktier{4, 4};\bracktier{4,2};\bracktier{2,1};\bracktier{};\bracktier{};\bracktier{}})\\
        &= \F(\bracksig{\bracktier{4, 4,2};\bracktier{2,1};\bracktier{};\bracktier{};\bracktier{}})\\
        &= \false \textrm{ (because $\bracktier{4, 4, 2}$ is not palandromic.)}
    \end{align*}

    \fig{0.6}{863000 tiered}{$\bracksig{\bracktier{4, 4};\bracktier{4,2};\bracktier{2,1};\bracktier{};\bracktier{};\bracktier{}}$}

    We can verify this result intuitively with the help of the bracket $\A$. In the second round, for example, two of the Tier 4 teams play each other, while two of them play the winner of a Tier 5 vs Tier 6 matchup. If the Tier 5 and 6 teams are much worse than the rest of the teams, it is not hard to imagine that the two Tier 4 teams who have to play each other are at a severe disadvantage.

    Let's instead consider the bracket signature $\A = \bracksig{8;0;6;0;0;1;0}$ along with the tiered seeding $\B = \bracktier{8,4,2,1}.$ $\A$ weakly respects $\B$ with tiered signature $$\C = \bracksig{\bracktier{8};\bracktier{};\bracktier{4,2};\bracktier{};\bracktier{};\bracktier{1};\bracktier{}}$$

    Applying the palandromic algorithm,
    \begin{align*}
        \F(\C) &= \F(\bracksig{\bracktier{8};\bracktier{};\bracktier{4,2};\bracktier{};\bracktier{};\bracktier{1};\bracktier{}})\\
        &= \F(\bracksig{\bracktier{4};\bracktier{4,2};\bracktier{};\bracktier{};\bracktier{1};\bracktier{}})\\
        &= \F(\bracksig{\bracktier{2,4,2};\bracktier{};\bracktier{};\bracktier{1};\bracktier{}})\\
        &= \F(\bracksig{\bracktier{2,2};\bracktier{};\bracktier{1};\bracktier{}})\\
        &= \F(\bracksig{\bracktier{2};\bracktier{1};\bracktier{}})\\
        &= \F(\bracksig{\bracktier{1, 1};\bracktier{}})\\
        &= \F(\bracksig{\bracktier{1}})\\
        &= \true
    \end{align*}

    So $\A$ does strongly respect $\B$. This can also be seen intuitively by looking at the bracket: teams in each tier have the same exact path throughout the tournament.

    \fig{0.6}{806001 tiered}{$\bracksig{\bracktier{8};\bracktier{};\bracktier{4,2};\bracktier{};\bracktier{};\bracktier{1};\bracktier{}}$}

    Finally, we leave as an exercise to the reader to use the palandromic algorithm to show that the 2016 Olympic Basketball Tournament was strongly respectful.

    Hopefully, these three examples have given a sense as to why the palandromic algorithm accurately ascertains whether a bracket signature strongly respects a tiered seeding. We will prove it by induction.

    \theo{}{
        The palandromic algorithm correctly ascertains whether a bracket signature strongly respects a tiered seeding.
    }{
        Let $\A$ be a bracket signature with $r$ rounds and $\B$ be a tiered seeding. If $\A$ doesn't weakly respect $\B$, then the palandromic algorithm will correctly say that it doesn't strongly respect $\B$ either. Assume then that $\A$ does weakly respect $\B$, where $\C = \bracksig{\C_{0}; ...; \C_{r}}.$ is the tiered signature of the pair $(\A, \B).$\\

        We proceed by induction on $r$. If $r = 0$, then $\A = \bracksig{1}$, $\B = \bracktier{1}$, and $\C = \bracksig{\bracktier{1}}.$ The palandromic algorithm will correctly claim that $\A$ strongly respects $\B$ without any recursive calls.\\

        For any other $r$, we will show that the palandromic algorithm returns $\false$ if and only if $\A$ does not strongly respect $\B$.\\
        
        Assume first that $A$ does not strongly respect $B$. Then, for some tier, either teams in that tier are not all equally likely to make it out of the first round, or they are not all equally likely to win the bracket, conditional on having made it out of the first round. In the former case, this would be caused by teams in the same tier having first-round matchups in different tiers, meaning $C_0$ would not be palandromic, and so the palandromic algorithm would fail on its first iteration. In the latter case, this would imply that $\D = \bracksig{\D_0; \C_2; ...; \C_{r}}$ is not a strongly respectful tiered signature, (where $\D_0$ is the right half of $\C_0$ concatenated with $\C_1$), so by induction, $\F(\C) = \F(\D) = \false.$ In either case, the palandromic algorithm correctly identifies that $A$ does not strongly respect $B.$\\

        Now, assume that the palandromic algorithm returns $\false.$ If it did so in the first iteration, then that means that there are two tiers $T_0, T_1$ for which some but not all teams in $T_0$ are matched up in the first-round against teams in $T_1$. Consider a list of teams such that teams in $T_1$ always lose, and all other games are coin-flips. Then, the teams in $T_0$ matched up against $T_1$ teams in the first-round will win the tournament with probability $(0.5)^{r-1}$, while the teams that are not will win with probability $0.5^r$, so $\A$ does not strongly respect $\B$.\\
        
        Meanwhile, if the palandromic algorithm failed at a later iteration, then by induction, $\D = \bracksig{\D_0; \C_2; ...; \C_{r}}$ is not a strongly respectful tiered signature, (where $\D_0$ is the right half of $\C_0$ concatenated with $\C_1$). However, if we consider a set of teams such that all of the first-round matchups in $\C$ are guaranteed wins for the higher tier, then a team's probability of winning the entire bracket (as long as they are in a tier that will win in the first-round) is the same as their probability of winning $\D.$ Because $\D$ is not a strongly respectful tiered signature, some teams in the same tier have different tournament-win probabilities, so $\C$ is also not strongly respectful. Thus, $\A$ does not strongly respect $\B$.\\

        So by induction, the palandromic algorithm claims that a bracket signature strongly respects a tiered seeding if and only if it truly does so.
    }{}

    With the palandromic algorithm in our back pocket, we can now quickly identify the relation between a given bracket signature and tiered seeding: whether it is strongly, weakly, properly, or not at all respectful. The concept of tiered seedings will show up in a few different places down the line: tiers are a powerful and generalizable tool for understanding tournament formats from Wimbledon to the NCAA Softball Tournament to the World Cup, as we shall investigate in the coming sections.
}