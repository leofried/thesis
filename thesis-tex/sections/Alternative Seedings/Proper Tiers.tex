\sub{



}


% Finally, weak and strong respectfulness give names to the intuition that teams in the same tier should be treated similarly. There is a separate intuition that teams in higher tiers should be treated better than teams in lower tiers. To that end, we introduce the notion of \textit{proper respectfulness}.

%     \begin{definition}{Properly Respectful}{}
%         A bracket \textit{properly respects} a tiered seeding $\B = \bracktier{b_1, ..., b_m}$ if the bracket is proper and teams in tier $i$ are given seeds $1 + \sum_{j=1}^{i-1} b_j$ through $\sum_{j=1}^{i} b_j.$
%     \end{definition}

%     If a bracket is not properly respectful, then lower tiers might grant more byes or an easier road to winning the tournament than higher tiers. The 2016 Olympic Basketball Bracket is properly respectful, as well as weakly and strongly respectful.


    % \begin{oq}{}{}
    %     Are properly respectful brackets always best?
    % \end{oq}

    % To illustrate the point, consider the bracket signature $\bracksig{2;3;2;0;0}$ and tiered seeding $\bracktier{2, 3, 1, 1}.$ The properly respectful bracket is on the left, but a potentially better bracket is on the right.

    % \fig{0.65}{23200 tiered}{Signature $\bracksig{2;3;2;0;0}$ with Seeding $\bracktier{2, 3, 1, 1}$}

    % In the properly respectful bracket, one tier 3 team gets extremely lucky: they have an easier quarterfinal matchup (a tier 4 team rather than another tier 3 team), and an easier semifinal matchup (a tier 2 team rather than a tier 1 team). In the bracket on the right, the luck is more evenly distributed: the tier 3 team that draw the easier quarterfinal matchup also gets the harder semifinal one.

    % A more trivial but more clear example of the same effect is with bracket signature $\bracksig{4;2;0;0}$ and tiered seeding $\bracktier{5, 1}.$ Again the properly respectful bracket is displayed on the left and the alternative bracket is on the right.
    
    % \fig{0.65}{4200 tiered}{Signature $\bracksig{4;2;0;0}$ with Seeding $\bracktier{5, 1}$}

    % In the properly respectful bracket, one tier 2 team gets a first round bye, and dodges the lone tier 1 team until the final. The alternative bracket distributes the advantage by having the tier 2 team that receives the bye be matched up with tier 1 team in the semifinals.

    % Unfortunately, this notion of distributing the luck more fairly is difficult to make rigorous. Additionally, it requires sometime using non-proper bracket \textit{shapes} meaning we lose access to the powerful fundamental theorem. Finally, we can avoid this odd effect by ensuring that our bracket signatures strongly respect the tiered seedings that are given to us, if possible. (As we will show next section, a strongly respectful bracket will give each team in a given tier the same path to win the tournament.) For these reasons, we focus primarily on properly respectful brackets, even though there is a compelling argument to be made that other brackets might be preferable in certain circumstances.