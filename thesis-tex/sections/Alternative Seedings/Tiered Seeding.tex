\sub{

    Consider the 2016 Olympic Basketball Tournament. Twelve teams qualified for the Olympics, and they were divided into two groups of six teams each. Each group conducted a mini-tournament, ranking the teams in each group from first through sixth (the specifics of the mini-tournament are not relevant). Then, the bottom two teams in each group were eliminated, with the remaining eight teams (four from each group) entering the bracket $\bracksig{8;0;0;0}.$ The entire format as it played out is displayed in Figure \ref{fig:olympics}.

    \fig{0.78}{olympics}{The 2016 Olympic Basketball Tournament}

    The seeding going into the bracket portion of the 2016 Olympic Basketball Tournament is a little different than the seedings that we have discussed so far. Rather than the ranking of the teams being a complete ordering, it is a partial one: teams are grouped into tiers, and the tiers are ranked. Two teams, one from each group, occupy each tier.

    \begin{definition}{Tiered Seeding}{}
        A \textit{tiered seeding} is a partial ordering on the teams entering a tournament.
    \end{definition}

    This is as opposed to traditional seedings, which are a complete ordering (although we can view a traditional seeding as a special example of a tiered seeding where each tier has a single team). When filling out a bracket using a tiered seeding, we continually assign the top remaining seeds to the teams in the top remaining tier. Recall the proper bracket $\bracksig{8;0;0;0}$:

    \fig{0.8}{eight}{$\bracksig{8;0;0;0}$}

    The United States and Croatia, as the two teams in the top tier, are given seeds one and two. The two tier-two teams, Australia and Spain, get seeds three and four, and so on. The actual algorithm used for assigning the seeds to the teams within each tier can be arbitrary: in the particular case of the 2016 Olympic Basketball Tournament, teams from Group A were given the odd seeds and teams from Group B the evens.

    We can describe a tiered seeding with a list of integers indicating how many teams are in each tier. The eight teams that advanced to bracket at the Olympics were divided into four pools of two teams each, so we write $\bracktier{2,2,2,2}.$ A quick notational note: we list tier sizes in reverse order, with the size of the lowest tier coming first, and the size of the top tier coming last. This is done to keep it consistent with bracket signatures, in which the lower-seeded teams are listed earlier, and higher-seeded teams that get more byes are listed later.

    The tiered seeding $\bracktier{2,2,2,2}$ interacts very nicely with the proper bracket $\bracksig{8;0;0;0}$: there is no advantage for a team being assigned a particular seed within their tier.

    \begin{definition}{Strongly Respecful}{}
        A bracket \textit{strongly respects} a tiered seeding if, as long as teams' win probabilities are defined only by what tier they are in (that is, for all $t_i, t_j$ that share a tier and $t_k, t_\ell$ that share a tier, $\G{t_i}{t_k} = \G{t_j}{t_\ell}$), then teams in the same tier have the same probability of winning the tournament.
    \end{definition}

    Sometimes, it is not possible to generate a bracket that strongly respects a tiered seeding (for example, the tiered seeding $\bracktier{3, 1}$), so we also introduce the concept of a bracket weakly respecting a tiered seeding.

    \begin{definition}{Weakly Respectful}{}
        A bracket \textit{weakly respects} a tiered seeding if each team in a tier is given the same number of byes.
    \end{definition}

    Although no signature strongly respects the tiered seeding $\bracktier{3, 1},$ the bracket $\bracksig{4;0;0}$ is preferable to $\bracksig{2;1;1;0}$ because at least it weakly respects it. The names of the two conditions come from strong respectfulness being a stronger condition than weak respectfulness.

    \theo{}{If a bracket strongly respects a tiered seeding then it weakly respects it as well.}{
        If a bracket strongly respects a tiered seeding, then all teams within the same tier must have the same probability of winning the tournament if every game is a coin flip. If indeed every game is coinflip, two teams have the same chance of winning the tournament only if they have the same number of byes, so the bracket must weakly respect the tiered seeding as well.
    }{}

    Finally, weak and strong respectfulness give names to the intuition that teams in the same tier should be treated similarly. There is a separate intuition that teams in higher tiers should be treated better than teams in lower tiers. To that end, we introduce the notion of \textit{proper respectfulness}.

    \begin{definition}{Properly Respectful}{}
        A bracket \textit{properly respects} a tiered seeding $\B = \bracktier{b_1, ..., b_m}$ if the bracket is proper and teams in tier $i$ are given seeds $1 + \sum_{j=1}^{i-1} b_j$ through $\sum_{j=1}^{i} b_j.$
    \end{definition}

    If a bracket is not properly respectful, then lower tiers might grant more byes or an easier road to winning the tournament than higher tiers. The 2016 Olympic Basketball Bracket is properly respectful, as well as weakly and strongly respectful.

    How can we tell whether an arbitrary bracket strongly, weakly, or properly respects an arbitrary tiered seeding?  Proper respectfulness is the easiest: just check if the bracket is proper and then if the tiers are being assigned to the right seeds.

    Checking if a bracket signature weakly respects a tiered seeding is also somewhat straightforward: we can simply matchup the signature with the tiered seeding to see if we ever have to split a tier across two different levels in the signature. For example, consider the bracket signature $\A = \bracksig{8;6;3;0;0;0}$ and the tiered seeding $\bracktier{4,4,4,4,1}.$ The four teams in the second-highest tier are distributed over two levels: two the them (seeds 2 and 3) get two byes and two of them (seeds 4 and 5) get a single bye, so $\bracksig{8;6;3;0;0;0}$ does not weakly respect $\bracktier{4,4,4,4,1}.$

    \fig{0.6}{863000 seeded}{$\bracksig{8;6;3;0;0;0}$}

    If a bracket signature \textit{does} weakly respect a tiered seeding, we can combine the information of the bracket signature and the tiered seeding into a single list of lists called the \textit{tiered signature}.

    \begin{definition}{Tiered Signature}{}
        If a bracket signature $\A = \bracksig{a_0; ...; a_r}$ weakly respects a tiered signature $\B$, then the \textit{tiered signature} of the signature-seeding pair $(\A, \B)$ is a list $\C = \bracksig{\C_0; ...; \C_r}$ where $\C_i$ is the sublist of $\B$ corresponding to the $a_i$ teams that get $i$ byes.
    \end{definition}

    The bracket $\A = \bracksig{8;6;3;0;0;0}$ weakly respects the tiered seeding $\B = \bracktier{4,4,4,2,2,1},$ and the associated tiered signature of this pair is $$\C = \bracksig{\bracktier{4, 4};\bracktier{4,2};\bracktier{2,1};\bracktier{};\bracktier{};\bracktier{}}.$$ The somewhat trivial tiered signature of the 2016 Olympic Basketball Tournament is $$\bracksig{\bracktier{2, 2, 2, 2};\bracktier{};\bracktier{};\bracktier{}}.$$

    Note that we can easily extract both the bracket signature and the tiered seeding from the tiered signature. For the former, sum each sublist, and for the latter, concatenate the sublists into a single list. Sometimes, we will refer a tiered signature as being strongly respectful as a shorthand for saying that the associated tiered seeding respects the associate bracket signature.

    Checking for strong respectfulness seems to be much trickier than weak respectfulness. Somehow, we need to be able to verify that for any distribution of win probabilities, (as long as teams within the same tier have the same matchup table,) teams within the same tier have the same probability of winning the tournament. Luckily, there is a simple algorithm for doing just that, which we will explore in the next section.
}




    % \begin{oq}{}{}
    %     Are properly respectful brackets always best?
    % \end{oq}

    % To illustrate the point, consider the bracket signature $\bracksig{2;3;2;0;0}$ and tiered seeding $\bracktier{2, 3, 1, 1}.$ The properly respectful bracket is on the left, but a potentially better bracket is on the right.

    % \fig{0.65}{23200 tiered}{Signature $\bracksig{2;3;2;0;0}$ with Seeding $\bracktier{2, 3, 1, 1}$}

    % In the properly respectful bracket, one tier 3 team gets extremely lucky: they have an easier quarterfinal matchup (a tier 4 team rather than another tier 3 team), and an easier semifinal matchup (a tier 2 team rather than a tier 1 team). In the bracket on the right, the luck is more evenly distributed: the tier 3 team that draw the easier quarterfinal matchup also gets the harder semifinal one.

    % A more trivial but more clear example of the same effect is with bracket signature $\bracksig{4;2;0;0}$ and tiered seeding $\bracktier{5, 1}.$ Again the properly respectful bracket is displayed on the left and the alternative bracket is on the right.
    
    % \fig{0.65}{4200 tiered}{Signature $\bracksig{4;2;0;0}$ with Seeding $\bracktier{5, 1}$}

    % In the properly respectful bracket, one tier 2 team gets a first round bye, and dodges the lone tier 1 team until the final. The alternative bracket distributes the advantage by having the tier 2 team that receives the bye be matched up with tier 1 team in the semifinals.

    % Unfortunately, this notion of distributing the luck more fairly is difficult to make rigorous. Additionally, it requires sometime using non-proper bracket \textit{shapes} meaning we lose access to the powerful fundamental theorem. Finally, we can avoid this odd effect by ensuring that our bracket signatures strongly respect the tiered seedings that are given to us, if possible. (As we will show next section, a strongly respectful bracket will give each team in a given tier the same path to win the tournament.) For these reasons, we focus primarily on properly respectful brackets, even though there is a compelling argument to be made that other brackets might be preferable in certain circumstances.