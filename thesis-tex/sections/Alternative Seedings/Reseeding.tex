\sub{
% One might reasonably object to randomized cohort seeding being used as a method to enforce ordereness on the grounds that the format is only ordered in expectation over the randomness. Once the bracket is constructed, the format is no longer ordered, and since the players have no ability to influence the randomness present in constructing of the bracket, this is still unfair. Reseeding is an alternative seeding method that aims to preserve ordeness without introducing randomness.

While Edwards's Theorem is certainly a very powerful result, with a little clever engineering we can still construct a bracket-like tournament without requiring that the number of rounds grow linearly with the number of participating teams.

Ultimately, the reason that proper brackets are not, in general, ordered, is that lower-seeded teams are treated, if they win, as the team that they beat for the rest of the format. Consider again the proper bracket analyzed by Silver: $\bracksig{16; 0; 0; 0; 0}.$ If an 11-seed wins in the first-round, they take on the schedule of a 6-seed for the rest of the tournament, while if the 9-seed wins, they take on the schedule of an 8-seed. Given that a 6-seed has an easier schedule than an 8-seed, it's not hard to see why it might be preferable to be an 11-seed rather than a 9-seed.

\textit{Reseeding} (poorly named) fixes this by resorting the match-ups every round: if an 11-seed keeps winning, they will have to play teams according to their seed, rather than getting an effective upgrade to 6-seed status.

\begin{definition}{Reseeded Brackets}{}
    In a \textit{reseeded} bracket, after each round, match-up the highest-seeded team with the lowest-seeded team, second highest vs second-lowest, etc.
\end{definition}

Note that by Definition \ref{def:bracket}, a reseeded bracket is not a bracket at all, as matchups between teams that have not yet lost are not determined in advance of the outcomes of any games. However, because reseeded brackets act so similarly to traditional brackets, and because colloquially they are referred to as brackets, we opt to continue using the word ``bracket'' to describe them.

Both National Football League conferences use a reseeded bracket with signature $\bracksig{6; 1; 0; 0}.$ If the first-round of the bracket goes chalk, then it looks just like a normal bracket:

\fig{0.9}{2023afc}{2023 National Football League AFC Playoffs}

The dotted lines are drawn after the first round of games have been played: if there are some first-round upsets, then the bracket is rearranged to ensure that it still better to be a higher seed rather than a lower seed.

\fig{0.9}{2023nfc}{2023 National Football League NFC Playoffs}

In the NFC, 6-seed New York upset 3-seed Minnesota. Had a conventional bracket been used, the semifinal matchups would have been 1-seed vs 5-seed and 2-seed vs 6-seed: the 2-seed would have had an easier draw than the 1-seed, while the 6-seed would have an easier draw than the 5-seed. Reseeding fixes this by matching 6-seed New York is with top-seed Philadelphia, and 2-seed San Francisco with 5-seed Dallas.

Reseeding is powerful technique. For one, the fundamental theorem still applies to reseeded brackets.

\theo{}{
    There is exactly one proper reseeded bracket with each bracket signature.
}{
    The definition of properness ensures that there is only one way byes can be distributed such that a reseeded bracket can be proper. Additionally, because reseeded brackets have no additional paramters beyond which seeds get how many byes, there is no more than one reseeded bracket with each signature that could be proper. Finally, that bracket is indeed proper: if the bracket goes to chalk, the matchups will be the exact same as a traditional bracket, which by the fundamental theorem is a proper set of matchups.
}{}

And, more excitingly, Edward's theorem does \textit{not} apply (a fact first shown by Hwang \cite{reseeding}).

\begin{definition}{$\WW$}{}
    If $\A$ is a reseeded bracket, and $\T$ is a set of teams including $t$, then $\WW_\A(t, \T)$ is the probability that $t$ wins the tournament $\A$ when played by $\T.$
\end{definition}

\lemm{}{
    If $\A = \bracksig{a_0; ...; a_r}$ is a proper reseeded bracket on $n$ teams, $\T + \{x, y\}$ is an SST list of $n + 1$ teams, and $x$ and $y$ are one seed apart in $\T + \{x, y\}$, then $\WW_{\A}(x, \T + \{x\}) \geq \WW_\A(y, \T + \{y\}).$
}{
    We proceed by induction on $r$. If $r = 0$, then $\A = \bracksig{1}$ and so the theorem is immediately true. For any other $r$, if $x$ would receive a first-round bye when $\A$ is run on $\T + \{x\}$ (and thus as would $y$ when $\A$ is run on $\T + \{y\}),$ then the statement follows immediately by induction. \\
    
    Otherwise, let $\B = \bracksig{a_0 / 2 + a_1; a_2; ...; a_r},$ let $z$ be the first-round opponent of $x$, let $\Y$ be any subset of $\T - \{z\}$ of size $n - a_0/2 - 1$, and let $P(\Y)$ be the probability that all teams in $\Y$ advance to the second round when $\A$ is run on $\T + \{x\}$ (or $\T + \{y\}$). Then,
    \begin{align*}
        &\WW_\A(x, \T + \{x\})\\
        =\; &\P{\G{x}{z}} \cdot \sum_\Y \WW_\B(x, \Y + \{x\}) \cdot P(\Y)\\
        \geq\; &\P{\G{x}{z}} \cdot \sum_\Y \WW_\B(y, \Y + \{y\}) \cdot P(\Y) & \textrm {by induction}\\
        \geq\; &\P{\G{y}{z}} \cdot \sum_\Y \WW_\B(y, \Y + \{y\}) \cdot P(\Y) & \textrm {bc $\T + \{x, y\}$ is SST}\\
        =\; &\WW_\A(y, \T + \{y\})
    \end{align*}
}{}




\theo{}{
    All proper reseeded brackets are ordered.
}{
    % Let $\A = \bracksig{a_0; ...; a_r}$ be a proper reseeded bracket. We proceed by induction on $r$. If $r = 0$, then the only possible bracket is $\bracksig{1},$ which is immediately ordered.\\

    % For any other $r$, we show that $\P{\W{t_i}{\A}} \geq \P{\W{t_{i+1}}{\A}}.$ 
}{}

However, reseeding is not without its drawbacks. In a reseeded bracket, teams and spectators alike don't know who they will play or where their next game will be until the entire previous round is complete. This can be an especially big issue if parts of the bracket are being played in different locations on short turnarounds: in the 2022 NCAA Women's Basketball Tournament, the first two rounds are played over a weekend on the various college campuses of the highest-seeded teams. It would cause problems if teams had to pack up and travel across the country because their opponent changed because of reseeding.

In addition, part of what makes the NCAA Basketball Tournament (affectionately known as ``March Madness'') such a fun spectator experience is that fact that these matchups are known ahead of time. In ``bracket pools,'' groups of fans each fill out their own brackets, predicting who will win each game and getting points based on how many they get right. If it wasn't clear where in the bracket the winner of a given game was supposed to go, this experience would be diminished.

Finally, reseeding gives the top-seed(s) an even greater advantage than they already have: instead of playing against merely the \textit{expected} lowest-seeded team(s) each round, they would get to play against the \textit{actual} lowest-seeded team(s). In March Madness, ``Cinderella Stories,'' that is, deep runs by low seeds, would become much less common.

In many ways, the NFL conferences playoffs are a perfect place to use a reseeded bracket: games are played once a week, giving plenty of time for travel; only seven teams make the playoffs in each, so a huge March Madness-style bracket challenge is unlikely; as a professional league, the focus is far more on having the best team win and protecting Cinderella Stories isn't as important; and because the bracket is only three rounds long, reseeding is only required once.

Other leagues with similar structures might consider adopting forms of reseeding to protect their incentives and competitive balance (looking at you, Major League Baseball), but in many cases, the traditional bracket structure is too appealing to adopt a reseeded one.
}