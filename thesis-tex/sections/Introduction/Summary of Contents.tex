\sub {

    We begin our analysis in \bsec{Definitions}, where we formally develop the notions of games, matchup tables, tournament formats, and networked tournament formats, setting the stage for the rest of the thesis.
    
    \bcha{Brackets} focuses on \i{brackets}: a kind networked format with the additional restriction that teams are eliminated after their first loss, and games are played until only a single undefeated remains. We note that a bracket is defined by its \i{shape}, the binary tree that determines the matchup between game winners, and its \i{seeding}, which tells each team which node of the tree to start in.

    We begin with the shape in \bsec{Bracket Signatures}, where we define the \i{bracket signature}, a compression of the the shape of a bracket into a list of natural numbers specifying how many teams get each number \i{byes} (that is, how many games each team must win in order to win the tournament). We prove that an arbitrary list satisfies a specific formula if and only if it is a bracket signature for some bracket. Finally we introduce the notion of a \i{balanced} bracket, one with no byes in which every team is tasked with winning the same number of games.

    In \bsec{Proper Brackets}, we move on to the \i{seeding} of a bracket, observing that in real tournaments, seedings are used to give advantages to the better and more deserving teams. We formalize this practice into that of a \i{proper seeding}, and then prove the \i{fundamental theorem of brackets}: there is exactly one proper bracket with each bracket signature.

    \bsec{Ordered Brackets} introduces Edwards's \cite{montana} notion of an \i{ordered bracket}, a bracket in which a team's odds of winning the format monotonically increases with the skill of the team. We show that all ordered brackets are proper, and set the stage for Edwards's Theorem, which fully categorizes the space ordered brackets, by looking at some simple brackets and determining their orderedness.

    While the key results from the previous three sections concerning properness and signatures were all novel, \bsec{Edwards's Theorem} is dedicated to a proof of its namesake theorem, which was of course first proved in \i{The Combinatorial Theory of Single-Elimination Tournaments}. Still, we offer a much quicker proof of the statement: we first make use of the fundamental theorem to establish two novel lemmas that relate the orderedness of brackets to the sub-brackets that comprise them, before then using the lemmas to derive the theorem.

    Edwards's Theorem turns out to be quite constraining on the space of ordered brackets: the balanced brackets for three or more rounds (eight or more teams) are not ordered. This can be quite disturbing given that one of the primary reasons for using a bracket over other tournament formats is that they can crown a champion in only a logarithmic number of rounds: requiring orderedness makes this impossible. 

    We spend the next two sections attempting to solve this problem. Our first attempt, in \bsec{Reseeded Brackets}, is to use \i{reseeding}, a modification to brackets where after each round, the matchups are rearranged to pair the top seeds with the bottom seeds. Hwang \cite{reseeding} actually published a proof that reseeding allows for balanced ordered brackets for any number of rounds. Unfortunately, we find that his proof was incorrect, and using a nearly identical process as our new proof of Edward's Theorem, complete determine the space of ordered reseeded brackets. Balanced reseed brackets, too, are ordered only for two or fewer rounds.

    Finally, in \bsec{Randomization}, we attempt a second approach to the problem presented by Edward's Theorem by randomizing which teams go where in the bracket. We cite Chen and Hwang's \cite{totally_random_balanced} proof that total randomization does in fact allow for a balanced ordered brackets of arbitrary size. Unfortunately, total randomization can lead to \i{unexciting} formats, where all the best matchups are played very early on. We also consider Wimbledon-style randomization, which ensures that these matchups are delayed until the later rounds, but ultimately show that they too are not ordered for more than two rounds.

    Thus the only balanced ordered \i{knockout tournament} (that is, bracket-like) format for more than two rounds that we have located is the totally randomized one, which has the dual undesirable properties of being unexciting and non-deterministic. We conclude the chapter by asking the two natural open questions: does there exist a balanced ordered knockout-tournament for arbitrary numbers of rounds that is deterministic, and does there exist a balanced ordered knockout-tournament for arbitrary numbers of rounds that is exciting. We pessimistically conjecture that the answer to both questions is no.

    



    % In \bsec{Reseeded Brackets}





}