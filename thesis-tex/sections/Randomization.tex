\sub {

    INTRODUCTION

    \begin{definition}{Randomly Seeded Bracket}{}
        A \i{randomly seeded bracket} is a traditional bracket except the teams are randomly placed onto the starting lines instead of being placed according to seed.
    \end{definition}

    This randomization means that the shape of a randomly seeded bracket uniquely defines it. There is a no notion of two randomly seeded brackets having the same shapes but different seedings, as the teams are randomly placed in the bracket the beginning of the format.

    Chung and Hwang conjectured that all randomly seeded brackets were ordered \cite{define_ordered}. After all, the teams are all being treated identically: how could a better team be at a disadvantage relative to a worse one?
    
    \begin{conj}{}{randord}
        All totally randomized brackets are ordered.
    \end{conj}

    Indeed, Lemma \ref{th:rand_lemma} seems to provide som evidence for the conjecture.

    \lemm{}{
        % Let $\A$ be a randomly seeded bracket, let $\S$ be a list of teams, and let $\T$ be identical to $\S$ except with $t_i$ replaced by a team that is at least as good as $t_i$ against every other team. Then $\W{t_i}{\S}{\T}
    }{

    }{rand_lemma}

    
    Unfortunately, despite the lemma, Conjecture \ref{conj:randord} turns out to be false. This was first shown by Israel with the following counterexamle \cite{seventeen_team}.

    \fig{1}{17teams}{The Proper Bracket Shape of Signature $\bracksig{16;0;0;0;1;0}$}

    \theo{}{
        The randomly seeded bracket whose shape is the proper bracket shape of signature $\bracksig{16;0;0;0;1;0}$ is not ordered.
    }{
        Let $\A$ be the format in question, and let $\T$ be the list of seventeen teams containing one copy of each of $t_1, t_3, t_4,$ and $t_5,$ and thirteen copies of $t_2$ with the following matchup table.

        \begin{center}
            \begin{tabular}{c | c c c c c c c}
    & $t_1$ & $t_2$ & $t_3$ & $t_4$ & $t_5$\\ 
    \hline
    $t_1$ &  &  &  &  & \\
    $t_2$ & $0.5$ &  &  &  & \\
    $t_3$ & $0$ & $2p$ &  &  &  \\
    $t_4$ & $0$ & $p$ & $2p$ &  & \\
    $t_5$ & $0$ & $p$ & $p$ & $0.5$ & \\
            \end{tabular}
        \end{center}

        Let $i \in \{4, 5\}$ and let $j = 9-i.$ For $t_i$ to win $\A$ without getting placed on the red starting line, they must win at least four games against teams $t_1$, $t_2$, or $t_3$, which happens with probability $O(p^4).$ Thus we let $\B_i$ be the format identical to $\A$ except we enforce that $t_i$ will be placed on the red starting line and note that
        $$\W{\A}{t_i}{\T} = \frac{1}{17}\W{\B_i}{t_i}{\T} + O(p^4).$$
        
        Now $t_j$ reaches the finals of $\B_i$ with probability $O(p^4)$, $t_3$ reaches the finals of $\B_i$ with probability $O(p^3)$ and so $t_i$ beats them in the finals with probability $O(p^4)$, and of course $t_i$ cannot beat $t_1$ in the finals. Thus,
        \begin{align*}
            \W{\B_i}{t_i}{\T} &= p \cdot \P{\textrm{$t_2$ reaches the finals of $\B_i$}} + O(p^4).
        \end{align*}
        Since $t_3$ and $t_j$ reach the finals of $\B_i$ with probability $O(p^3)$ and $O(p^4)$ respectively,
        \begin{align*}
            \W{\B_i}{t_i}{\T} &= p \cdot \P{\textrm{$t_1$ doesn't reach the finals of $\B_i$}} + O(p^4).
        \end{align*}

        Assume without loss of generality that $t_1$ gets placed on the orange starting line.\\

        Any difference in $\P{\textrm{$t_1$ doesn't reach the finals of $\B_i$}}$ between $i \in \{4, 5\}$ will have to come as a result of a game involving $t_j$ (as $t_j$ is the only difference in $t_1$'s route to the finals between $\B_4$ and $\B_5$), and because $t_4$ and $t_5$ have the same probability of beating every team other than $t_3$, it will have to be as a result of a game against $t_3.$ However, because neither $t_3$ nor $t_j$ can beat $t_1$, they will have to be placed on two colored starting lines of the same color.\\

        If $t_3$ and $t_j$ are placed on two of the light blue or dark blue starting lines, then any difference in $\P{\textrm{$t_1$ doesn't reach the finals of $\B_i$}}$ between $i \in \{4, 5\}$ will be induced by $t_j$ winning its first three games, with happens with probability $O(p^3).$\\

        However, if $t_3$ and $t_j$ are placed on the two dark green or two light green starting lines, then when $i = 4$, $t_1$  will play $t_2$ in the yellow game with probability $$p_{35}p_{23} + p_{53}p_{25} = ((1-p)(1-2p) + (p)(1-p)) = 1-2p+p^2,$$ while when $i = 5$, $t_1$ will play $t_2$ in the yellow game with probability $$p_{34}p_{23} + p_{43}p_{24} = ((1-2p)(1-2p) + (2p)(1-p)) = 1-2p+2p^2.$$
        Thus, \begin{align*}
            &\P{\textrm{$t_1$ plays $t_2$ in the yellow game of $\B_5$}}\\
            - \;&\P{\textrm{$t_1$ plays $t_2$ in the yellow game of $\B_4$}}\\
             = \;&cp^2 + O(p^3)
        \end{align*}
        for some constant $c$, so
        \begin{align*}
            &\P{\textrm{$t_1$ doesn't reach the finals of $\B_5$}}\\
            - \;&\P{\textrm{$t_1$ doesn't reach the finals of $\B_4$}}\\
             = \;&cp^2 + O(p^3)
        \end{align*}
        for some constant $c$, so
        $$\W{\B_5}{t_5}{\T} - \W{\B_4}{t_4}{\T} = cp^3 + O(p^4)$$
        for some constant $c$, so
        $$\W{\A}{t_5}{\T} - \W{\A}{t_4}{\T} = cp^3 + O(p^4)$$ for some constant $c.$\\

        Therefore $\A$ is not ordered.
    }{}

    Chung and Hwang's conjecture was rescued by Chen and Hwang who restricting to the domain of the claim to balanced formats \cite{totally_random_balanced}.

    \theo{}{
        All randomly seeded balanced brackets are ordered.
    }{
        Let $\A_r$ be the randomly seeded balanced bracket on $2^r$ teams. We induct on $r$. Clearly the one-team format $\A_0$ is ordered. For any other $r$, let $\T$ be a list of teams, and let $t_i$ and $t_j$ be teams such that $i < j.$\\

        Let $\B_r$ be the randomly seeded balanced bracket on $2^r$ teams except $t_i$ and $t_j$ are forced to play each other in the first round, and let $\C_r$ be the randomly seeded balanced bracket on $2^r$ teams except $t_i$ and $t_j$ cannot play each other in the first round. Then,
        $$\W{A_r}{t_i}{\T} = \left(\frac{1}{2^r-1}\right)\W{B_r}{t_i}{\T} + \left(\frac{2^r-2}{2^r-1}\right)\W{C_r}{t_i}{\T}$$ and likewise for $t_j.$\\

        Because $p_{ij} \geq p_{ji}$, and by Lemma \ref{th:rand_lemma}, $\W{B_r}{t_i}{\T} \geq \W{B_r}{t_j}{\T}.$ Thus left is to show that $\W{C_r}{t_i}{\T} \geq \W{C_r}{t_j}{\T}.$\\
        
        For two other teams $t_a$ and $t_b$, let $M_{ab}$ be the set of $2^{r-1}-2$ team subsets of $\T \setminus \{t_i, t_j, t_a, t_b\},$ and for $\S \in M_{ab},$ let $P_\S$ be the probability that the teams in $\S$ all win their first-round games and none of them play any of $t_i, t_j, t_a,$ or $t_b$ in the first round.

        Now,
        \begin{align*}
            \W{C_r}{t_i}{\T} &= \frac{1}{2}\sum_{t_a, t_b \in \T \setminus \{t_i, t_j\}} \sum_{\S \in M_{ab}} P_\S \cdot ((p_{ia}p_{jb} + p_{ib}p_{ja}) \cdot \W{A_{r-1}}{t_i}{\S \cup \{t_i, t_j\}}\\
            &+p_{ia}p_{bj} \cdot \W{A_{r-1}}{t_i}{\S \cup \{t_i, t_b\}}+p_{ib}p_{aj} \cdot \W{A_{r-1}}{t_i}{\S \cup \{t_i, t_a\}})\\
            &\geq \frac{1}{2}\sum_{t_a, t_b \in \T \setminus \{t_i, t_j\}} \sum_{\S \in M_{ab}} P_\S \cdot ((p_{ia}p_{jb} + p_{ib}p_{ja}) \cdot \W{A_{r-1}}{t_j}{\S \cup \{t_i, t_j\}}\\
            &+p_{ja}p_{bi} \cdot \W{A_{r-1}}{t_j}{\S \cup \{t_j, t_b\}}+p_{jb}p_{ai} \cdot \W{A_{r-1}}{t_j}{\S \cup \{t_j, t_a\}})\\
            &=\W{C_r}{t_i}{\T}
        \end{align*}

        The inequality follows by comparing each term to its corresponding term: the $\W{A_{r-1}}{t_i}{\S \cup \{t_i, t_j\}}$ inequality is by induction, while the other two terms are by 
    }{}











}



% %finish proofs

%     Consider the 2022 Wimbledon Mixed Double's Championship, whose bracket is depicted in Figure \ref{fig:wimbledon}.

%     \fig{0.8}{wimbledon}{2022 Wimbledon Mixed Double's Championship}

%     The 2022 Wimbledon Mixed Double's Championship used a balanced bracket of signature $\bracksig{32;0;0;0;0;0}.$ However its seeding is certainly not proper, and in fact, looks quite strange.
    
%     For one thing, 24 of the 32 starting lines in the bracket are not seeded at all. What does this mean? In Wimbledon, only the top 25\% of teams are seeded. The other 75\% of teams are randomly placed in the remaining starting lines.

%     But secondly, even the top 8 teams are not seeded properly: if all games go chalk, the quarterfinals will be 1v5, 2v7, 3v5, and 4v6, and the semifinal will be 1v3 and 2v4. How did Wimbledon come up with such a bizarre seeding? Randomly.

%     Before each Wimbledon Mixed Double's Championship, the seeding is constructed in the following way: the 1- and 2-seeds are placed normally. Then half the 3- and 4-seeds are placed normally, and half the time they are swapped. Then the 5- through 8-seeds are randomly placed into the four starting lines that are normally assigned to those four seeds. Finally, the remaining 24 teams are placed randomly into the remaining spots.

%     Thus the bracket in Figure \ref{fig:wimbledon} is only one of the forty eight different possible brackets that could be used in the Wimbledon Mixed Double's Championship: it happens to be the one used in 2022.

%     Why does Wimbledon use such a strange format? Wimbledon is one of many tournaments on the ATP Tour (the set of tournaments played by the professional tennis players) that all use almost identical formats: large balanced brackets. Additionally, the seeding for these tournaments is set by the ATP rankings, which tend to be slow to update. As a result, if every ATP Tour tournament used a proper seeding, the 6-seed and 27-seed would play each other in the first round at every tournament until one of them moved up or moved down. These rematches were deemed undesirable and so this randomization procedure was introduced: The 1-seed's quarterfinals matchup (if everything goes chalk) is now randomly drawn from the 5- through 8-seeds, instead of always being the 8-seed.

%     Additionally, this particular way of grouping seeds ensures that after round $s$ for $s>1$, if the format goes chalk, the top $2^{r-s}$ out of $2^r$ teams remain. Finally, it might even allow a balanced bracket on more than four teams to be ordered, a feat that proper and reseeded brackets were both unable to accomplish.

%     The format used in the Wimbledon Mixed Double's Championship is a particular example of a class of formats called \i{cohort randomized brackets} (a generalization of Schwenk's cohort randomized seeding \cite{randomized_cohort}).

%     \begin{definition}{Composition}{}
%         A \i{composition} $\B = \bracktier{b_1 + ... + b_n}$ of a natural number $n$ is a way of writing $n$ as the sum of a sequence of natural numbers.
%     \end{definition}

%     We will use compositions to describe which seeds are to be randomized with which other seeds (each of these groups is called a cohort). In the particular case of the Wimbledon Mixed Double's Championship, the bottom 24 teams are randomized, as are the next 4 and 2 after that, and then the 2-seed and 1-seed aren't randomized at all. So the composition of the Wimbledon Mixed Double's Championship is $$\B = \bracktier{24 + 4 + 2 + 1 + 1}.$$

%     \begin{definition}{Cohort Randomized Bracket}{}
%         A \i{cohort randomized bracket} is an $n$-team tournament format parameterized $(\A, \B)$, where $\A$ is an $n$-team bracket signature and $\B = \bracktier{b_1 + ... + b_k}$ is a composition of $n$. The proper bracket of signature $\A$ is constructed, but, for each $j$, seeds $$n + 1 - \sum_{i=1}^j b_i$$ through $$n - \sum_{i=1}^{j-1} b_i$$ are shuffled randomly. The resulting bracket is played out normally.
%     \end{definition}

%     Thus the 2022 Wimbledon Mixed Double's Championship employs the cohort randomized bracket parameterized by $$(\bracksig{32;0;0;0;0;0}, \bracktier{24 + 4 + 2 + 1 + 1}).$$ 

%     But it is certainly not the simplest example of a cohort randomized bracket: a format that totally ignores seeding and just randomly places all the teams into a bracket is also cohort randomized. 
    
%     \begin{definition}{Totally Randomized Bracket}{}
%         The \i{totally randomized bracket} of the $n$-team signature $\A$ is the cohort randomized bracket parameterized by $(\A, \bracktier{n}).$
%     \end{definition}

%     Chung and Hwang had a conjecture about totally randomized brackets \cite{define_ordered}.
    
%     \begin{conj}{}{}
%         All totally randomized brackets are ordered.
%     \end{conj}

%     Certainly once the randomization is been complete and starting lines have been set this is not true: the resulting bracket will either be improper, in which case it is certainly not ordered, or proper, in which case it is ordered only when it satisfies the condition of Edwards's Theorem. However, in expectation \i{over the randomization}, it is natural to hope that all totally randomized brackets are ordered: there is no advantage to being one seed or another, so certainly the better teams should win more?

%     Unfortunately, as was the case in the previous section, the sweeping conjecture ultimately falls short, this time due to a counterexample given by Israel \cite{seventeen_team}.

%     \theo{}{
%         Let $\A = \bracksig{16;0;0;0;1;0}$. The totally randomized bracket $\B = (\A, \bracktier{17})$ is not ordered.
%     }{
%     %proof_needed

%         % Consider the following SST matrix on 17 teams $\T$, four teams named 1, 3, 4, and 5, and thirteen teams named 2, whose matchups are identical. The upper half of the matrix is implied.

%         % \begin{center}
%         %     \begin{tabular}{c | c c c c c}
%         %         & 1 & 2 & 3 & 4 & 5\\ 
%         %         \hline
%         %         1 & & & & &\\
%         %         2 & 0.5 & & & &\\
%         %         3 & 0 & $2p$ & & &\\
%         %         4 & 0 & $p$ & $2p$ & & \\
%         %         5 & 0 & $p$ & $p$ & 0.5 & \\
%         %         \end{tabular}
%         % \end{center}

%         % Assume $p$ is very small. We will show that $\W{\B}{4}{\T} < \W{\B}{5}{\T}.$ The difference will be $O(p^3)$, so terms of $O(p^4)$ will be dropped throughout the analysis.\\

%         % Now, let $i \in \{4, 5\}$ and let $j = 9 - i$. First note that if team $i$ were to win $\B$ while starting in any of the sixteen starting lines that receive no byes, they will need to win at least four games against teams 1, 2, and 3, which happens with probability $O(p^4)$. Thus we restrict our study to cases when $i$ start with the quadruple bye.\\

%         % In such a case, they would play team $j$ in the championship game with probability $O(p^4)$, team $3$ with probability $O(p^3)$ (but beat them with probability $O(p)$), and of course if they play team $1$ they have no chance to win. Thus, we can restrict out study to worlds where team $i$ beats (a) team $2$ in the championship game. So letting $\S = \T \setminus \{i\}$, and $\C$ be the totally randomized bracket $(\bracksig{16;0;0;0;0}, \bracktier{16})$,        $$\W{\B}{i}{\T} = \frac{1}{17} \cdot p \cdot \W{\C}{2}{\S} + O(p^4).$$
%     }{}

%     However, again analogously to the previous section, orderedness can be rescued for a smaller set of brackets, this time by Chen and Hwang \cite{totally_random_balanced}.

%     \theo{}{
%         All totally randomized balanced brackets are ordered.
%     }{
% %proof_needed
%     }{}

%     Chen and Hwang conjectured that this theorem could be extended.

%     \begin{definition}{Nearly Balanced}{}
%         A bracket is \i{nearly balanced} if no team receives more than one bye.
%     \end{definition}

%     \begin{conj}{}{trnbbo}
%         All totally randomized nearly balanced brackets are ordered.
%     \end{conj}

%     Conjecture \ref{conj:trnbbo} remains open, but would be very powerful if true: unlike balanced brackets, there is a nearly balanced bracket for every number of teams, if the conjecture held, then total randomization would be a format that can provide orderedness to arbitrary numbers of teams $n$ without having $O(n)$ rounds, as the traditional and reseeded options do.

%     Of course, this orderedness does not come without drawbacks. For one, the randomization makes the orderedness feel a bit cheap: once the randomization is complete, before any games have even been played, the orderedness is lost. (Compare to the proper and reseeded ordered brackets, which maintain their orderedness throughout the whole tournament.)

%     But secondly, total randomness has the undesirable property that it might make for some very lopsided and anti-climatic brackets. It could be that top-two teams, whom everyone wants to see face off in the championship game, are set to play each other in the first round!

%     To fix this, we define a new class of cohort randomized brackets: \i{chalk randomized brackets}.

%     \begin{definition}{Chalk Randomized Brackets}{}
%         The \i{chalk randomized bracket} of the $n$-team signature $\A = \bracksig{a_0; ...; a_r}$ is the cohort randomized bracket parameterized by $(\A, \bracktier{b_1 + b_2 + ... + b_r + 1}),$ where $b_i$ is the number of games being played in round $i$ of $\A$.
%     \end{definition}

%     Chalk randomization solves the second issue with total randomness: that good teams might meet earlier than desired.

%     \theo{}{
%         If a chalk randomized bracket goes chalk, after each round, the $m$ remaining teams will be the $m$ top seeds.
%     }{
%         Let $\A = \bracksig{a_0; ...; a_r}$. We proceed by induction on $r.$ If $r = 0$, then $\A = \bracksig{1}$, and so the theorem holds. For any other $r$, in the first round, $a_0$ teams are playing $b_1 = a_0 / 2$ games. However, the $b_1$ lowest seeds in the tournament are shuffled between themselves, so none of them can play each other. Because the tournament goes chalk, they will all lose, so after the first round, the $n - b_1$ remaining teams are the top $n - b_1$ top seeds. Further, we are left with the chalk randomized bracket of signature $\bracksig{a_0/2 + a_1; a_2; ...; a_r}$, so the theorem holds by induction.
%     }{}

%     The chalk randomizations of balanced brackets have particularly nice forms: the chalk randomization of an $r$-round balanced bracket is parameterized by $$(\bracksig{2^r; 0; ...; 0}, \bracktier{2^{r-1} + 2^{r-2} + ... + 2^{0} + 1}).$$

%     This is very similar to the Wimbledon format, which employs a chalk randomized balanced bracket but merges the first two cohorts. Wimbledon even uses this scheme in their single's tournaments, which are on 128 teams and are parameterized by $$(\bracksig{128; 0; ...; 0}, \bracktier{96 + 16 + 8 + 4 + 2 + 1}).$$

%     Chalk randomized balanced brackets are of a nice enough form that we can show their orderedness.

%     \theo{}{
%         All chalk randomized balanced brackets are ordered.
%     }{
% %proof_needed
%     }{}

%     However, the natural extension, that all chalk randomized brackets are ordered, remains open.

%     \begin{conj}{}{}
%         All chalk randomized brackets are ordered.
%     \end{conj}

%     Finally, the Wimbledon formats are not technically chalk randomized, as they merge the first two cohorts. However, we conjecture that they too are ordered.

%     \begin{conj}{}{}
%         The Wimbledon formats are ordered.
%     \end{conj}

%     Ultimately, cohort randomization is a useful tool that uses randomness to generate ordered formats for a given signature (balanced ones in particular), when traditional and reseeded brackets were unable to do so. As we discussed, however, this orderedness can feel a bit fake, as the format is orderedness only over the randomness in the starting line selection: as soon as the bracket has been set, the format is no longer ordered.

%     Cohort randomization can still be a valuable tool in cases where many tournaments are being played and rematches would like to be avoided (as in the ATP), cases where a random distribution of advantages might be preferred to a disordered distribution of advantages (as in March Madness), or cases when the true seeding is unknown (and so total randomization is a more accurate model of the format).
    
%     But it is certainly not a completely satisfying tool for generating ordered brackets from arbitrary signatures. In fact, the question of whether such a tool exists at all remains an open one, and we will conclude this chapter without an answer. Instead, we will explore other kinds of formats all together, such as \i{multibrackets}, \i{round robins}, and \i{pools}.
%     %update list of other formats


%     %augment to tiered signature for non-weakly respectful things


%     %-tennis seeding was to avoid rematches


%     %Orderedness Results:
%     %--Literal Tennis is ordered (manual)
%     %--For balanced bracket: 2^n-1+1 - 2^n is ordered (ref)
%     %--Tiers based on chalky number of rounds survived is ordered (conj)
    
%     %--Complete randomization is not always ordered (ref)
%     %--But it is for balanced brackets (ref)
%     %--And for nearly (or nearly, nearly, etc) (conj)
    
%     %--Tiers by number of byes is ordered (conj)
%     %--Finally, if something is ordered then 2^n of those brackets with tiers of size 2^n is ordered.
%     %--General result about what makes something non-ordered (conj)
% }