\sub {

    Although tournaments have been in use for as long as humans have played sports and games, their formal study is relatively underdeveloped. We look to rectify this, presenting a monograph that combines the formal analysis of tournament design with the human element of competition to develop novel language for talking about tournaments as well a number of new results about their properties.

    We start by introducing the \i{bracket signature}, a new system for compressing an arbitrarily complex bracket with any number of teams into a single succinct signature, allowing for the easy communication of bracket formats between parties. We then observe that not all brackets seedings are created equal and define a \i{proper seedin}, which has a number of desirable properties that ensure virtually every league that uses a bracket uses a proper one. Finally, we establish the \i{fundamental theroem of brackets}: while multiple brackets can share a signature, there is exactly one bracket with each signature that admits a proper seeding, and it admits exactly one proper seeding.

    We then move on to a proof of Edwards's Theorem, which categorizes which brackets have the property that better teams are always more likely to win. (This property is know as being \i{ordered}). Somewhat surprisingly, most brackets are not ordered, a fact that Edwards \cite{montana} first proved in 1991. Of course, Edwards did not have access to the fundamental theorem when he published his thesis: with it equipped, we construct a much simpler and more direct proof that establishes a few other generalizable results along the way.
    
    Since Edwards's publication, there have been numerous attempts to construct a bracket-like format that is always or at least more often ordered. Hwang \cite{reseeding} published a proof that \i{reseeded brackets} do this, but we show that his proof was incorrect and derive an analogue to Edwards's Theorem for reseeded brackets implying that most of them are not ordered. We then show that the only yet discovered ordered bracket scheme is neither deterministic nor exciting, and conjecture that this is not just due to a lack of looking.

    In the following chapter, we consider a number of wildly different formats used by various leagues across space, time, and sport (including, but not limited to, \i{consolation brackets}, \i{semibrackets}, \i{linear multibrackets}, \i{swiss systems}, \i{Page-McIntyre systems}, and \i{double-elimination}), before unifying them all under a single umbrella. We use this unification to establish what concepts like signatures and properness might look like for these formats, as well as prove a number of key results about the number of such formats, their efficiency, and their accuracy.
    
    Over the course of this work, we analyze 16 tournament formats in use by leagues, as well as 42 constructed for the purpose of analysis. We prove 37 theorems, of which 28 are novel (including one that disproves the Hwang's theorem). We hope that this thesis will aid developing and established sports leagues alike in the design of their tournaments, as well as serve as a jumping off point for future work to develop the formal field tournament design.




}



    
    % BS 7 2
    % PB 5 1
    % OB 5 1
    % ET 6 0
    % RB 3 2
    % RA 3 1
    % T 19 7
    
    % CB 5 3
    % DB 4 1
    % LM 1 0
    % P  3 0
    % R  7 0
    % EM 4 0
    % SS11 1
    % NL 5 4
    % T 49 9

    % T 58 16



