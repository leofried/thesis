\sub {

    The theory of tournament design is a relatively underdeveloped one. Querying \i{tournament design theory} in Google Scholar at the time of writing returns a total of 152,000 hits, compared to 0.6, 2.1, 2.2, and 5.7 million for \i{auction design theory}, \i{sorting design theory}, \i{voting design theory}, and \i{market design theory}, respectively. Despite this, tournaments play a large role in society: 19 of the top 20 all-time television broadcasts, ranked by American viewers, are the Super Bowl \cite{super}, the culmination of the tournament that is the National Football League. Similar dominance of tournaments is reported for other countries, though of course the tournament itself is different, with the Olympics, World Cup, and Champions League Finals topping the charts. Further, over the last decade, a wide range of new sports and games have been rising to prominence, from more niche sports like ultimate frisbee and quadball, to esports like League of Legends and Rainbow Six Seige. Even chess has been seeing a recent rise in popularity. As each of these sports and games continue to grow and their leagues develop, the importance of designing an effective tournament will only increase.

    While tournaments themselves have been conducted for as long as humans have competed, the formal theory of tournament design was born out of the study of paired comparisons, a field that began in 1927 with Thurstone's \i{A Law of Comparative Judgement} \cite{thurstone}. Thurstone was a psychologist investigating how individuals rank a collection of objects on some axis (weight, beauty, excellence, etc), while only being able to examine two of the objects at a time. The similarity to the problems of tournament design is clear, though Thurstone did not draw the connection.

    In 1963, David wrote \i{The Method of Paired Comparisons} \cite{stochastic}, aiming to gather all the theory that had been developed about paired comparisons, as well as several contributions of his own, into a single monograph. At the time of publication, the field was still viewed through Thurstone's psychological lens, rather than the lens we will use involving teams competing in a game or sport. Where we will say ``two teams play a game,'' David says ``a single judge must chose between two objects.'' Still, the formalizations of the problems are equivalent.

    In the years following David's work, the field of tournament design came into its own, with various authors examining the efficacy and fairness of a wide range of tournament formats, most commonly either round-robins or knockout tournaments. Much of the work at the time, however, was an analysis of what happens when a specific set of teams (or a narrow class of sets of teams) takes part in a specific tournament design, rather than anything more general.

       % WHAT SHOULD BE IN THIS PARAGRAPH??
       % Progression to edwards theorem?
       % Progression of randomness?
       % Guy whose proof was wrong (reseeding)?
       % tournaments in combo being rrs?

    % Two particularly important papers of this era were Maurer's \i{On Most Effective Tournament Plans with Fewer Games than Competitors} \cite{maurer} which was the first analysis of different bracket shapes, rather than just balanced ones, and Horen and Riezman's \i{Comparing Draws for Single Elimination Tournaments} \cite{four_eight_ordered} which was the first to analyze what happens to specific bracket shapes on a wide class of range of possible team strengths.

    In 1991, Edwards submitted his doctoral thesis, \i{The Combinatorial Theory of Single-Elimination Tournaments} \cite{montana}, the single most complete analysis of brackets that has been published to date. Edwards examined the full space of brackets, counting and cataloging them, before completely determining the set that are ordered, which we will see is both a natural and desirable property. Edward's Theorem, after which Section \ref{sec:Edwards's Theorem} is named, was first proved in that thesis.

    Since then, the field has become much more statistical, with most of the analysis being done by the way of Monte Carlo simulations. Dabney's \i{Tourney Geek} \cite{geek}, for example, evaluates various tournament designs based on several different statistical measurements of fairness that are estimated via simulation.
    
    In this thesis, we return to the type of study conducted by Edwards: proofs of claims about the outcomes of various tournament designs, rather than statistical results. We will work from first principles, beginning with the definition of a game and a tournament format, constructing various specific classes of formats, and then examining those formats and the properties they might have. Like most studies in the field of tournament design, we are \i{game-ambivalent}. We abstract away the underlying game or sport: our results apply to football as well as they will to chess as well as they will to competitive rock-paper-scissors.
}
