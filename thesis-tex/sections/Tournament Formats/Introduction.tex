\sub {

    The theory of tournament design is a relatively underdeveloped one. Querying \i{tournament design theory} in Google Scholar at the time of writing returns a total of 152,000 hits, compared to 0.6, 2.1, 2.2, and 5.7 million for \i{auction design theory}, \i{sorting design theory}, \i{voting design theory}, and \i{market design theory}, respectively. Despite this, tournaments play a large role in society: 19 of the top 20 all-time television broadcasts, ranked by American viewers, are the Super Bowl \cite{super}, the culmination of the tournament that is the National Football League. Similar dominance of tournaments is reported for other countries, though of course the tournament itself is different, with the Olympics, World Cup, and Champions League Finals topping the charts. Further, over the last decade, a wide range of new sports and games have been rising to prominence, from more niche sports like ultimate frisbee and quadball, to esports like League of Legends and Rainbow Six Seige. Even chess has been seeing a recent rise in popularity. As each of these sports and games continue to grow and their leagues develop, the importance of designing an effective tournament will only increase.

    While tournaments themselves have been conducted for as long as humans have competed, the formal theory of tournament design was born out of the study of paired comparisons, a field that began in 1927 with Thurstone's \i{A Law of Comparative Judgement} \cite{thurstone}. Thurstone was a psychologist investigating how individuals rank a collection of objects on some axis (weight, beauty, excellence, etc), while only being able to examine two of the objects at a time. The similarity to the problems of tournament design is clear, though Thurstone did not draw the connection.

    In 1963, David wrote \i{The Method of Paired Comparisons} \cite{stochastic}, aiming to gather all the theory that had been developed about paired comparisons, as well as several contributions of his own, into a single monograph. At the time of publication, the field was still viewed through Thurstone's psychological lens, rather than the lens we will use involving teams competing in a game or sport. Where we will say ``two teams play a game,'' David says ``a single judge must chose between two objects.'' Still, the formalizations of the problems are equivalent.

    In the years following David's work, the field of tournament design came into its own, with various authors examining the efficacy and fairness of a wide range of tournament formats, most commonly either round-robins or knockout tournaments. Much of the work at the time, however, was an analysis of what happens when a specific set of teams (or a narrow class of sets of teams) takes part in a specific tournament design, rather than anything more general.

       % WHAT SHOULD BE IN THIS PARAGRAPH??
       % Progression to edwards theorem?
       % Progression of randomness?
       % Guy whose proof was wrong (reseeding)?
       % tournaments in combo being rrs?

    % Two particularly important papers of this era were Maurer's \i{On Most Effective Tournament Plans with Fewer Games than Competitors} \cite{maurer} which was the first analysis of different bracket shapes, rather than just balanced ones, and Horen and Riezman's \i{Comparing Draws for Single Elimination Tournaments} \cite{four_eight_ordered} which was the first to analyze what happens to specific bracket shapes on a wide class of range of possible team strengths.

    In 1991, Edwards submitted his doctoral thesis, \i{The Combinatorial Theory of Single-Elimination Tournaments} \cite{montana}, the single most complete analysis of brackets that has been published to date. Edwards counting and cataloging the full space of brackets, defined the property orderedness, which we will soon see is a natural and desirable property, and completely determined which brackets are ordered. Edward's Theorem, after which Section \ref{sec:Edwards's Theorem} is named, was first proved in that thesis.

    Since then, the field has become much more statistical, with most of the analysis being done by the way of Monte Carlo simulations. Dabney's \i{Tourney Geek} \cite{geek}, for example, evaluates various tournament designs based on several different statistical measurements of fairness that are estimated via simulation.
    
    In this thesis, we return to the type of study conducted by Edwards: proofs of claims about the outcomes of various tournament designs, rather than statistical results. We will work from first principles, beginning with the definition of a game and a tournament format, constructing various specific classes of formats, and then examining those formats and the properties they might have. Like most studies in the field of tournament design, we are \i{game-ambivalent}. We abstract away the underlying game or sport: our results apply to football as well as they will to chess as well as they will to competitive rock-paper-scissors.

    In this way, the field of tournament design and the field of sorting theory are quite similar: the types of questions posed in the fields are nearly identical as well. In both cases, the designer is given a list of objects (teams), may make an arbitrary number of comparisons (games), and then must output a sorting (champion). There are, however, a number of differences that separate the fields.

    The first difference is that of noise. The sorting theorist works with the guarantee that if two objects are compared twice, the comparison will give the same result both times. For this reason, the sorting theorist often finds it wasteful to compare the same pair of teams more than once. But the tournament theorist's job is much harder, as team performance is noisier. When two teams play, there is no guarantee that the better team will win, and when they play twice, there is no guarantee that the same team will win both times.

    The second difference is that of accuracy. An algorithm submitted by the sorting theorist is required to correctly sort any list of objects, otherwise it is not a sorting algorithm. The tournament theorist is under no such constraints: the noise makes such an algorithm impossible. Thus algorithms like ``randomly select a winner'' and ``play lots of games and then declare the team with the fewest wins champion'' are valid tournament designs, even if they are (probably) not particularly good ones.

    The third difference is that of priors. While the sorting theorists typically begins their algorithms with no priors on the set of objects, tournament theorists are often given a ``seeding'' of teams, identifying which teams are judged to be better. This seeding can be varyingly accurate: in some cases the tournament theorists begins their algorithm with very strong priors, while in others the seeding provides minimal information.

    The fourth difference is that of fairness. The sorting theorist is working with a set of lifeless objects whose feelings will not be hurt based on the algorithm, freeing the sorting theorist to focus only on the task of accurately sorting the objects. The tournament theorist, on the other hand, must appeal to the sense of fairness held by the competitors: in many cases, fairness is a more important consideration than accuracy. 
    
    The final difference is that of viewership. The sorting theorist works in private, comparing objects and gathering data until a sort can be published. The tournament theorist, on the other hand, works in front of an audience, who are looking not just for an accurate tournament, but for an exciting one: the NCAA College Basketball Tournament, is a classic example, as we will soon see, of a tournament that is not very accurate but none the less very exciting for viewers.

    Still, there is a lot of overlap between the two fields. The definitive sorting theory text, Knuth's \i{The Art of Computer Programming: Sorting and Searching} \cite{knuth} often used the tournament design theory language of teams and games when presenting various algorithms. We, too, borrow from the field of sorting theory: in particular the concept of a \i{sorting network}.

    Sorting networks, first patented by Armstrong, Nelson, and O'Connor \cite{pat}, are sorting algorithms with the additional property that, after two objects $a$ and $b$ are compared, the remaining comparisons are identical no matter the result of the comparison between $a$ and $b$, except with every instance of $a$ replaced with $b$, and every instance of $b$ replaced with $a$. Knuth's text has a section about the properties and space of sorting networks.

    This thesis will examine networked tournament formats, that is, tournament formats with this networking property. These formats are a particularly nice set of formats to study. For one thing, the networking property is particularly useful in aiding the study. But also, many tournament formats in use in the real world, most notably the bracket, are networked, giving our study applications to many tournaments and leagues across many sports. We begin our analysis in \bsec{Definitions}, where we formally develop the notions of games, matchup tables, tournament formats, and networked tournament formats, setting the stage for the rest of the thesis.
    
    \bcha{Brackets} focuses on \i{brackets}: a kind networked format with the additional restriction that teams are eliminated after their first loss, and games are played until only a single undefeated remains. We note that a bracket is defined by its \i{shape}, the binary tree that determines the matchup between game winners, and its \i{seeding}, which tells each team which node of the tree to start in.

    We begin with the shape in \bsec{Bracket Signatures}, where we define the \i{bracket signature}, a compression of the the shape of a bracket into a list of natural numbers specifying how many teams get each number \i{byes} (that is, how many games each team must win in order to win the tournament). We prove that an arbitrary list satisfies a specific formula if and only if it is a bracket signature for some bracket. Finally we introduce the notion of a \i{balanced} bracket, one with no byes in which every team is tasked with winning the same number of games.

    In \bsec{Proper Brackets}, we move on to the \i{seeding} of a bracket, observing that in real tournaments, seedings are used to give advantages to the better and more deserving teams. We formalize this practice into that of a \i{proper seeding}, and then prove the \i{fundamental theorem of brackets}: there is exactly one proper bracket with each bracket signature.

    \bsec{Ordered Brackets} introduces Edwards's \cite{montana} notion of an \i{ordered bracket}, a bracket in which a team's odds of winning the format monotonically increases with the skill of the team. We show that all ordered brackets are proper, and set the stage for Edwards's Theorem, which fully categorizes the space ordered brackets, by looking at some simple brackets and determining their orderedness.

    While the key results from the previous three sections concerning properness and signatures were all novel, \bsec{Edwards's Theorem} is dedicated to a proof of its namesake theorem, which was of course first proved in \i{The Combinatorial Theory of Single-Elimination Tournaments}. Still, we offer a much quicker proof of the statement: we first make use of the fundamental theorem to establish two novel lemmas that relate the orderedness of brackets to the sub-brackets that comprise them, before then using the lemmas to derive the theorem.

    Edwards's Theorem turns out to be quite constraining on the space of ordered brackets: the balanced brackets for three or more rounds (eight or more teams) are not ordered. This can be quite disturbing given that one of the primary reasons for using a bracket over other tournament formats is that they can crown a champion in only a logarithmic number of rounds: requiring orderedness makes this impossible. 

    We spend the next two sections attempting to solve this problem. Our first attempt, in \bsec{Reseeded Brackets}, is to use \i{reseeding}, a modification to brackets where after each round, the matchups are rearranged to pair the top seeds with the bottom seeds. Hwang \cite{reseeding} actually published a proof that reseeding allows for balanced ordered brackets for any number of rounds. Unfortunately, we find that his proof was incorrect, and using a nearly identical process as our new proof of Edward's Theorem, complete determine the space of ordered reseeded brackets. Balanced reseed brackets, too, are ordered only for two or fewer rounds.

    Finally, in \bsec{Randomization}, we attempt a second approach to the problem presented by Edward's Theorem by randomizing which teams go where in the bracket. We cite Chen and Hwang's \cite{totally_random_balanced} proof that total randomization does in fact allow for a balanced ordered brackets of arbitrary size. Unfortunately, total randomization can lead to \i{unexciting} formats, where all the best matchups are played very early on. We also consider Wimbledon-style randomization, which ensures that these matchups are delayed until the later rounds, but ultimately show that they too are not ordered for more than two rounds.

    Thus the only balanced ordered \i{knockout tournament} (that is, bracket-like) format for more than two rounds that we have located is the totally randomized one, which has the dual undesirable properties of being unexciting and non-deterministic. We conclude the chapter by asking the two natural open questions: does there exist a balanced ordered knockout-tournament for arbitrary numbers of rounds that is deterministic, and does there exist a balanced ordered knockout-tournament for arbitrary numbers of rounds that is exciting. We pessimistically conjecture that the answer to both questions is no.

    
    

    }


%index 
% --> say more about jargon
%explain edwards more
%explain my stuff more too
%more in depth table of contents
%define/address better in Sorting Networks
%orderedness -> fairness
%send first chapter over
%better definition of network in intro
%gameplay functions vs matchup table clarification
%better names for sections




%conclusion:
%glossary of definitions
%glossary of formats
%future work



