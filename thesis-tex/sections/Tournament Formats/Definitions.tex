\sub{

%matchuo tables are top half and pij = ti vs tj --> this needs fleching obviously

\begin{definition}{Gameplay Function}{}
    A \i{gameplay function} $g$ on a list of teams $\T = [t_1, ..., t_n]$ is a nondeterministic function $g : \T \times \T \to \T$ with the following properties:
        \begin{itemize}
            \item $\P{g(t_i, t_j) = t_i} + \P{g(t_i, t_j) = t_j} = 1.$
            \item $\P{g(t_i, t_j) = t_i} = \P{g(t_j, t_i) = t_j}.$
        \end{itemize}
\end{definition}

A gameplay function represents a process in which two teams compete in a game, with one of them emerging as the winner. This model simplifies away effects like home-field advantage or teams improving over the course of a tournament: a gameplay function is fully described by a single probability for each pair of teams in the list.

\begin{definition}{Playing, Winning, and Losing}{}
    When $g$ is queried on input $(t_i, t_j)$ we say that $t_i$ and $t_j$ \i{played a game}. We say that the team that got outputted by $g$ \i{won}, and the team that did not \i{lost}.
\end{definition}

The information in a gameplay function can be encoded into a \i{matchup table.}

\begin{definition}{Matchup Table}{}
    The \i{matchup table} implied by a gameplay function $g$ on a list of teams $\T$ of length $n$ is a $n$-by-$n$ matrix $\M$ such that $\M_{ij} = \G{t_i}{t_j}.$ (We adopt a convention that for all $t$, $\G{t}{t} = 0.5.$)
\end{definition}

For example, let $\T = $[Favorites, Rock, Paper, Scissors, Conceders$]$, and $g$ be such that the Conceders concede every game they play, the Favorites are $70\%$ favorites against Rock, Paper, and Scissors, and Rock, Paper, and Scissors match up with each other as their name implies. Then the matchup table would look like so:

\begin{figg}{The Matchup Table for $(\T, g)$}{}
    \begin{center}
        \begin{tabular}{ c | c c c c c}
        & Favorites & Rock & Paper & Scissors & Conceders\\
        \hline
        Favorites & 0.5 & 0.7 & 0.7 & 0.7 & 1.0\\
        Rock      & 0.3 & 0.5 & 0.0 & 1.0 & 1.0\\
        Paper     & 0.3 & 1.0 & 0.5 & 0.0 & 1.0\\
        Scissors  & 0.3 & 0.0 & 1.0 & 0.5 & 1.0\\
        Conceders & 0.0 & 0.0 & 0.0 & 0.0 & 0.5
        \end{tabular}
    \end{center}
\end{figg}

\theo{}{
    If $\M$ is the matchup table for $(\T, g)$, then $\M + \M^T$ is the matrix of all ones.
}{
    $(\M + \M^T)_{ij} = \M_{ij} + \M_{ji} = \G{t_i}{t_j} + \G{t_j}{t_i} = 1.$
}{matchup_diag}

Theorem \ref{th:matchup_diag} tells us that a matchup table is defined by the entries below the diagonal, so to reduce busyness we will often display only those entries.

\begin{figg}{The Matchup Table for $(\T, g)$}{}
    \begin{center}
        \begin{tabular}{ c | c c c c c}
        & Favorites & Rock & Paper & Scissors & Conceders\\
        \hline
        Favorites & &  &  &  & \\
        Rock      & 0.3 & & &  & \\
        Paper     & 0.3 & 1.0 & & & \\
        Scissors  & 0.3 & 0.0 & 1.0 & & \\
        Conceders & 0.0 & 0.0 & 0.0 & 0.0 &
        \end{tabular}
    \end{center}
\end{figg}

\begin{definition}{Tournament Format}{}
    A \i{tournament format} is an algorithm that takes as input a list of teams $\T$ and a gameplay function $g$ and outputs a champion $t \in \T.$ %actually not just a champion but
\end{definition}

We use a gameplay function rather than a matchup table in the definition of a tournament format because a tournament format cannot simply look at the matchup table itself in order to decide which teams are best. Instead, formats query the gameplay function (have teams play games) in order to gather information about the teams. That said, matchup tables will often be useful in our \i{analysis} of tournament formats. 

We also introduce some shorthand to help make notation more concise.

\begin{definition}{$\W{\A}{t}{\T}$}{}
    $\W{\A}{t}{\T}$ is the probability that team $t \in \T$ wins tournament format $\A$ when it is run on the list of teams $\T$.
\end{definition}

%also do \WW --> make note that not more than m teams can granted to m, so co-champions each come in second, etc.

% \begin{definition}{$\G{t_1}{t_2}$}{}s
%     $\G{t_1}{t_2}$ is the random variable that is true when team $t_1$ beats team $t_2$. So $\P{\G{t_1}{t_2}}$ is the probability that team $t_1$ beats team $t_2$ in a game.
% \end{definition}

Finally, we will focus our study on the subset of tournament formats that fulfill the \i{network condition}.

\begin{definition}{Network Condition}{}
    A tournament format fulfills the \i{network condition} if after $t_i$ plays $t_j$, the rest of the format is identical no matter which team won, except for $t_i$ and $t_j$ are swapped.
\end{definition}

(This chapter will be fleshed out but I'm including the important definitions here for the sake of the next chapter.)
}







%Things to talk about:
%--Sorting algorithms
%--Shifted Double eliminations
%--Analytics (excitement)
%--RR, pools, scheduling, tiebreakers etc
%--Ultimate frisbee works
%--space of formats, equality
%--round robin vs bracket
%--what is a tournament: regular season + playoffs is one, but we are considering playoff formats for the most part
%-- thirteen team format?
%--det/sym (all randomness can be done at the beginning)