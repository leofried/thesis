\sub {
% Efficient Multibrackets (how many linear, efficient for n, m)
% Swiss??? + generalized swiss
% Loser's bracket Design (this is basically hte additional' considerations section) -- perhaps this should be second? and then nonlinear third?
}

%[[[
    %Consolation (respectfulness, etc) after Third place games
    %Linear (efficient, swiss, semibrackets)
    %Nonlinear
%]]]




%different chapter title?



% Third-Place Games
% Linear/Non-linear (Page -> Double Elimination -> NBA)
% Efficient Multibrackets (how many linear, efficient for n, m)
% Swiss??? + generalized swiss
% Considerations (proper, respectful, etc)




   
% The comparison between sorting networks and multibrackets (or more generally, between sorting algorithms and tournament formats) is a fascinating one. The problem of tournament design and the space of tournament formats differs from that of sorting algorithms in two important ways.

% First, in tournament design we only have access to a noisy comparison operator: while a sorting algorithm is permitted to depend on the same comparison giving always giving the same results, we get no such luxury. In fact, tournament designers are not even guaranteed that their teams are transitive (although it certainly a useful assumption.)

% Second, while sorting algorithms are required to output a full list and that this full list correctly reflect the ordering of the elements, tournament designed face no such restriction: it is okay to design a tournament that only crowns a champion, or top-three, or whatever, and the best team need not always win. (This may be a desirable property, but it also may not be!)
