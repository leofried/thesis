\sub {

    % In the last section, we saw a number of multibrackets, each of which consisted of a primary bracket and a consolation bracket to determine some of the lower places. While properly seeding the primary bracket using the fundamental theorem of brackets is easy enough, properly seeding the consolation brackets, where teams that lost in the same round are of ``equivalent'' seed is trickier. In this section, we will digress away from multibrackets momentarily and ask the question of what does it mean to properly seed a list of teams when some teams are of equivalent seed.

    % Formally, we want to answer the following question: given some fixed bracket signature, and some list of teams that, rather than being seeded linearly, are grouped into tiers, how do we populate the bracket with those teams in a ``proper'' way.

    % By convention, we label the tiers with letters, with lower seeds having earlier letters and higher seeds having later ones. When the AFC 




    -----------------------------

    A first attempt ty simple copy over the definition of a proper seeding from Section \ref{sec:Proper Brackets}.

    \begin{definition}{Proper Seeding}{}
        A \i{proper seeding} of a bracket is one such that if the bracket goes chalk, in every round it is better to be a higher-seeded team than a lower-seeded one, where: \begin{itemize}
            \item[(1)] It is better to have a bye than to play a game.
            \item[(2)] It is better to play a lower seed than to play a higher seed.
        \end{itemize}
    \end{definition}
    
    This is a great start, but needs one quick tweak: unlike in Chapter \ref{cha:Brackets}, we are not assuming that every seeded team is necessarily even given a spot in the bracket. In the 2015 AFC Asian Cup, for example, only the semifinal losers got a chance at 3rd place, while the first-round losers were eliminated outright. To address this, we had a third bullet to the definition.

    \begin{definition}{Proper Tiered Seeding}{}
        A \i{proper teired seeding} of a bracket is one such that if the bracket goes chalk, in every round it is better to be a higher-seeded team than a lower-seeded one, where: \begin{itemize}
            \item[(1)] It is better to have a bye than to play a game.
            \item[(2)] It is better to play a lower seed than to play a higher seed.
        \end{itemize}
    \end{definition}















}