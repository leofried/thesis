\sub{

\begin{definition}{Round-Robin}{rrdef}
    A \textit{round-robin} is a tournament format in which each team plays each other team once, and then teams are ranked according to how many games they won.
\end{definition}

Round-robins, or close variants, are used in many leagues across many sports, especially during the regular season or qualifying rounds. For example, the 2014 Ivy League Football Regular Season was structured as round-robin. At the conclusion of a round-robin, a league table can be used to display the results and rank the teams.

\begin{figg}{2014 Ivy League Football Regular Season}{}
    \centering
    \begin{tabular}{| c | c | c | c | c |}
        \hline
        Rank & Team & Games & Wins & Losses\\ \hline
        1 & Harvard & 7 & 7 & 0\\ \hline
        2 & Dartmouth & 7 & 6 & 1\\ \hline
        3 & Yale & 7 & 5 & 2\\ \hline
        4 & Princeton & 7 & 4 & 3\\ \hline
        5 & Brown & 7 & 3 & 4\\ \hline
        6 & Penn & 7 & 2 & 5\\ \hline
        7 & Cornell & 7 & 1 & 6\\ \hline
        8 & Columbia & 7 & 0 & 7\\ \hline
    \end{tabular}
\end{figg}

At the end of an $n$-team round-robin, each team has played each other team once, for a total of $n-1$ games. There are $n$ possible records a team could have after playing $n-1$ games, so it is possible for each team to end the tournament with a different record: the 2014 Ivy League Football Regular Season has this property.

However, this is far from guaranteed: consider the 2019 Big 12 Football Regular Season (strangely enough, in 2019 the Big 12 had only ten teams).

\begin{figg}{2014 Ivy League Football Regular Season}{}
    \centering
    \begin{tabular}{| c | c | c | c | c |}
        \hline
        Rank & Team & Games & Wins & Losses\\ \hline
        1 & Harvard & 7 & 7 & 0\\ \hline
        2 & Dartmouth & 7 & 6 & 1\\ \hline
        3 & Yale & 7 & 5 & 2\\ \hline
        4 & Princeton & 7 & 4 & 3\\ \hline
        5 & Brown & 7 & 3 & 4\\ \hline
        6 & Penn & 7 & 2 & 5\\ \hline
        7 & Cornell & 7 & 1 & 6\\ \hline
        8 & Columbia & 7 & 0 & 7\\ \hline
    \end{tabular}
\end{figg}

\;\\\;\\\;\\\;\\\;\\\;\\\;\\\;\\\;\\\;\\

Its natural to think of Definition \ref{def:rrdef} as saying that after all of the games are played, a partial ordering is imposed on the teams such that $t_1 > t_2$ if and only if $t_1$ won more games than $t_2.$ However, we in the tournament design buisness are often interested in assigning total orders to teams at the conclusion of a tournament: if two (or more) teams finish the tournament with the most game wins, who wins the tournament? This is where \textit{tiebreakers} come in.

\begin{definition}{Tiebreaker}{}
    A \textit{tiebreaker} is an algorithm that takes a set of teams who just played a round robin and are incomparable according to some partial ordering and outputs a partial ordering on those teams.
\end{definition}

By using a sequence of tiebreakers, then, we can take the partial ordering on the teams guaranteed to us by the definition of a round robin, sand equentially apply tiebreakers to varius sets of incomparable teams until we get a total ordering.
}