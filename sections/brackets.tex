\documentclass[../main.tex]{subfiles}

\begin{document}

\section {Brackets} 

\begin{definition}{Bracket}{}
A \textit{bracket} is a tournament format in which teams are first placed in the leaves of a binary tree, and then games are successively played between teams in nodes that share a parent, placing the winner of each game in the shared parent node. At the end of the tournament, the entire tree will be filled out, and the team that has been in the root of the tree (which is the only team that has not yet lost a game) is declared champion.
\end{definition}

Brackets are traditionally stylized like so:
\fig{1}{2023 College Football Playoff Empty.png}{The 2023 College Football Playoff, Start}{}
By the end of the tournament, the bracket would look like this:
\fig{1}{2023 College Football Playoff Full.png} {The 2023 College Football Playoff, End}{}
In the first round, Georgia played Ohio State, and Michigan played TCU. Georgia and TCU won their respective games, so they advanced to the next round. Then Georgia beat TCU, winning the tournament.

\subsection{Balanced Brackets}

The 2023 College Football Playoff has a special property that not all brackets have: it is \textit{balanced}. 
\begin{definition}{Bye}{}
    When a team doesn't have to play during a certain round of a bracket, we say that team has a \textit{bye}.
\end{definition}
\begin{definition}{Balanced Bracket}{}
    A \textit{balanced bracket} is a bracket with no byes.
\end{definition} 

The 2023 West Coast Conference Men's Basketball Tournament, on the other hand, is unbalanced:
\fig{0.8}{2023 West Coast Conference Men's Basketball Tournament.png}{The 2023 West Coast Conference Men's Basketball Tournament}{}



Saint Mary's and Gonzaga each have three byes and so only need to win two games to win the tournament, while Portland, San Diego, Pacific, and Pepperdine need to win five. Unsurprisingly, this format conveys a massive advantage to Saint Mary's and Gonzaga, but this was intentional: those two teams were being rewarded for doing the best during the regular season.

In many cases, however, it is undesirable to grant advantages to certain teams over others. One might hope, for any $n$, to able to construct a balanced bracket for $n$ teams, but unfortunately this is rarely possible.

\theo{}{
There exists an $n$-team balanced bracket if and only if $n$ is a power of two.}{
In a balanced bracket, no byes are assigned, so at the conclusion of every round, there are half as many teams alive as at the beginning of the round. If $n$ is not a power of two, then this process will eventually lead to a non-one odd number of teams remaining, at which point a bye will have to be assigned, meaning the bracket is not in fact balanced.\\

If $n$ is a power of two, however, we can inductively build up a balanced bracket. For $n = 1$, the unique one-team bracket is balanced, and for any other $n$, once we have a balanced bracket for $n / 2$ teams, we can replace each leaf node with  starting line with a play-in game, resulting in an $n$-team balanced bracket.
}{balanced brackets}

Given this, brackets are often not a great option when we want to avoid giving some teams advantages over others. They are a great tool, however, when we want to dole out advantages, for example, after some teams do better during the regular season and ought to be rewarded with an easier path in bracket.

\subsection{Seeded Brackets}
\begin{definition}{Seeding}{}
    \textit{Seeding} is a process in which teams are seeded (ranked) by how good and/or deserving they are, and then higher seeded teams are given an easier schedule in the ensuing tournament.
\end{definition}
To seed a bracket, first assign each of the starting lines an integer between $1$ and $n$. Then, put each team on the line designated by their seed and run the bracket as normal.

As an example, on the left is the an $8$-team bracket used in the 2015 NBA Eastern Conference Playoffs. At the end of the regular season, the top eight teams in the Eastern Conference were seeded and placed into the bracket as shown on the right. Finally, the bracket was played out normally.
\fig{.7}{2015 NBA Eastern Conference Playoffs.png}{2015 NBA Eastern Conference Playoffs}{}

Note that despite this bracket being balanced. the higher seeds are still at advantage: they have an easier set of opponents. Compare $1$-seeded Atlanta, who's first two rounds are versus $8$-seeded Brooklyn and then (most likely) $4$-seeded Toronto, versus $7$-seeded Boston, who's first two rounds are versus $2$-seeded Cleveland and then (most likely) $3$-seeded Chicago. Atlanta's schedule is far easier: despite them having the same number of games to win as Boston, Atlanta will have to play lower seeds in each round than Boston will.

Thus we've identified two ways in which brackets can convey an advantage onto certain teams: by giving them more byes, and by giving them easier (expected) opponents. Not every seeding of a bracket does this: for example, consider the following alternative seeding for the 2015 NBA Eastern Conference Playoffs.

\fig{1}{NBA Bad Seeding.png}{An Alternative Seeding of the 2015 NBA Eastern Conference Playoffs}{}

This seeding does a very poor job of rewarding the higher-seeded teams: the $1$- and $2$- seeds are matched up in the first round, while the easiest road is given to the $7$-seed, who plays the $8$-seed in the first round and then (most likely) the $5$-seed in the second. Since the whole point of seeding is to give the higher-seeded teams an advantage, we introduce the concept of a \textit{proper seeding.}

\begin{definition}{Proper Seeding}{}
    A bracket is \textit{properly seeded} if, should that bracket go to chalk (that is, should the higher-seeded team always win), then, in every round, it is better to be a higher-seeded team than a lower-seeded one, where: \begin{itemize}
        \item It is better to still be alive than to have been eliminated.
        \item It is better to have a bye than to be playing a game.
        \item It is better to be playing a lower seed than to be playing a higher seed.
    \end{itemize}
\end{definition}
It is clear that the actual 2015 NBA Eastern Conference Playoffs was properly seeded, while our alternative seeding was not. In fact, most seedings are not proper. In order to catalog those that are we need one more piece of machinery.

\subsection{Bracket Signatures}

\begin{definition}{Bracket Signature}{}
    The \textit{signature} of a bracket is a list of integers of length $r + 1$, where $r$ is the number of rounds in that bracket, such that the $i$th position in the list is equal to the number of teams that play their first game in that round.
\end{definition}

For example, the signature of the 2023 College Football Playoff is $\bracksig{[4; 0; 0]},$ the signature of the 2023 West Coast Conference Men's Basketball Tournament is $\bracksig{[4;2;2;2;0;0]},$ and the signature of the 2015 NBA Eastern Conference Playoffs is $\bracksig{[8; 0; 0; 0]}
.$

\theo{}{
    Let $A = [a_0; ...; a_r]$ be a list of integers.  Then $A$ is a bracket signature if and only if $$\sum_{i=0}^r a_i \cdot \left(\frac{1}{2}\right)^{r - i} = 1.$$}{
        Let $A$ be the signature for some bracket. Assume that every game in the bracket was a coin flip, and consider each team's probability of winning the tournament. A team that gets $i$ byes must win $r-i$ games, and so will win the tournament with probability $\left(\frac{1}{2}\right)^{r - i}.$ For each $i \in \{0, ..., r\}$, there are $a_i$ teams that got that many byes, so (because any two teams winning are mutually exclusive) $$\sum_{i=0}^r a_i \cdot \left(\frac{1}{2}\right)^{r - i}$$ is the probability that one of the teams wins, which is just $1.$\\

        We prove the other direction by induction on $r$. If $r = 0$, then the only list with the desired property is $[1]$, which is the signature for the unique $1$-team bracket. For any other $r$, first note that $a_0$ must be even: if it were odd, then \begin{align*}
            \sum_{i=0}^r a_i \cdot \left(\frac{1}{2}\right)^{r - i}
            &= \frac{1}{2^r} \cdot \sum_{i=0}^r a_i \cdot 2^i\\
            &= \frac{1}{2^r} \cdot \left(a_0 + 2 \sum_{i=1}^r a_i \cdot 2^{i-1}\right)\\
            &= k/2^r &\textrm{for some odd $k$}\\
            &\neq 1.
        \end{align*}
        Now, consider the signature $B = [a_1 + a_0/2; a_2; ...; a_r].$ By induction, there exists a bracket with signature $B$. But if we take that bracket and replace $a_0/2$ of the starting lines that receive no byes with play-in games, we get a new bracket with signature $A.$
}{signaturesum}

Note that two different brackets might have the same signature: for example, the following two $6$-team brackets each have the signature $[4;2;0;0].$

\fig{.8}{6 team bracket signatures.png}{Two brackets with the same signature}{sixes}

However,

\theo{}{The right bracket in Figure \ref{fig:sixes} does not admit a proper seeding.}{
Proper seedings must give higher-seeded teams more byes than lower-seeded teams, so the two teams with byes must be the 1-seed and the 2-seed. However, in the semifinals, this would match the 1-seed against the 2-seed, meaning that each of them would play a more difficult opponent than the other two semifinalists. Thus the seeding is not proper, and so the bracket admits no proper seedings.}{}

We will see that it is not a coincidence that only one of the two brackets with signature $\bracksig{[4;2;0;0]}$ admits a proper seeding.

\subsection{Proper Brackets}

\begin{definition}{Proper Bracket}{}
    A \textit{proper bracket} is a bracket that admits a proper seeding.
\end{definition}

\theo{}{Each bracket signature admits exactly one proper bracket, which itself admits only one proper seeding.}{
    Let $A = [a_0; ... a_r]$ be an $n$-team bracket signature. We proceed by induction on $r.$ If $r = 0$, then the only possible bracket signature is $[1]$, and it points to the unique one-team bracket, which can only be seeded in one way and is indeed proper.\\

    For any other $r$, note first that, because it is better to have a bye than to be playing, the $a_0$ teams that don't have a first-round bye must be seeds $n - a_0 + 1$ through $n$. Additionally, since higher-seeded teams must have lower-seeded opponents, the first-round matchups must to be $n - a_0 + 1 + i$ vs $n - i$ for $i \in \{0, ..., a_0/2 - 1\}.$\\

    Now, consider the bracket signature $B = [a_1 + a_0/2; a_2; ...; a_r].$ By induction, $B$ admits exactly one proper bracket which admits one proper seeding. Additionally, if the first round of the proper bracket with signature $A$ goes to chalk, we will be left with a bracket with signature $B$ for seeds $1$ through $n - a_0/2.$ This bracket is still subject to the proper seeding constraints, and so must be exactly the the proper bracket and seeding admitted by $B$.\\

    Thus both the first-round matchups and the rest of the bracket are determined, and by combining them we get a proper bracket and seeding with signature $A$, so $A$ admits exactly one proper bracket which itself admits only one proper seeding.
}{signatureproper}

This theorem means that we can refer to the proper bracket $A = [a_0; ...; a_r]$ in a well-defined way, as long as $$\sum_{i=0}^r a_i \cdot \left(\frac{1}{2}\right)^{r - i} = 1.$$

%For the rest of this paper, we will focus on the space of proper brackets rather than brackets in general. In other chapters the word ``bracket" likely refers only to proper brackets.

%(As a quick aside, there are times where a bracket being proper is not as important as it might seem. Two possible reasons are if there is some other factor more important than a proper seeding (like avoiding rematches from earlier in the tournament) or if our seeding is in the form of tiers rather than a rank-order. That said, for convenience as well as prevalence reasons, we will focus just on proper brackets.)

It also gives us a handle on how many (proper) brackets exist for $n$ teams:

\theo{}{There are $P(n)$ $n$-team proper brackets, where $P(n)$ is the number of partitions of $1$ into $n$ powers of $\frac{1}{2}.$}{
    By Theorem \ref{th:signatureproper}, there is one $n$-team proper bracket for each $n$-team bracket signature, and by Theorem \ref{th:signaturesum}, the set of $n$-team bracket signatures the set of lists $A = [a_0; ...; a_n]$ such that $$\sum_{i=0}^r a_i \cdot \left(\frac{1}{2}\right)^{r - i} = 1.$$ However, such lists are in a bijective correspondence with partitions of $1$ into $n$ powers of $\frac{1}{2}$: the number $\left(\frac{1}{2}\right)^{r - i}$ occurs in the partition $a_i$ times.
}{}

The sequence $P(n)$ is detailed in \href{https://oeis.org/A002572}{A002572}.


\subsection{Seeding Issues}

Let's consider the proper bracket $\bracksig{[16; 0; 0; 0; 0]}$, which was used in the 2021 NCAA Men's Basketball Tournament South Region, and is shown here: (Sometimes brackets are drawn in the manner below, with teams starting on both sides and the winner of each side playing in the championship game.)

\fig{1}{2021 NCAA Basketball Tournament South Region.png}{2021 NCAA Men's Basketball Tournament South Region}{}

The definition of a proper seeding ensures that as long as the bracket goes to chalk, it will always be better to be a higher seed than a lower seed. But what if it doesn't go to chalk?

One counter-intuitive fact about the NCAA Men's Basketball Tournament is that it is probably better to be a 10-seed than a 9-seed. (This doesn't violate the proper seeding property because proper seedings only care about what happens if the bracket goes to chalk, which would eliminate both the 9-seed and 10-seed in the first round.) Why? Let's look at who each seed-line matchups against in the first two rounds:

\begin{figg}{2021 NCAA Men's Basketball Tournament 9- and 10-seed Schedules}{}
    \centering
    \begin{tabular}{ c | c c }
         Seed & First Round & Second Round \\
         \hline
         9 & 8 & 1\\
        10 & 7 & 2
    \end{tabular}
\end{figg}

The 9-seed has an easier first-round matchup, while the 10-seed has an easier second-round matchup. However, this isn't quite symmetrical. Because the teams are (most likely) drawn from a roughly normal distribution, the difference in skill between the 1- and 2-seeds is far greater than the difference between the 7- and 8-seeds, implying that the 10-seed does in fact have an easier route than the 9-seed.

Nate Silver \href{https://fivethirtyeight.com/features/when-15th-is-better-than-8th-the-math-shows-the-bracket-is-backward/}{investigated} this matter in full, finding that in the NCAA Men's Basketball Tournament, seed-lines 10 through 15 give teams better odds of winning the region than seed-lines 8 and 9. Of course this does not mean that the 11-seed (say) has a better chance of winning a given region than the 8-seed does, as the 8-seed is a much better team than the 11-seed. But it does mean that the 8-seed would love to swap places with the 11-seed, and that doing so would increase their odds to win the region.

This is obviously not a great state of affairs: the whole points of seeding is confer an advantage to higher-seeded teams, and giving lower-seeded teams an easier route than higher-seeded ones can incentivize teams to lose during the regular season in order to try to get a lower but more advantagous seed.

Unfortunately, there is no conventional seeding method that can fully eliminate these incentives: Theorem \ref{th:signatureproper} showed that for any given bracket signature, there is only proper bracket and proper seeding, and we just found that this seeding can sometime lead to these perverse incentives. If we went with a non-proper seeding, then one of the proper-seeding conditions would fail, leading to preverse incentives for the higher seeds that are less likley to be eliminated and thus covered by those conditions.

However, there are a few unconvential seeding methods that attempt to remove perverse incentives entirely.

\subsection{Reseeding}

Ultimately, the issues outlined by Silver are caused by teams that are seeded in the bottom half of seeds being treated, if they win, as the team that they beat for the rest of the format. If a 11-seed wins in the first-round, they take on the schedule of a 6-seed for the rest of the tournament, while if the 9-seed wins, they take on the schedule of an 8-seed. Given that a 6-seed has an easier schedule than an 8-seed, it's not hard to see why it might be preferable to be a 11-seed rather than a 9-seed.

\textit{Reseeding} (poorly named) fixes this by resorting the match-ups every round: if an 11-seed keeps winning, they will have to play teams according to seed, rather than getting an effecitve upgrade to 6-seed status.

\begin{definition}{Reseeding}{}
    In a \textit{reseeded} bracket, after each round, match-up the highest-seeded team with the lowest-seeded team, second highest vs second-lowest, etc.
\end{definition}

Both National Football League confrences use a $\bracksig{[6; 1; 0; 0]}$ reseeded bracket. If the first-round of the bracket goes to chalk, then it looks just like a normal bracket:

\fig{1}{2023 AFC.png}{2023 National Football League AFC Playoffs}{}

The dotted lines are drawn after the first round of games have been played: if there are some first-round upsets, then the bracket is rearranged to ensure that it still better to be a higher seed rather than a lower seed.

\fig{1}{2023 NFC.png}{2023 National Football League NFC Playoffs}{}

In the NFC, 6-seed New York upset 3-seed Minnesota. Had a conventional bracket been used, the semifinal matchups would have been 1-seed vs 5-seed and 2-seed vs 6-seed: the 2-seed would have had an easier draw than the 1-seed, while the 6-seed would have an easier draw than
 the 5-seed. Reseeding fixes this by matching 6-seed New York is with top-seed Philadelphia, and 2-seed San Fransisco with 5-seed Dallas.

Reseeding is not without its drawbacks. If a bracket uses reseeding, teams and spectators alike don't know who they will play or where their next game will be until the entire previous round is complete. This can be be an espeially big issue if parts of the bracket are being played in different locations on short turn-arounds: in the 2019 NCAA Women's Basketball Tournament, the first two rounds are played over a weekend on the various college campuses of the highest-seeded teams. It would cause problems if teams had to pack up and travel across the country because their oppoenent changed because of reseeding.

In addition, part of what makes the NCAA Basketball Tournament (affectionately known as ``March Maddness'') such a fun spectator experience is that fact that these matchups are known ahead of time. In ``bracket pools,'' groups of fans each fill out their own brackets, predicting who will win each game and getting points based on how many they get right. If it wasn't clear where in the bracket the winner of a given game was supposed to go, this experience would be diminished.

Finally, reseeding gives the top-seed(s) an even greater advantage than they already have: instead of playing against merely the \textit{expected} lowest-seeded team(s) each round, they would get to play agains the \textit{actual} lowest-seeded team(s). In March Madness, ``Cindarella Stories,'' that is, deep runs by low seeds, would become much less common.

In many ways, the NFL conferences are a perfect place to implement reseeding: games are played once a week, giving plenty of time for travel; only seven teams make the playoffs in each, so a huge March Madness-style bracket challenge is unlikely; as a professional league, the focus is far more on having the best team win and protecting Cindarella Stories isn't as important; and because the bracket is only three rounds long, reseeding is only required once.

Other leagues with similar structures might consider adopting forms of reseeding to protect their incentives and competative balance (looking at you, Major League Baseball), but for many leagues, the traditional bracket structure is too appealing to adopt a reseeded one.

\subsection{Randomized Proper Seedings}

RESEEDED THEORY



%Consider which seed-line has an easier road to the championship game: the 8-seed or the 9-seed.

%The 9-seed sees the 8-seed in the first round, before almost certainly playing the 1-seed in the second-round, and then most likely semifinals and then probably matching up with the 2-seed in the championship game.

%The 6-seed sees the 3-seed in the first round, before almost certainly playing the 2-seed in the semifinals and then probably matching up with the 1-seed in the championship game.

%So which of these is better? The 5-seed has an easier first-round matchup than the 6-seed does, but the 6-seed has an easier second-round matchup. (This is possible despite the bracket being properly seeded because neither the 5- nor the 6-seed are ''supposed" to win in the first round.) The 




%\subsection{Multiple Elimination}




\end{document}