\sub {

Let's consider the proper bracket $\bracksig{[16; 0; 0; 0; 0]}$, which was used in the 2021 NCAA Men's Basketball Tournament South Region, and is shown here: (Sometimes brackets are drawn in the manner below, with teams starting on both sides and the winner of each side playing in the championship game.)

\fig{0.8}{2021 NCAA Basketball Tournament South Region.png}{2021 NCAA Men's Basketball Tournament South Region}{}

The definition of a proper seeding ensures that as long as the bracket goes to chalk (that is, higher seeds always beat lower seeds), it will always be better to be a higher seed than a lower seed. But what if it doesn't go to chalk?

One counter-intuitive fact about the NCAA Men's Basketball Tournament is that it is probably better to be a 10-seed than a 9-seed. (This doesn't violate the proper seeding property because 9-seeds have an easier first-round matchup than 10-seeds, and for further rounds, proper seedings only care about what happens if the bracket goes to chalk, which would eliminate both the 9-seed and 10-seed in the first round.) Why? Let's look at whom each seed-line matchups against in the first two rounds:

\begin{figg}{2021 NCAA Men's Basketball Tournament 9- and 10-seed Schedules}{}
    \centering
    \begin{tabular}{ c | c c }
         Seed & First Round & Second Round \\
         \hline
         9 & 8 & 1\\
        10 & 7 & 2
    \end{tabular}
\end{figg}

The 9-seed has an easier first-round matchup, while the 10-seed has an easier second-round matchup. However, this isn't quite symmetrical. Because the teams are (most likely) drawn from a roughly normal distribution, the difference in skill between the 1- and 2-seeds is far greater than the difference between the 7- and 8-seeds, implying that the 10-seed does in fact have an easier route than the 9-seed.

Nate Silver \cite{nate_silver} investigated this matter in full, finding that in the NCAA Men's Basketball Tournament, seed-lines 10 through 15 give teams better odds of winning the region than seed-lines 8 and 9. Of course this does not mean that the 11-seed (say) has a better chance of winning a given region than the 8-seed does, as the 8-seed is a much better team than the 11-seed. But it does mean that the 8-seed would love to swap places with the 11-seed, and that doing so would increase their odds to win the region.

This is not a great state of affairs: the whole point of seeding is confer an advantage to higher-seeded teams, and the proper bracket $\bracksig{[16; 0; 0; 0; 0]}$ is failing to do that. Not to mention that giving lower-seeded teams an easier route than higher-seeded ones can incentivize teams to lose during the regular season in order to try to get a lower but more advantageous seed.

To fix this, we need a stronger notion of what makes a bracket effective than properness. The issue with proper seedings is the problematic assumption that higher-seeded teams will always beat lower-seeded teams. A more nuanced assumption might look like this:

%does seeding maen the ordering of the teams or placement of the teams in the bracket
\begin{definition}{Strongly Stochastically Transitive \cite{stochastic_is_this_paper_even_real}}{}
    A list of teams $[t_1, ..., t_n]$ is \textit{strongly stochastically transitive} if:
    \begin{itemize}
        \item $i < j$ implies that $\P(t_i \textrm{ beats } t_j) \geq 0.5,$ and
        \item For each $i, j, k$ such that $j < k$, $\P(t_i \textrm{ beats } t_j) \leq \P(t_i \textrm{ beats } t_k).$
    \end{itemize}
\end{definition}
A list of teams being strongly stochastically transitive (or SST) captures the intuition that higher-seeded teams ought to be favored against lower-seeded teams, and then each team ought to do better against lower-seeded teams than against higher-seeded ones.

Note that not every set of teams can be arranged to be SST. Consider, for example, the game of rock-paper-scissors. Rock beats paper which beats scissors which beats rock, so no ordering of these ``teams'' will be SST. For our purposes, however, SST will work well enough.

Our new, nuanced alternative a proper bracket is an \textit{ordered bracket}.

\begin{definition}{Ordered Bracket \cite{four_eight_ordered}}{}
    A bracket and seeding are \textit{ordered} if, for any SST list of teams, if $i < j$, then $\P(t_i\textrm{ wins the tournament} )\geq \P(t_j\textrm{ wins the tournament}).$
\end{definition}

In many ways, a bracket being ordered is the strongest thing we can want without making any ``moral'' decisions. It might be unclear (and vary by situation) whether a it is better for a format to almost always declare the most-skilled team the winner, or to give each team roughly the same chance of winning, or anywhere in between. But certainly, better teams should win more, which is what the ordered bracket stipulation requires.

In particular, a bracket being ordered is a stronger claim than it being proper.
\theo{}{Every ordered bracket is proper.}{
    Let $\mathcal{B}$ be an ordered $n$-team bracket with $r$ rounds.\\
    
    Consider a set of teams such that every team wins every game with probability 0.5. This matchup table is SST. A team that plays their first game in the $i$th round will win the tournament with probability $(\frac{1}{2})^{r-i}$, so teams that get more byes will have a higher probability to win the tournament than teams with fewer byes. Thus, higher-seeded teams must have more byes than lower-seeded teams, so in each round, the teams with byes must be the highest-seeded teams that are still alive. Thus, condition (1) is met.\\

    Consider an arbitrary round $i$. Let $t_1, ... t_m$ be the teams that would be playing in this round if the bracket went to chalk so far, ordered by seed. Consider the following SST matchup table on the teams remaining: every team wins every game with probability 0.5, except for games involving $t_m$, who is guarenteed to lose every game they play. Then, each team playing this round will win the tournament with probability $(\frac{1}{2})^{r-i},$ other than $t_m$ who wins with probability $0$ and the team playing $t_m$ who wins with probability $(\frac{1}{2})^{r-i-1}.$ Since that team has a better chance of winning than the other teams playing this round, and $\mathcal{B}$ is ordered, the team playing $t_m$ has to be the highest-seeded team playing this round, which is $t_1.$  We can then build more SSTs, adding the next-lowest seed as also losing every game they play to ensure that they must play the next-highest seed, until we've shown that all the teams playing this round must be matched up according to condition (2).\\

    Thus $\mathcal{B}$ satisfies both conditions, and so is a proper bracket.
}{ordered_proper}

With Theorem \ref{th:ordered_proper}, we can use the language of bracket signatures to describe ordered bracket without concerns over collision. For the next few theorems, we will attempt to describe which bracket signatures induce ordered brackets by first enumerating three categories of ordered bracket and then proving that all other proper brackets are not ordered.

Firstly,

\theo{}{The unique one-team bracket of signature $\bracksig{[1]}$ is ordered.}{There is only one team in the bracket, so the ordered bracket condition is satisfied.}{one_team_ordered}

Next we show that brackets of the following form are ordered:

\fig{1}{ladder_ordered.png}{The Second Category of Ordered Brackets}{ladder_ordered}

\theo{}{
    Brackets of the form of Figure \ref{fig:ladder_ordered} are ordered \cite{montana}.
}{
    Assume the participating teams are SST. We decompose the ordered bracket conditions into two parts: first that the 1-seed has (at least tied for) the best chance of winning, and second that the other seeds have probability to win the bracket in the order of their seed.\\

    First, the 1-seed beats every other team with probability with at least probability 0.5, and if they win their one game, they win the tournament, so the 1-seed wins the tournament with probability at least 0.5, meaning they have (at least tied for) the best chance of the winning.\\

    Next, note that the probability for any other team to win is the probability that they win the sub-bracket multiplied by the probability that they beat the 1-seed. The first list is ordered because the sub-bracket is ordered, and the second list is ordered because the teams are SST, products are also ordered, implying that the rest of the seeds have probability to win the bracket in the order of their seed.\\

    Thus, brackets of the form of Figure \ref{fig:ladder_ordered} are ordered.
}{ladder_ordered_proof}

\begin{corollary}{}{ladder_ordered_signatures}
    If the bracket $[a_0, ..., a_r]$ is ordered, then so is the bracket $[a_0, ..., a_{r-1}, a_r + 1, 0].$
\end{corollary}

\begin{corollary}{}{ladder_ordered_examples}
    The following brackets are ordered:
    \begin{itemize}
        \item $\bracksig{[2; 0]}$
        \item $\bracksig{[2; 1; 0]}$
        \item $\bracksig{[2; 1; 1; 0]}$
        \item $\bracksig{[2; 1; 1; 1; 0]}$
        \item $\bracksig{[2; 1; 1; 1; 1; 0]}$
        \item etc.
    \end{itemize}
\end{corollary}

\fig{.59}{ladder_brackets.png}{The Ordered Brackets Described by Corollary \ref{th:ladder_ordered_examples}}{ladder_brackets}

The brackets pictured in Figure \ref{fig:ladder_brackets} are often referred to as ``ladder brackets,'' referencing the fact that lower-seeded teams must climb up them one rung (game) at a time. Ladder brackets are as punishing a bracket to the lower-seeded teams as one could design, forcing the two lowest-seeded teams to beat every other team in order to win the tournament. On the other hand, by putting the 1-seed just a single game away from winning the bracket, if your teams really are seeded in an SST manner, it does a fantastic job of selecting the best team as the winner.

Finally, let's examine the third category of ordered brackets.

\begin{definition}{The Third Category of Ordered Brackets}{}
    FILL THIS IN
\end{definition}

\lemm{}{
    If two games are played in a given round of an ordered bracket, then the winners of those games must play in the following round.
}{
    Assume for contradiction that an ordered bracket exists in which two games are played in a given and round and the winners of those games do not play each other.
    Let $a, b, c$ and $d$ be the seeds of the four teams who would play in those two games if the bracket went to chalk such that the $a$-seed would play the $d$-seed and the $b$-seed would play the $c$-seed, and such that $a < d$ and $b < c$. 
}{}

% \begin{corollary}{}{}
%     No more than two games can be played in an ordered bracket.
% \end{corollary}


%excersize: proof that [4;0;0] is ordered.
%bracket vs seeding vs list of teams

}