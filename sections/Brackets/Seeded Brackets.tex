\sub{

\begin{definition}{Seeding}{}
    \textit{Seeding} is a process in which teams are seeded (ranked) by how good and/or deserving they are, and then higher seeded teams are given an easier schedule in the ensuing tournament.
\end{definition}
To seed a bracket, first assign each of the starting lines an integer between $1$ and $n$. Then, put each team on the line designated by their seed and run the bracket as normal.

As an example, on the left is the $8$-team bracket used in the 2015 NBA Eastern Conference Playoffs. At the end of the regular season, the top eight teams in the Eastern Conference were seeded and placed into the bracket as shown on the right. Finally, the bracket was played out normally.
\fig{.65}{2015 NBA Eastern Conference Playoffs.png}{2015 NBA Eastern Conference Playoffs}{}

Note that despite this bracket being balanced, the higher seeds are still at advantage: they have an easier set of opponents. Compare $1$-seeded Atlanta, who's first two rounds are versus $8$-seeded Brooklyn and then (most likely) $4$-seeded Toronto, versus $7$-seeded Boston, who's first two rounds are versus $2$-seeded Cleveland and then (most likely) $3$-seeded Chicago. Atlanta's schedule is far easier: despite them having the same number of games to win as Boston, Atlanta will have to play lower seeds in each round than Boston will.

Thus, we've identified two ways in which brackets can convey an advantage onto certain teams: by giving them more byes, and by giving them easier (expected) opponents. Not every seeding of a bracket does this: for example, consider the following alternative seeding for the 2015 NBA Eastern Conference Playoffs.

\fig{1}{NBA Bad Seeding.png}{An Alternative Seeding of the 2015 NBA Eastern Conference Playoffs}{}

This seeding does a very poor job of rewarding the higher-seeded teams: the $1$- and $2$- seeds are matched up in the first round, while the easiest road is given to the $7$-seed, who plays the $8$-seed in the first round and then (most likely) the $5$-seed in the second. Since the whole point of seeding is to give the higher-seeded teams an advantage, we introduce the concept of a \textit{proper seeding.}

\begin{definition}{Proper Seeding}{}
    A bracket is \textit{properly seeded} if, should that bracket go to chalk (that is, should the higher-seeded team always win), then, in every round, it is better to be a higher-seeded team than a lower-seeded one, where: \begin{itemize}
        \item It is better to still be alive than to have been eliminated.
        \item It is better to have a bye than to be playing a game.
        \item It is better to be playing a lower seed than to be playing a higher seed.
    \end{itemize}
\end{definition}
It is clear that the actual 2015 NBA Eastern Conference Playoffs was properly seeded, while our alternative seeding was not. In fact, most seedings are not proper. In order to catalog those that are we need one more piece of machinery.

}